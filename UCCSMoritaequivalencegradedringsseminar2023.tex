\documentclass{beamer}
%\usepackage[all]{xy}
%\usepackage{xypic}
\usepackage{verbatim}


%%% from the paper with Efren and Mark, trying to get \varinjlim to work
\usepackage[]{latexsym,amssymb,amsmath,
%amsfonts,
 amsthm}
\usepackage[all,cmtip,ps]{xy}
%\usepackage{showkeys}
\usepackage{enumitem}







%  From Roozbeh:

 \usepackage{wrapfig}
%\usepackage[dvips]{graphicx}
\usepackage{tikz}
\usetikzlibrary{shapes,shadows,arrows}

\usepackage{times}





%\usepackage[all,poly]{xy}




\usepackage{amsfonts}
\usepackage[mathcal]{eucal}
\usepackage{eufrak}
\usepackage{amssymb}
\usepackage{amsmath}
\usepackage{mathrsfs}







\newcommand{\rmod}[1]{\operatorname{Mod}\text{-}#1}


\newcommand{\grmod}[1]{\operatorname{Gr}\text{-}#1}



\newcommand{\Pp}{\underline{p}}

\newcommand{\SP}{\operatorname{SP}}

\newcommand{\s}{\operatorname{\backslash S}}


\newcommand{\lo}{\mathfrak l}

\newcommand{\no}{\mathfrak n}

%\newcommand{\rmod}[1]{\operatorname{Mod}\text{-}#1}


%\newcommand{\grmod}[1]{\operatorname{Gr}\text{-}#1}


\newcommand{\gr}{\operatorname{gr}}

\newcommand{\hgr}{\operatorname{hgr}}

\newcommand{\RH}[1]{\textcolor{cyan}{{\it Roozbeh}$\rightarrow$ #1}}






\usepackage{venndiagram}



\usepackage[utf8]{inputenc}
\usepackage{tikz}
\usetikzlibrary{backgrounds,calc}





\input xy
\xyoption{all}

%\usetheme{Luebeck}
\usetheme{Berlin}

\usecolortheme{beaver}




%\logo{\includegraphics[height=0.4cm]{UCCS Signature.png}}



%\title[Morita equivalence for graded rings]
{}
%Leavitt path algebras: \\ something for everyone

\author[Gene Abrams   \hspace{2.7in}Morita equivalence for graded rings  ]{
%Gene Abrams
}
\vspace{2in}
\institute[UCCS]{
%University of Colorado at Colorado Springs
}
 \date{
 %Colloquia Patavina
 }



\newcommand{\N}{\mathbb{N}}
\newcommand{\C}{\mathbb{C}}
\newcommand{\Z}{\mathbb{Z}}
\newcommand{\R}{\mathbb{R}}
\newcommand{\coker}{coker}
\renewcommand{\O}{\mathcal{O}}
\newcommand{\VR}{\mathcal{V}(R)}

\begin{document}

%\begin{frame}
	%\titlepage
%\end{frame}





\begin{frame}

\begin{center}

\Huge

{\bf Morita equivalence for  \Huge    \\  

graded rings}


\large

\bigskip

\bigskip


Gene Abrams  

\bigskip

%\includegraphics[width= .4\linewidth]{UCCSSignature.png} 

\bigskip

Colorado Springs Algebra Seminar

\bigskip

April 5, 2023  
\end{center}


\end{frame}


\begin{frame}
\frametitle{Morita equivalence for graded rings}


\begin{center}
This is joint work with 

\bigskip

\Large
Efren Ruiz
\normalsize
\bigskip


\end{center}

\end{frame}


\begin{frame}
\frametitle{Some background and history}

$A,B,R,S,T$ will denote unital associative rings.

\medskip

$\rmod{A}$  denotes the category of right $A$-modules, with $A$-module homomorphisms.   

\bigskip
\bigskip
\pause

%Equivalence of categories.  

%\bigskip
%\pause

 Equivalence of $\rmod{A}$ and $\rmod{B}$.   
 
  
 \end{frame}


\begin{frame}
\frametitle{Some background and history}


Important / guiding example:
For any ring $R$ and positive integer $n$, 
$$\rmod{R} \ \ \mbox{ and } \ \ \rmod{{\rm M}_n(R)}$$
 are equivalent module categories.  
 
  \bigskip
 \pause
 
For instance:      \ \ $\rmod{R} \ \ \mbox{ and } \ \ \rmod{{\rm M}_2(R)}$ \ \   are ``the same" ... 
$$M  \ \ \ \mapsto \ \ \  \begin{pmatrix}
M & M \\
0 & 0 
\end{pmatrix}$$
AND, it's not hard to show that 
$${\rm Hom}_R(M,N) \ \cong \ {\rm Hom}_{{\rm M}_2(R)} \big( \begin{pmatrix}
M & M \\
0 & 0 
\end{pmatrix},\begin{pmatrix}
N & N \\
 0 & 0 
\end{pmatrix} \big).$$
 
 
 \pause
 
 But, e.g.,  $\rmod{\mathbb{R}}$ and $\rmod{\mathbb{C}}$ are not equivalent.  


\end{frame}



\begin{frame}
\frametitle{Some background and history}

{\bf Definition}:   $S$ any ring, $e = e^2 \in S$.  

\medskip

\ \ \ \ \ \ \ \ \ \ \ \   $e$ is {\it full} in $S$ in case $SeS = S$.   

\bigskip

Note:   $SeS$ denotes {\it sums of} elements of the form $ses'$ for $s,s' \in S$.

\bigskip

Example:  $S = {\rm M}_n(R)$,  \ \ $e = e_{1,1}$.  \ \  Then $e$ is full in $S$.   

\medskip

(So is any $e_{i,i}$.)  





\end{frame}


\begin{frame}
\frametitle{Some background and history}
Verbiage:

\medskip

\qquad \qquad ``the rings $R$ and $S$ are {\it Morita equivalent}" \ \ \ \ 

\bigskip

means

\bigskip

\ \ \  the categories $\rmod{R}$ and $\rmod{S}$ are equivalent categories.  

\bigskip
\bigskip



 Notation:  
$R \sim_{ME} S$.




\end{frame}


\begin{frame}
\frametitle{Some background and history}


\noindent {\bf The Original Morita Theorem}: \ \ \ These are equivalent.  

\smallskip


\pause

\begin{itemize}

\item[(M1)]  $R$ and $S$ are Morita equivalent (i.e., the categories $\rmod{R}$ and $\rmod{S}$ are equivalent).  \ \ 
\item[(M2)]  There exist $n\in \N$ and an idempotent $e \in {\rm M}_n(S)$ that is full in ${\rm M}_n(S)$ and for which the rings $R$ and $ e{\rm M}_n(S)e$ are isomorphic.
 
\item[(M3)] There exist an $R$-$S$-bimodule $P$ and an $S$-$R$-bimodule $Q$ and appropriate surjective bimodule homomorphisms  $P \otimes_S Q \to R$ and $Q\otimes_R P \to S$.  

\bigskip

\footnotesize
K. Morita, \emph{Duality for Modules and Its Applications to the Theory of Rings with Minimum Conditions}, Sci. Reports Tokyo Kyoiku Daigaku {\bf 6}A, 1958, 83 -- 142. 
\normalsize
 

\end{itemize}





\end{frame}



\begin{frame}
\frametitle{Some background and history}

\noindent For any ring $T$,   \ \  ${\rm FM}_\infty(T)$ \ \  denotes:  

\medskip

\qquad  countably infinite square matrices over $T$  that contain \\ \qquad  at most finitely many nonzero entries.  
 
 \bigskip
 
 \pause
 
 Note:  ${\rm FM}_\infty(T)$ does not contain a multiplicative identity.  
 
 \medskip
 
  BUT, there are ``local identities" in ${\rm FM}_\infty(T)$.  
 
 \pause
 
 \bigskip
 
   Note:   $e_{1,1}$ is full in ${\rm FM}_\infty(T)$.   
 

\end{frame}



\begin{frame}
\frametitle{Some background and history}
 
 


 
 A fourth  condition  equivalent to those in The Original Morita Theorem:
  %\cite{Stephenson}
 % (see also \cite{Abrams}).
 
% ${\rm M}_\infty(T)$
 

     
 
\begin{itemize}
\item[(M4)]  The rings  ${\rm FM}_\infty(R)$ and ${\rm FM}_\infty(S)$ are isomorphic.        
\end{itemize}
     
     
     \medskip

 \footnotesize
     
 
W. Stephenson, \emph{Characterization of rings and modules by means of lattices}, Ph.D. thesis, Bedford College, University of London, 1965.
\normalsize
 \pause
 
 \bigskip



\smallskip



\noindent  Statements (M1) through (M4): 

\medskip

\qquad \qquad   ``{\bf The Extended Morita 
 Theorem}"


\end{frame}





\begin{frame}
\frametitle{Some background and history}
\footnotesize
\begin{center}
Comments on (M4):
  \ \ \   ${\rm FM}_\infty(R) \cong {\rm FM}_\infty(S)$ as rings.  
  \end{center}
\normalsize

\medskip

1.   Stephenson's proof that (M4) implies $R \sim_{ME} S$  invoked some of his own work on isomorphisms between lattices of submodules of various modules.   

\bigskip
\pause

2.    Using a now-well-understood notion of module categories over (nice) nonunital rings, it's not hard to get $$R \sim_{ME}  {\rm FM}_\infty(R) $$ for any unital $R$.  \ From this, the result that  (M4) implies $R \sim_{ME} S$ is immediate.   


\end{frame}





\begin{frame}
\frametitle{Some background and history}
\footnotesize
\begin{center}
Comments on (M4):
  \ \ \   ${\rm FM}_\infty(R) \cong {\rm FM}_\infty(S)$ as rings.  
  \end{center}
\normalsize

\smallskip

3.   Stephenson's proof that $R \sim_{ME} S$ implies (M4)  consists of two steps.

\medskip

 First, show that $R \sim_{ME} S$ yields an (explicitly constructed)  isomorphism  

 $$ \Phi:  {\rm RFM}_\infty(R) \to {\rm RFM}_\infty(S).$$
 
\pause

Then, show that $\Phi$ restricts to an isomorphism between ${\rm FM}_\infty(R)$ and ${\rm FM}_\infty(S)$.   
 
  \medskip
 
 View the isomorphism in (M4) as ``top down".   

\end{frame}





\begin{frame}
\frametitle{Some background and history}
\footnotesize
\begin{center}
Comments on (M4):
  \ \ \   ${\rm FM}_\infty(R) \cong {\rm FM}_\infty(S)$ as rings.  
  \end{center}
\normalsize

\medskip


4.      A (really beautiful!) result of Camillo gives an additional equivalent condition:

\medskip

\ \ \ \ \ \ \ (M5)   \ \  ${\rm RFM}_\infty(R)$ and ${\rm RFM}_\infty(S)$  are isomorphic as rings.   
%Then one shows this isomorphism takes  ${\rm FM}_\infty(R)$ onto ${\rm FM}_\infty(S)$. 

\bigskip
\medskip
\pause

5.  Many ring theorists were not so impressed by (M4)   ... 

\medskip

\ \ \ \ after all, ${\rm FM}_\infty(R)$ and ${\rm FM}_\infty(S)$ are NON-unital rings.   

\end{frame}





\begin{frame}
\frametitle{Some background and history}

 
$\mathbb{Z}$-graded  rings.

\bigskip

A ring $R$ is $\mathbb{Z}$-graded  in case:

\medskip

\ \ \ 1. \ \ $(R,+)  = \oplus_{t\in \mathbb{Z}} R_t$  \ \  as abelian groups,  and 

\medskip

\ \ \ 2.  \ \ $R_t \cdot R_u \subseteq R_{t+u}$ \ \  for all $t,u \in \mathbb{Z}$.  


\bigskip
\pause

\footnotesize

(All this can be done more generally for any abelian group.)

\smallskip

\pause

(So every element of $R$ is a {\it finite} sum of homogeneous elements.) 

\normalsize

\pause
\bigskip

Familiar examples:   \ \   $k[x]$,   \pause  \ \ $k[x,x^{-1}]$.   

\bigskip
\pause

Silly (but important?) example:   EVERY ring $S$ admits a $\mathbb{Z}$-grading.

\medskip

\ \ \ \ \ \ \ \ \ \ \ \ \ $S_0 := S$; \ \ \  \ $S_t := \{0\}$ for all $t\neq 0$.  


\end{frame}

\begin{frame}
\frametitle{Some background and history}

Note:   If $R$ is $\mathbb{Z}$-graded then $R_0$ is a ring.   




\bigskip

\pause

The gradings might allow for some modifications ...



\bigskip


Example:   Let  $R = k[x,x^{-1}]$.   For  $t\in \mathbb{Z}$,    define  

\medskip

$$R_{2t} = kx^t, \ \mbox{and}   \ R_{2t+1} = \{0\} .$$




\bigskip
\pause

Graded homomorphisms and isomorphisms between $\mathbb{Z}$-graded rings:  defined  as expected.   




\end{frame}

\begin{frame}
\frametitle{Some background and history}

{\bf Lemma}:   $n \in \mathbb{N}$.   If $R$ is $\mathbb{Z}$-graded, then ${\rm M}_n(R)$ is  $\mathbb{Z}$-graded.  

\medskip

 For each $t \in \mathbb{Z}$, 
$$ ({\rm M}_n(R))_t \ := \ {\rm M}_n(R_t).$$


\medskip

In the same way, ${\rm FM}_\infty(R)$ is $\mathbb{Z}$-graded as well.    



\bigskip

The {\it standard}  $\mathbb{Z}$-grading on ${\rm M}_n(R)$ or  ${\rm FM}_\infty(R)$.  


\pause

\medskip

(Note:  ``larger" infinite matrix rings are not necessarily $\mathbb{Z}$-graded.)  

\bigskip
\pause

{\bf Notation for this talk}:   ``graded"  means $\mathbb{Z}$-graded.   



\end{frame}



\begin{frame}
\frametitle{Some background and history}
There are other ``natural"  gradings on ${\rm M}_n(R)$.   For example, take {\it any} $R$ (not necessarily graded).  Then  e.g., on $ {\rm M}_3(R)$, 

\scriptsize
$$ ( {\rm M}_3(R) )_0  :=   \begin{pmatrix}
R & 0 & 0 \\
0  & R & 0 \\
0 &  0 & R
\end{pmatrix},  \ \   ( {\rm M}_3(R) )_1  :=   \begin{pmatrix}
0 & R & 0 \\
0  & 0 & R \\
0 & 0 & 0 
\end{pmatrix}, \ \  ( {\rm M}_3(R) )_{2} \ := \   \begin{pmatrix}
0 & 0 & R \\
0   & 0 & 0 \\
0 & 0 & 0
\end{pmatrix}, $$
$$ ( {\rm M}_3(R) )_{-1}  :=   \begin{pmatrix}
0 & 0 & 0 \\
R  & 0 & 0 \\
0 &  R & 0
\end{pmatrix},  \ \   ( {\rm M}_3(R) )_{-2}  :=   \begin{pmatrix}
0 & 0 & 0 \\
0  & 0 & 0 \\
R & 0 & 0 
\end{pmatrix},$$

\bigskip

and $    \  ( {\rm M}_3(R) )_{i} \ := \   \begin{pmatrix}
0 & 0 & 0 \\
0   & 0 & 0 \\
0 & 0 & 0
\end{pmatrix} \mbox{ for all }  i \neq -2, -1, 0, 1, 2.$




\normalsize

\end{frame}


\begin{frame}
\frametitle{Some background and history}

Suppose $S$ is graded, and $e = e^2 \in S_0$.   Then the {\it corner ring} $eSe$ inherits a grading from $S$:  \ \ 
$ (eSe)_t := eS_te \ \ \forall  t\in \mathbb{Z}.$
\pause  

\medskip


{\bf Key observation}:     Suppose $S$ is graded, and $e = e^2 \in S_0$. 

\medskip

Then,   even if $e$ is full in $S$, $e$ need NOT be full in $S_0$.    

\medskip

Example:   $e_{1,1}$ is full in $S = {\rm M}_3(R)$.   And in the previous example, 
%$e_{1,1} \in S_0$.  
%And $e_{1,1} \in S_0$ in  all previous examples of gradings on $S$.    But, e.g.,  
$$S_0  :=   \begin{pmatrix}
R & 0 & 0 \\
0  & R & 0 \\
0 &  0 & R
\end{pmatrix},  $$
so $e_{1,1} \in S_0$.   But  clearly   $e_{1,1}$ is NOT full in $S_0$.  

\pause

\medskip

But there are  plenty of examples where $e \in S_0$ being full in $S$ does imply that $e$ is full in $S_0$.  



\end{frame}


\begin{frame}
\frametitle{``The Algebraic Stabilization Theorem"}

Recall (M2) and (M4) from the Extended Morita Theorem:
\bigskip 

(M2)     There exist $n\in \N$ and an idempotent $e \in {\rm M}_n(S)$ that is full in ${\rm M}_n(S)$ and for which  $R$ and $ e{\rm M}_n(S)e$ are isomorphic.

\medskip

\ \ \ \ and

\medskip

(M4)    The rings  ${\rm FM}_\infty(R)$ and ${\rm FM}_\infty(S)$ are isomorphic.     

\bigskip
\bigskip


Question:   is there a graded version of this result?   
\end{frame}


\begin{frame}
\frametitle{``The Algebraic Stabilization Theorem"} 

{\bf Algebraic Stabilization Theorem}: \ \  \ \ (A-, Ruiz, Tomforde)

  Let $R$ and $S$ be unital graded rings.   Assume all gradings on matrix rings and corner rings are standard.   Then these two statements are equivalent.

\bigskip

(HG2) \    There exist $n\in \N$ and an idempotent $e \in {\rm M}_n(S)_0$ that is full in ${\rm M}_n(S)_0$ and for which the rings $R$ and $ e{\rm M}_n(S)e$ are graded isomorphic.
  
  \medskip
  
\ \ \ \ \   and

\medskip

(HG4)   \  The rings  ${\rm FM}_\infty(R)$ and ${\rm FM}_\infty(S)$ are graded isomorphic.    


\end{frame}

\begin{frame}
\frametitle{``The Algebraic Stabilization Theorem"}


Strategy  of the proof:   The key situation is where $n=1$ and $S = eRe$ for $e \in R_0$ that is full in $R_0$.   

\medskip


Construct idempotents $\{Q_n, P_n \ | \ n\in \mathbb{N}\}$ in  $\mathrm{RFM}_\infty(R_0)$, 
together with graded homomorphisms 
   $$ \phi_n  \colon  \ Q_n {\rm FM}_\infty(R) Q_n  \ \to \  P_n {\rm FM}_\infty(R) P_n \ \  \ \ \ \ \mbox{and} $$
   $$  \psi_n \colon \ P_n {\rm FM}_\infty(R) P_n \  \to \  Q_{n+1} {\rm FM}_\infty(R) Q_{n+1}$$
   
   

   
%\end{frame}

%\begin{frame}
%\frametitle{``The Algebraic Stabilization Theorem"}


   


\noindent
such that for all $n\in \mathbb{N}$,

\medskip

%\begin{enumerate}
  
  \ \ \ \ \   $Q_{n}Q_{n+1} = Q_n = Q_{n+1} Q_n$, \ \ \ 
  $P_n P_{n+1} = P_n = P_{n+1} P_n$,
   
       \bigskip
    
   $ \bigcup_n P_n {\rm FM}_\infty(R) P_n = {\rm FM}_\infty( eRe ), \  \  \ \bigcup_n Q_n {\rm FM}_\infty(R) Q_n = {\rm FM}_\infty(R)$, 

%\end{enumerate}



\end{frame}


   
%\begin{comment}


\begin{frame}
\frametitle{``The Algebraic Stabilization Theorem"}






and for which the diagram
    
    \[
    \xymatrix{
    Q_n {\rm FM}_\infty(R) Q_n \ar@{^{(}->}[rr]^{i_n} \ar[d]_-{\phi_n} & &  Q_{n+1} {\rm FM}_\infty(R) Q_{n+1} \ar[d]^-{\phi_{n+1}}  \\
     P_{n} {\rm FM}_\infty(R) P_n \ar[urr]^-{\psi_n} \ar@{^{(}->}[rr]^{j_n} & & P_{n+1} {\rm FM}_\infty(R) P_{n+1}
    }
    \]
    commutes.      \ \ \ (``Intertwining" homomorphisms")  
    
    \pause
    \bigskip
    
    
   
    
Note:  fullness of $e$ in $R_0$ is needed to get  that the idempotents   $\{Q_n, P_n \ | \ n\in \mathbb{N}\}$ can be chosen in  $\mathrm{RFM}_\infty(R_0)$ \ (as opposed to in $\mathrm{RFM}_\infty(R)$). 
 
%   $  \bigcup_n Q_n {\rm FM}_\infty(R) Q_n = {\rm FM}_\infty(R)$.)
    
    \end{frame}


\begin{frame}
\frametitle{``The Algebraic Stabilization Theorem"}



The four conditions  imply

$${\rm FM}_\infty(R) \cong \varinjlim ( Q_n {\rm FM}_\infty(R) Q_n , i_n )  $$ 
and
$${\rm FM}_\infty(eRe) \cong \varinjlim ( P_n {\rm FM}_\infty(R) P_n , j_n ),$$

\medskip

and the commutativity of the diagram  implies that these direct limits are not only isomorphic, but in fact graded isomorphic. 
        

   \end{frame}


\begin{frame}
\frametitle{``The Algebraic Stabilization Theorem"}

Remarks:    

\bigskip

1)  The construction is motivated by work done  in the context of C$^*$-algebras.    

\medskip

\footnotesize

L. G. Brown,  \emph{Stable isomorphism of hereditary subalgebras of {$C^*$}-algebras}, Pacific J. Math, \textbf{71}(2), 1977, pp.~335--348.

\normalsize

\bigskip

2)   By imposing the trivial grading on non-graded rings, the Algebraic Stabilization Theorem yields the equivalence of (M2) and (M4) in the Extended Morita Theorem.

\bigskip

3)   This is a ``bottom-up" approach to the isomorphism between ${\rm FM}_\infty(R)$ and ${\rm FM}_\infty(S)$.   

   \end{frame}


\begin{frame}
\frametitle{``The Algebraic Stabilization Theorem"}

4)   So the ``naive" extension of (M4) to graded rings

\medskip

\qquad  (i.e., to use the standard grading on the ${\rm FM}_\infty( - )$ rings) 
 
 \medskip
 
 is NOT equivalent to the ``naive" extension of (M2).
 
 \medskip
 
 
 The additional condition that $e$ be full in ${\rm M}_n(S)_0$ is required.  
 
 \bigskip
 \bigskip
 \pause
 
 {\bf Question}:  What, then, is the appropriate extension of (M1) to the graded setting?  





\end{frame}





\begin{frame}
\frametitle{Graded modules, the category $\grmod{R}$}




\hspace{.5in}Graded modules and graded homomorphisms.

\bigskip


$R$ a graded ring,  $M_R$ a right $R$-module.   

\medskip

$M$ is {\it graded} in case:

\medskip



\ \ \ \ \ $M = \oplus_{t \in \mathbb{Z}} M_t$,  \ \ and \ \  $ M_u R_t \subseteq M_{t+u}$ for all $t, u \in \mathbb{Z}$. 

\bigskip

If $M, N$ are graded right $R$-modules,  an $R$-homomorphism $f: M \to N$ is called {\it graded} in case $f(M_t) \subseteq N_t$ for all $t\in \mathbb{Z}$.  



$$\grmod{R}$$
 denotes the category of graded right  $R$-modules with graded homomorphisms.  


\end{frame}

\begin{frame}
\frametitle{Graded modules, the category $\grmod{R}$}


Here's statement (M1) from the Extended Morita Theorem:

\medskip


(M1)    $R$ and $S$ are Morita equivalent (i.e., the categories $\rmod{R}$ and $\rmod{S}$ are equivalent). 


\bigskip

{\bf Question, recast}:  Is there some appropriate statement analogous to (M1)  about $\grmod{R}$ and $\grmod{S}$ which would be equivalent to (HG2) and (HG4) ?





\end{frame}




\begin{frame}
\frametitle{Graded modules:  some terminology}
 

    For a graded right $A$-module $M$ and $i \in \mathbb{Z}$, the \emph{$i$-suspension of $M$}, denoted $M(i)$, is the graded right $A$-module having $M(i) = M$, with grading given by $M(i)_j = M_{i+j}$.  
    
    \bigskip
    
      For $i \in \mathbb{Z}$,   $\mathcal{T}_i$ denotes the \emph{$i$-suspension functor}  $$\mathcal{T}_i:  \grmod{A} \to \grmod{A}$$  
      given by $M \mapsto M(i)$ on objects, and the identity on morphisms.
      
      \bigskip
      

A functor $\phi: \grmod{A} \to \grmod{B}$ is called {\it graded} when $$\phi \circ \mathcal{T}_\alpha = \mathcal{T}_\alpha \circ \phi $$ for each $\alpha \in \mathbb{Z}$.   


\end{frame}




\begin{frame}
\frametitle{Graded modules:  some terminology}


  A graded functor $\phi : \grmod{A} \to \grmod{B}$ is a {\it graded equivalence} if there is a graded functor $\psi : \grmod{B} \to \grmod{A}$ such that $\phi$ and $\psi$ compose appropriately to the identity functors on each category.   


\bigskip
\bigskip


  If there is a graded equivalence between $\grmod{A}$ and $\grmod{B}$, we say $A$ and $B$ are {\it graded equivalent} or, more formally,  {\it graded Morita equivalent}.  
  
  

\end{frame}




\begin{frame}
\frametitle{Graded modules:  some terminology}
 

  

For any graded ring $A$, we let $U_A$ (or simply by $U$) denote the {\it forgetful functor} 
$$U_A: \grmod{A} \to \rmod{A}.$$

  A functor $\phi^{\prime} : \rmod{A} \to \rmod{B}$ is called a {\it graded functor} if there is a graded functor $\phi: \grmod{A} \to \grmod{B}$ such that  
  $$ U_B \circ \phi   =   \phi^{\prime} \circ U_A$$ as functors from $\grmod{A}$ to $\rmod{B}$.    
 In this situation the functor $\phi$ is called an {\it associated graded functor} of $\phi^{\prime}$.    


\bigskip

A functor $\phi^\prime : \rmod{A} \to \rmod{B}$ is called a {\it graded equivalence} if it is both graded and an equivalence.  




\end{frame}




\begin{frame}
\frametitle{Graded modules:  some terminology}
 
Let $S$ be a graded ring.

\medskip


If $M$ is any right $S_0$-module, then 
$M \otimes_{S_0} S$
is a graded right $S$-module, where
$$(M \otimes_{S_0} S)_i = M \otimes_{S_0} S_i$$
 for each $i \in \mathbb{Z}$.  

\medskip

This gives a functor
$$ -  \otimes_{S_0} S : \rmod{S_0} \to \grmod{S}.$$



\end{frame}




\begin{frame}
\frametitle{Graded modules:  some terminology}
 



{\bf Definition.}      We  call the graded rings $A$ and $B$

\smallskip
\ \ \ \ \ \ \ \ \ \ \ \ \ \  {\it homogeneously graded equivalent}
 \smallskip
 
  in case there exists  a graded equivalence $\psi: \grmod{A} \to \grmod{B}$ 
for which there is an equivalence  of categories
$$\eta: \rmod{A_0} \to \rmod{B_0}$$
such that the diagram
$$  \xymatrix{\rmod{A_0} \ar[d]_{- \otimes_{A_0}A}\ar[r]^\eta&\rmod{B_0} \ar[d]^{- \otimes_{B_0}B}\\
\grmod{A}\ar[r]^\psi &\grmod{B}  }$$
\noindent
commutes  on objects of $\rmod{A_0}$ (up to isomorphism).  




\end{frame}



\begin{frame}
\frametitle{Graded modules:  some terminology}
 


Rephrased:    

\bigskip

 $A$ and $B$ are called {\it homogeneously graded equivalent} in case  there is a category equivalence $$\eta: \rmod{A_0} \to \rmod{B_0}$$  and a graded equivalence  $$\psi: \grmod{A} \to \grmod{B}$$   for which, for each object $M$ of $\rmod{A_0}$, there is an isomorphism  
 $$\psi(M\otimes_{A_0}A) \cong_{gr} (\eta(M))\otimes_{B_0}B$$ as objects of $\grmod{B}$. 
 
 \medskip
 
%  In this situation we  write $A \approx_{\hgr} B$. 

\



\end{frame}


\begin{frame}
\frametitle{The connection between these ideas}

(Recall the Extended Morita Theorem ...)


\pause

\bigskip

{\bf Theorem.}   Let $R$ and $S$ be unital graded rings.   These are equivalent:

\medskip

(HG1)   $R$ is homogeneously graded equivalent to $S$.

\medskip


(HG2) \    There exist $n\in \N$ and an idempotent $e \in {\rm M}_n(S)_0$ that is full in ${\rm M}_n(S)_0$ and for which the rings $R$ and $ e{\rm M}_n(S)e$ are graded isomorphic.

\medskip

(HG4)    ${\rm FM}_\infty(R)$ is graded isomorphic to  ${\rm FM}_\infty(S)$ in the standard grading.  

\end{frame}




\begin{frame}
\frametitle{The Homogeneously Graded Version of the Extended Morita Theorem}

Proof that (HG1) is equivalent to (HG2):   Omitted here.  

\medskip

The proof uses a number of known results about graded rings.    

\bigskip
\bigskip

Here is a great resource:


\ \ \ \ \ R. Hazrat, Graded rings and graded Grothendieck groups. London Mathematical Society Lecture Note Series, \textbf{435}. Cambridge University Press, Cambridge, 2016. vii+235 pp.
% ISBN: 978-1-316-61958-2




\end{frame}




\begin{frame}
\frametitle{The Homogeneously Graded Version of the Extended Morita Theorem}

There is an appropriate ``tensor product of graded bimodules" statement, which is the analog of (M3) in the Extended Morita Theorem, which is equivalent to (HG1), (HG2), (HG4).   Omitted today.   

\bigskip
\bigskip

This completes the picture corresponding to the existence of a graded isomorphism between  ${\rm FM}_\infty(R)$ and  ${\rm FM}_\infty(S)$ (where  the standard grading is used to grade the infinite matrix rings).

\end{frame}




\begin{frame}
\frametitle{More gradings on matrix rings}

Recall this example.   (``Grading $\#$1")  

  (Here $R$ need NOT be graded.)  
%There are other ``natural"  $\mathbb{Z}$-gradings on ${\rm M}_n(R)$.   For example, take {\it any} $R$ (not necessarily graded).  Then  e.g., 
\ \ \ \ \ On $ {\rm M}_3(R)$, 

\scriptsize
$$ ( {\rm M}_3(R) )_0  :=   \begin{pmatrix}
R & 0 & 0 \\
0  & R & 0 \\
0 &  0 & R
\end{pmatrix},  \ \   ( {\rm M}_3(R) )_1  :=   \begin{pmatrix}
0 & R & 0 \\
0  & 0 & R \\
0 & 0 & 0 
\end{pmatrix}, \ \  ( {\rm M}_3(R) )_{2} \ := \   \begin{pmatrix}
0 & 0 & R \\
0   & 0 & 0 \\
0 & 0 & 0
\end{pmatrix}, $$
$$ ( {\rm M}_3(R) )_{-1}  :=   \begin{pmatrix}
0 & 0 & 0 \\
R  & 0 & 0 \\
0 &  R & 0
\end{pmatrix},  \ \   ( {\rm M}_3(R) )_{-2}  :=   \begin{pmatrix}
0 & 0 & 0 \\
0  & 0 & 0 \\
R & 0 & 0 
\end{pmatrix},$$

\bigskip

and $    \  ( {\rm M}_3(R) )_{i} \ := \   \begin{pmatrix}
0 & 0 & 0 \\
0   & 0 & 0 \\
0 & 0 & 0
\end{pmatrix} \mbox{ for all }  i \neq -2, -1, 0, 1, 2.$




\normalsize


\end{frame}



%\begin{comment}
\begin{frame}
\frametitle{More gradings on matrix rings}
Here's another   $\mathbb{Z}$-grading on ${\rm M}_3(R)$.     (``Grading $\#$2")  
  
   (Again, {\it any} $R$.)
\scriptsize
$$ ( {\rm M}_3(R) )_0  :=   \begin{pmatrix}
R & 0 & 0 \\
0  & R & 0 \\
0 &  0 & R
\end{pmatrix},  $$
$$ \ \   ( {\rm M}_3(R) )_5  :=   \begin{pmatrix}
0 & R & 0 \\
0  & 0 & 0 \\
0 & 0 & 0 
\end{pmatrix}, \ \  ( {\rm M}_3(R) )_{3} \ := \   \begin{pmatrix}
0 & 0 & 0 \\
0   & 0 & R \\
0 & 0 & 0
\end{pmatrix}, \ \ 
 ( {\rm M}_3(R) )_{8}  \  :=  \ \   \begin{pmatrix}
0 & 0 & R \\
0  & 0 & 0 \\
0 & 0 & 0
\end{pmatrix}, $$
 $$    ( {\rm M}_3(R) )_{-5}  :=   \begin{pmatrix}
0 & 0 & 0 \\
R  & 0 & 0 \\
0 & 0 & 0 
\end{pmatrix}, \ \   ( {\rm M}_3(R) )_{-3}  :=   \begin{pmatrix}
0 & 0 & 0 \\
0  & 0 & 0 \\
0 & R & 0 
\end{pmatrix},
 \ \   ( {\rm M}_3(R) )_{-8}  :=   \begin{pmatrix}
0 & 0 & 0 \\
0  & 0 & 0 \\
R &0 & 0 
\end{pmatrix}
$$

\bigskip

and $    \  ( {\rm M}_3(R) )_{i} \ := \   \begin{pmatrix}
0 & 0 & 0 \\
0   & 0 & 0 \\
0 & 0 & 0
\end{pmatrix} \mbox{ for all other values of  }  i  .$




\normalsize



\end{frame}
%\end{comment}



\begin{frame}
\frametitle{More gradings on matrix rings}

So if $R$ is graded, we can grade  ${\rm M}_n(R)$ using the standard grading.  

\medskip

And for any $R$, we have gradings  on  ${\rm M}_n(R)$  coming from the matrix structure.  
\medskip

 We  combine these two ways to grade matrix rings over graded rings. 

\bigskip

{\bf Definition.}   If $R$ is graded, we can define a grading on ${\rm M}_n(R)$ as follows.  
 
 \medskip
 
 Pick any sequence $\delta = (z_1, z_2, \dots, z_n) $ in $ \mathbb{Z}^n$.  \ \   For $t\in \mathbb{Z}$, 
 
$$ (( {\rm M}_n(R) )_t )_{i,j}   \ := \  R_{t + z_j - z_i}. $$ 

%$$ (( {\rm M}_n(R) )_t )_{i,j}   \ := \  R_{t + z_j - z_i}. $$ 




% $$ \ \ \mbox{placed in the } (i,j) \mbox{ entry}$$

% placed in any $(i,j)$ entry  for which $j-i = t$.  


\end{frame}

\begin{frame}
\frametitle{More gradings on matrix rings}



{\bf Example}.      $\delta = (12,7,4) = (z_1, z_2,z_3)$.   Let $R$ be any graded ring.  

\medskip
 
\ \ \   We grade ${\rm M}_3(R)$ by setting, for each $t \in \mathbb{Z}$, 




$$( {\rm M}_3(R) )_{t} :=  \begin{pmatrix}
R_{t +  12-12}  & R_{t +  7 - 12}  & R_{t + 4- 12}  \\
R_{t +  12-7}    & R_{t +  7 - 7}  & R_{t + 4 - 7} \\
 R_{t + 12-4} & R_{t + 7-4}  & R_{t + 4-4} 
\end{pmatrix}$$

\medskip

$$ =  \begin{pmatrix}
R_{t }  & R_{t -5}  & R_{t - 8}  \\
R_{t  + 5}    & R_{t }  & R_{t - 3} \\
 R_{t  + 8} & R_{t + 3}  & R_{t } 
\end{pmatrix}$$



\end{frame}


  
  
\begin{frame}
\frametitle{More gradings on matrix rings}

So, if $R$ is not graded, then by trivially grading $R$  

  (i.e., $R_0 = R$, $R_t = 0$ for all $t \neq 0$):

\medskip
\bigskip


 we recover Grading $\#$1 on ${\rm M}_3(R)$ using $\delta = (2,1,0)$,  and
 
 \bigskip
 
  we recover Grading $\#$2 on ${\rm M}_3(R)$ using $\delta = (12,7,4)$.


\end{frame}


\begin{frame}
\frametitle{More gradings on matrix rings}



For $R$ a graded ring, and $\delta = (z_1, z_2, \dots , z_n)$ in $\mathbb{Z}^n$, denote by  
$$ {\rm M}_n(R)[(\delta)]$$

the ring ${\rm M}_n(R)$ with the above grading.   

\pause
\bigskip


It's not hard to see:   if $a \in \mathbb{Z}$, and 
$\delta = (z_1, z_2, \dots , z_n) $ in $ \mathbb{Z}^n$, if we define
$$\delta' := (z_1 - a, z_2 - a, \dots , z_n - a),$$
 then \ \ $ {\rm M}_n(R)[(\delta)] =  {\rm M}_n(R)[(\delta')].$

\pause
\bigskip


Also, if $\kappa = (z, z, \dots , z)$ is constant, then ${\rm M}_n(R)[(\kappa)]$ gives the standard grading on ${\rm M}_n(R)$. 

\end{frame}

\begin{frame}
\frametitle{More gradings on matrix rings}


AND ... all of these ideas work in the same way to give gradings on ${\rm FM}_\infty(R)$: 

\bigskip

Given a graded ring $R$, and  sequence $\delta = (z_1, z_2, z_3, \dots )$ in 
$\mathbb{Z}^{\mathbb{N}}$,  define a grading on ${\rm FM}_\infty(R)$ by setting, for each $t\in \mathbb{Z}$, 

$$ ( ( {\rm FM}_\infty(R) )_t )_{i,j} \ := R_{t + z_j - z_i}. $$

\medskip

Denote this by $ {\rm FM}_\infty(R) [(\delta)]$.  


\end{frame}


\begin{frame}
\frametitle{The Graded Version of The Original Morita Theorem}


\hspace{-.25in} {\bf \  The Graded Version of The Original Morita Theorem.} (Hazrat)     \ 
\bigskip


 
 For graded unital rings $R$ and $S$ these are equivalent.  
 
\begin{itemize} 
\item[(GM1)]  The categories $\rmod{R}$ and $\rmod{S}$  are graded equivalent.
%, via a functor that is compatible with a specific type of equivalence functor between  $\grmod{R}$ and $\grmod{S}$.   
 
 
\item[(GM2)] There exist $n\in \N$ and an  idempotent $e \in {\rm M}_n(S)$ that is full in  ${\rm M}_n(S)$ and a sequence $(\delta)$ in $\mathbb{Z}^n$ for which the rings $R$ and $ e{\rm M}_n(S)[(\delta)]e$ are graded isomorphic.
%\end{itemize}
 
%\noindent Although not explicitly included in \cite{RoozbehBook}, the following is easily seen to be equivalent to (GM1) and (GM2).   (See the comment following Theorem~\ref{GradedVersionExtMoritaThm}.)  

%\begin{itemize} 
\item[(GM3)] There exist a graded $R$-$S$-bimodule $P$ and a graded $S$-$R$-bimodule $Q$ and appropriate surjective graded bimodule homomorphisms  $P \otimes_S Q \to R$ and $Q\otimes_R P \to S$.  
\end{itemize}

\end{frame}





\begin{frame}
\frametitle{The Graded Version of The Extended Morita Theorem}

{\bf Question}.  \ Is there an appropriate (GM4) statement about isomorphisms between infinite matrix rings analogous to (M4) or (HG4) which can be added to the Graded Version of the Original Morita Theorem?

\pause
\medskip

Recall that if $\kappa := (z,z,z, \dots)$ is any constant sequence in $\mathbb{Z}^{\mathbb{N}}$, then ${\rm FM}_\infty(R)[(\kappa)]$ is just the standard grading on ${\rm FM}_\infty(R)$.    

\bigskip

{\bf Theorem.}  (A-, Ruiz, Tomforde)     The equivalent statements (GM1), (GM2), and (GM3) are equivalent to:

\bigskip

(GM4)  There exists a sequence $(\delta)$ in $\mathbb{Z}^{\mathbb{N}}$ such that 

\medskip
 \ \ \ \ \ \ \ \ \  ${\rm FM}_\infty(R)[(\kappa)] \mbox{ is graded isomorphic to }{\rm FM}_\infty(S) [ (\delta) ].$

\end{frame}







\begin{frame}
\frametitle{Graded finitely generated projective modules }


If $R$ is graded then $$ \mathcal{V}^{gr}(R)$$ denotes the graded-isomorphism classes of graded finitely generated projective right $R$-modules.  

\medskip

   $ \mathcal{V}^{gr}(R)$ is an abelian monoid under $\oplus$.   

\bigskip
\pause

  There is a natural ``action" of $\mathbb{Z}[x,x^{-1}]$ on   $ \mathcal{V}^{gr}(R)$, via the suspension functor.    
  
  \medskip
  
    We can then view $ \mathcal{V}^{gr}(R)$ as a $\mathbb{Z}[x,x^{-1}]$-module.  


\bigskip
\bigskip


%Intuitively:  View $\mathcal{V}^{gr}(R)$ as living inside $Gr - R$.   
\end{frame}






\begin{frame}
\frametitle{Leavitt path algebras}


Any Leavitt path algebra is $\mathbb{Z}$-graded, with grading given by setting

$$pq^* \in L_K(E)_n \ \ \mbox{in case} \ \ \ell(p) - \ell(q) = n$$

\smallskip

for paths $p,q$ in $E$, and $n\in \mathbb{Z}$.  

\end{frame}



\begin{frame}
\frametitle{Hazrat's Talented Monoid Conjecture}

One of the two most-discussed currently-open questions in the subject of Leavitt path algebras is 
\begin{center}
{\bf Hazrat's ``Talented Monoid Conjecture"}
\end{center}
Let $E$ and $F$ be finite graphs.  

\medskip

Suppose there is a monoid isomorphism between $\mathcal{V}^{gr}(L_K(E))$ and $\mathcal{V}^{gr}(L_K(F))$ which is compatible with the suspension functors.

\medskip

That is, suppose  there is an isomorphism $\mathcal{V}^{gr}(L_K(E)) \to \mathcal{V}^{gr}(L_K(F))$ as $\mathbb{Z}[x,x^{-1}]$-modules.   
%, and for which $[L_K(E)] \mapsto [L_K(F)]$.  

\bigskip

 Question:   Are $L_K(E)$ and $L_K(F)$ graded Morita equivalent?   
 
 \end{frame}



\begin{frame}
\frametitle{Hazrat's Conjecture}
 
 Hazrat conjectures that the answer is YES.   

\bigskip


(Our current work is therefore at least tangentially related to Hazrat's Conjecture ...)   


\end{frame}



\begin{frame}
\frametitle{Connections}



A graded ring $R$ is {\it strongly} graded in case $R_t R_u = R_{t+u}$ for all $t \in \mathbb{Z}$.   

\bigskip

{\bf Dade's Theorem}:   If $R$ is strongly graded, then $$- \otimes_{R_0} R: \   \rmod{R_0} \to \grmod{R}$$
 is an equivalence of categories.



\bigskip

So for strongly graded rings, graded equivalence and homogeneous graded equivalence reduce to the same idea.  


\end{frame}




\begin{frame}
\frametitle{Connections: \ Leavitt path algebras}

For Leavitt path algebras:  

\bigskip

{\bf Theorem.}   (Hazrat)    \ Suppose $E$ is a finite graph, and $K$ any field.   Then $L_K(E)$ is strongly graded (in the natural $\mathbb{Z}$-grading)   if and only if $E$ has no sinks.    

\bigskip

So for finite graphs with no sinks, $L_K(E)$ and $L_K(F)$ are homogeneously graded equivalent if and only if they are graded equivalent.  



\end{frame}


\begin{frame}
\frametitle{Connections: \ Leavitt path algebras}


For case where the graphs have sinks, the situation is not so clear.

\bigskip

Here's the flavor of one result.  

\bigskip

  

For a finite graph $E$, let \ 
$E^n$ \ denote the paths of length $n$.   Let  
 \ ${\rm Path}(E)$ \  denote the set of all paths in $E$; so  $${\rm Path}(E) = \bigcup_{n\in \mathbb{Z}^{+}} E^n.$$  




%\end{frame}


%\begin{frame}
%\frametitle{Connections: \ Leavitt path algebras}


%For each $n \in \mathbb{Z}^+$
%, we let $G_n$ denote the graph with $n+1$ vertices $z_0, z_1, \ldots, z_n$,  and exactly one edge from $z_i$ to $z_{i-1}$.
%$$G_n :  \qquad  \xymatrix{ \bullet^{z_n} \ar[r] &  \bullet^{z_{n-1}}  \ar[r] & \cdots  \ar[r]& \bullet^{z_1} \ar[r] & \bullet^{z_0}} \ .$$

\end{frame}


\begin{frame}
\frametitle{Connections: \ Leavitt path algebras}


{\bf Proposition.}  
Let $E$ and $F$ be finite acyclic graphs.  Suppose $E$ has exactly one sink $v$ and $F$ has exactly one sink $w$.  Then $L_K(E)$ is homogeneously graded  equivalent to $L_K(F)$ if and only if 
$$\max\{ {\rm length}(\mu) : \mu \in {\rm Path}(E), r(\mu) = v \}$$  $$ \ = \ \max \{ {\rm length} ( \nu ) : \nu \in {\rm Path}(F), r(\nu) =  w \}.$$

\pause

\bigskip

Consequently, for example, the Leavitt path algebras of these graphs are not homogeneously graded equivalent.  

$$E:= \ \ \bullet \ \ \ \ \  \mbox{and} \ \ \ \ \ F:= \ \ \bullet \longrightarrow \bullet$$

%Note:   this is NOT the same as the condition which yields that $L_K(E)$ and $L_K(F)$ are graded equivalent in this setting.    One can show, e.g., that the Leavitt path algebras of the graphs  $$\bullet \ \ \ \ \  \mbox{and} \ \ \ \ \ \bullet \to \bullet$$  are graded equivalent.  But they are not homogeneously graded equivalent.  

\end{frame}


\begin{frame}
\frametitle{Connections: \ Leavitt path algebras}

$$E:= \ \ \bullet \ \ \ \ \  \mbox{and} \ \ \ \ \ F:= \ \ \bullet \longrightarrow \bullet$$

\bigskip

Well known:  $L_K(E)\cong K$ and $L_K(F) \cong {\rm M}_2(K)$.   

\bigskip

So $L_K(E)$ and $L_K(F)$ are Morita equivalent.  


\bigskip

 The natural $\mathbb{Z}$-grading on these Leavitt path algebras: easy to describe.  

\end{frame}


\begin{frame}
\frametitle{Connections: \ Leavitt path algebras}

   

\bigskip

Clearly $${\rm FM}_\infty(K) \ \  \cong \ \  {\rm FM}_\infty({\rm M}_2(K)).$$

   This isomorphism is not a graded isomorphism in standard grading.   (It can't be, by the previous proposition.)   

\bigskip

  But this isomorphism  becomes a graded isomorphism 
  $${\rm FM}_\infty(K)[(0,-1,0,-1,0,-1, ...)]   \ \ \cong_{gr}   \ \   {\rm FM}_\infty({\rm M}_2(K)).$$
  So $L_K(E)$ and $L_K(F)$ are in fact graded Morita equivalent.   

\end{frame}




\begin{frame}
\frametitle{Connections: \ C$^*$-algebras}


\pause

The notion of Morita equivalence is well known to C$^*$-algebraists.  

\medskip



Morita equivalence of the C$^*$-algebras $A$ and $B$ is defined by the existence of an imprimitivity Hilbert bimodule ${}_AX_B$.  

\medskip

Let $\mathcal{K}$ denote the algebra of compact operators on a separable infinite-dimensional Hilbert space. 

\bigskip


{\bf Theorem}:  (Brown-Green-Rieffel) \ \   The $\sigma$-unital $C^*$-algebras $A$ and $B$ are Morita equivalent if and only if $A$ and $B$ are stably isomorphic (i.e., $A \otimes \mathcal{K} \cong B \otimes \mathcal{K}$).    

\end{frame}


\begin{frame}
\frametitle{Connections: \ C$^*$-algebras}


 Since $\mathcal{K} = \overline{\mathrm{FM}_\infty(\mathbb{C})}$ and $A \otimes \mathcal{K} \cong \overline{\mathrm{FM}_\infty(A)}$, we see that $A \otimes \mathcal{K}$ is the analytic analogue of $\mathrm{FM}_\infty(A)$.
 
 \bigskip   
 
 So  having $A$ stably isomorphic to $B$ (i.e., $A \otimes \mathcal{K} \cong B \otimes \mathcal{K}$) is the analytic analogue of having $\mathrm{FM}_\infty (A) \cong \mathrm{FM}_\infty (B)$.  
 
 
 \bigskip
 
 
 So for  operator algebraists inquiring about corresponding ring-theoretic results, (M4) is a quite natural condition.
 
 
 \end{frame}
 

\begin{frame}
\frametitle{More C$^*$-algebra Connections: graph C$^*$-algebras}

Let $E$ be a graph and let $C^*(E)$ be the graph $C^*$-algebra.  

\medskip

Then  there is an action $\gamma^E$ of the circle $\mathbb{T}$ on $C^*(E)$.  
%comes equipped with an action $\gamma^E$ of the circle $\mathbb{T}$, the \emph{gauge action}.  
Specifically, on the generators of $C^*(E)$, $\gamma^E$  is given by  
$$
\gamma_z^E(p_v)=p_v \quad \text{and} \quad \gamma_z^E(s_e)=z s_e.
$$
for $z\in \mathbb{T}$.  
This `` gauge action" induces a $\mathbb{Z}$-grading on $C^*(E)$  via
$$
C^*(E)_n = \{ a \in C^*(E) \ | \  \gamma_z^E(a)=z^n a \}.
$$
and then taking the closure.  

 \end{frame}
 

\begin{frame}
\frametitle{More C$^*$-algebra Connections: graph C$^*$-algebras}

{\bf Theorem}.  
Let $E$ and $F$ be finite graphs.  Then 

\medskip

there exists a $*$-isomorphism $$\varphi \colon C^*(E) \to C^*(F) \ \ \mbox{having} \ \  \gamma_z^F \circ \varphi = \varphi \circ \gamma_z^E$$

\medskip

 if and only if 
 
 \medskip
 
 there exists a graded $*$-isomorphism from $C^*(E)$ to $C^*(F)$.


 \end{frame}
 

\begin{frame}
\frametitle{More C$^*$-algebra Connections: graph C$^*$-algebras}

We define 
 $$\gamma_z^{E, s} := \gamma_z^{E} \otimes \iota : C^*(E) \otimes \mathcal{K} \to C^*(E) \otimes \mathcal{K}.$$   
 Call $\gamma_z^{E, s}$ the \emph{stabilized action}.

\bigskip


  Then $\gamma_z^{E,s}$ is an action of $\mathbb{T}$ on $C^*(E) \otimes \mathcal{K}$ which induces a $\mathbb{Z}$-grading on  $C^*(E) \otimes \mathcal{K}$ (after closing) via
$$
(C^*(E)\otimes\mathcal{K})_n = \{ x \in C^*(E)\otimes \mathcal{K}  \ | \  \gamma_z^{E,s}(x)=z^n x \}.
$$
This grading is the ``standard" grading of $C^*(E)\otimes \mathcal{K}$.  In fact,
$$
(C^*(E) \otimes \mathcal{K})_n = \overline{ \bigcup_{k=1}^\infty {\rm M}_k( C^*(E)_n)  }
$$



 \end{frame}
 

\begin{frame}
\frametitle{More C$^*$-algebra Connections: graph C$^*$-algebras}



{\bf Theorem}. 
Let $E$ and $F$ be graphs.  Then there exists a $*$-isomorphism 
$$\varphi \colon C^*(E) \otimes \mathcal{K} \to C^*(F) \otimes \mathcal{K} \ \ \mbox{such that} \ \  \gamma_z^{F,s} \circ \varphi = \varphi \circ \gamma_z^{E,s}$$
 if and only if 
 
 \medskip
 
 there exists a graded $*$-isomorphism
 
 $$\psi:  C^*(E) \otimes \mathcal{K} \to C^*(F) \otimes \mathcal{K},$$ 
 
 \medskip
 
 (where the stabilizations are given the standard grading).



\bigskip
\bigskip

This is the C$^*$-analog to  condition (HG4), for graph C$^*$-algebras.   
 \end{frame}


\begin{frame}
\frametitle{More C$^*$-algebra Connections}



There is a C$^*$-algebra  analog to the (HG1) condition, in situations more general than the one described above for graph C$^*$-algebras.  


\bigskip
\bigskip

(It has been worked out by Efren Ruiz; still work in progress.)  



 \end{frame}
 




\begin{frame}
\frametitle{More C$^*$-algebra Connections}

{\bf Theorem.}  (Ruiz)  
Let $G$ be a locally compact group.  Let $A$ and $B$ be unital $C^*$-algebras and let $\alpha$ and $\beta$ be actions of $G$ on the $C^*$-algebras $A$ and $B$ respectively.  TFAE:
\bigskip

1.  There exists a $*$-isomorphism $$\varphi \colon A \otimes \mathcal{K} \to B \otimes  \mathcal{K} $$  such that 
$
\beta_g^s \circ \varphi(x) = \varphi \circ \alpha_g^s(x)
$
for all $x \in A \otimes \mathcal{K}$ and for all $g \in G$, where $\alpha_g^s$ and $\beta_g^s$ are the stabilized actions.

\bigskip 
2.  The systems $(A, \alpha)$ and $(B, \beta)$ are Morita equivalent via an imprimitivity $A-B$-bimodule $(M, \gamma)$ such that
$$
M^\gamma = \{ x \in M \ | \  \gamma_g(x)=x \text{ for all } g \in G \}
$$
is an imprimitivity $A^\alpha-B^\beta$-bimodule.





 \end{frame}













\begin{frame}
\frametitle{Morita equivalence for graded rings}



\bigskip
\bigskip

\huge
\begin{center}
{\bf Thank you \\ for your time.}
\end{center}
\normalsize

\end{frame}


\end{document}



%%%%%%%%%%%%%%%%%%%%%%%%%%%%%%%%%%%%%
%
%
%   END DOCUMENT
%
%
%%%%%%%%%%%%%%%%%%%%%%%%%%%%%%%%%%%%%




\begin{frame}
\frametitle{More C$^*$-algebra Connections}

There is an analog to the (XXX) condition;  it has been worked out by Efren Ruiz (in preparation). 

\bigskip


{\bf Theorem.}  (Ruiz)  
Let $A$ and $B$ be unital $C^*$-algebras and let $\alpha$ and $\beta$ be actions of $G$ on the $C^*$-algebras $A$ and $B$ respectively.  Then the following are equivalent.



1.  There exists a $*$-isomorphism $$\varphi \colon A \otimes \mathcal{K} \to B \otimes  \mathcal{K} $$  such that 
$
\beta_g^s \circ \varphi(a) = \varphi \circ \alpha_g^s(a)
$
for all $a \in A$ and for all $g \in G$, where $\alpha_g^s$ and $\beta_g^s$ are the stabilized actions.

\bigskip 
2.  The systems $(A, \alpha)$ and $(B, \beta)$ are Morita equivalent via an imprimitivity $A-B$-bimodule $(M, \gamma)$ such that
$$
M^\gamma = \{ x \in M \ | \  \gamma_g(x)=x \text{ for all } g \in G \}
$$
is an imprimitivity $A^\alpha-B^\beta$-bimodule.





 \end{frame}
















\begin{frame}
\frametitle{Some background and history}
Here's another   $\mathbb{Z}$-grading on ${\rm M}_3(R)$.   
\tiny
$$ ( {\rm M}_3(R) )_0  :=   \begin{pmatrix}
R & 0 & 0 \\
0  & R & 0 \\
0 &  0 & R
\end{pmatrix},  $$
$$ \ \   ( {\rm M}_3(R) )_3  :=   \begin{pmatrix}
0 & R & 0 \\
0  & 0 & 0 \\
0 & 0 & 0 
\end{pmatrix}, \ \  ( {\rm M}_3(R) )_{5} \ := \   \begin{pmatrix}
0 & 0 & 0 \\
0   & 0 & R \\
0 & 0 & 0
\end{pmatrix}, \ \ 
 ( {\rm M}_3(R) )_{8}  \  :=  \ \   \begin{pmatrix}
0 & 0 & R \\
0  & 0 & 0 \\
0 & 0 & 0
\end{pmatrix}, $$
 $$    ( {\rm M}_3(R) )_{-3}  :=   \begin{pmatrix}
0 & 0 & 0 \\
R  & 0 & 0 \\
0 & 0 & 0 
\end{pmatrix}, \ \   ( {\rm M}_3(R) )_{-5}  :=   \begin{pmatrix}
0 & 0 & 0 \\
0  & 0 & 0 \\
0 & R & 0 
\end{pmatrix},
 \ \   ( {\rm M}_3(R) )_{-8}  :=   \begin{pmatrix}
0 & 0 & 0 \\
0  & 0 & 0 \\
R &0 & 0 
\end{pmatrix}
$$

\bigskip

and $    \  ( {\rm M}_3(R) )_{i} \ := \   \begin{pmatrix}
0 & 0 & 0 \\
0   & 0 & 0 \\
0 & 0 & 0
\end{pmatrix} \mbox{ for all other values of  }  i  .$




\normalsize



\end{frame}




\begin{frame}
\frametitle{Some background and history}




\hspace{.5in}Graded modules and graded homomorphisms.

\bigskip


$R$ a graded ring,  ${}_RM$ a right $R$-module.   

\medskip

$M$ is {\it graded} in case:

\medskip



\ \ \ \ \ $M = \oplus_{t \in \mathbb{Z}} M_t$,  \ \ and \ \  $ M_u R_t \subseteq M_{t+u}$ for all $t, u \in \mathbb{Z}$. 

\bigskip

If $M, N$ are graded right $R$-modules,  an $R$-homomorphism $f: M \to N$ is called {\it graded} in case $f(M_t) \subseteq N_t$ for all $t\in \mathbb{Z}$.  



$$\grmod{R}$$
 denotes the category of graded right $R$-modules with graded homomorphisms.  


\end{frame}

\begin{frame}
\frametitle{Some background and history}






\end{frame}



\begin{frame}
	\frametitle{ General overview  }


%Now that our main results have been established, we present a brief overview of a monoid-theoretic aspect of this article.  

%A ``distinguished element"  in a monoid $M$ is an element $I\in M$ for which for each $a\in M$ there exists $b\in M$ and $n\in \mathbb{N}^+$ with $a+b = nI$.  If $R$ is a unital ring then   $[R]$ is a distinguished element in $\mathcal{V}(R)$. For a sandpile monoid $\SP(G)$,  both  $I_1 = \sum_{v\in G^0 \setminus\{s\}}v$ and $I_2 =    \sum_{v\in G^0 \setminus\{s\}} (|s^{-1}(v)-1|)v$ are easily seen to be distinguished elements.  (The element $I_2$ is sometimes denoted  MAX in the literature.)  


{\bf Assume that ``monoid" means ``finitely generated commutative monoid written additively."}   

\begin{center}
Let $\mathcal{U}$  denote the class of monoids that are isomorphic to \\ 
the $ \mathcal{V}$-monoid of some associative unital ring.  

\medskip

Let $\mathcal{S}_{M_{(E,\omega)}}$ denote the class of monoids that are isomorphic to  \\ the vertex-weighted monoid $M_{(E,\omega)}$  of a vertex-weighted finite directed graph $(E,\omega)$.   
%the $\mathcal{V}$-monoid of a weighted Leavitt path algebra of a finite graph.
 
\medskip



 Let $\mathcal{S}_{M_E}$ denote the class of monoids that are isomorphic to  \\
 the    monoid $M_{E}$  of a finite directed graph $E$.   %the $\mathcal{V}$-monoid  of a standard (i.e., unweighted) Leavitt path algebra of a finite graph.

\medskip


Let $\mathcal{S}_{\rm con-sand}$ denote the class of monoids that are isomorphic to \\
the  sandpile monoid of a conical sandpile graph. 

\end{center}

\end{frame}


%------slide-----------------------------------------------------------------%


\begin{frame}
	\frametitle{ General overview  }
Then:  

 
 
 
\begin{enumerate}
\item  By Bergman's Realization Theorem,  $\mathcal{U}$ is precisely the class of monoids that are conical and have a distinguished element.      

\pause

\item    In our main result  we established that $\mathcal{S}_{\rm con-sand} \subseteq \mathcal{S}_{M_{(E,\omega)}}$.  There are plenty of examples to show that this inclusion is proper.  (E.g. $\Z^{\geq 0}$.)  

\pause

\item  Clearly $\mathcal{S}_{M_E} \subseteq \mathcal{S}_{M_{(E,\omega)}}$, and this inclusion is proper (e.g. $C_{m,n}$ for any $m \geq 2$).  

\pause

\item   {\bf Proposition}:  The monoids in the intersection $\mathcal{S}_{\rm con-sand} \cap \mathcal{S}_{M_E}$  are precisely direct sums of copies of $C_{n_j}$ over any finite sequence of integers $n_j$, $1 \leq j \leq t$.   \ \ 

\ \ \ \small (Proof:  use the refinement property of the elements of $\mathcal{S}_{M_E}.$)  \normalsize 

\end{enumerate}

   
%> We also know obviously that S2 \subsetneq S1.    
%> We then identify the monoids in S3 \cap S2  (finite direct sum of C_n monoids).  

\medskip


\end{frame}


%------slide-----------------------------------------------------------------%


\begin{frame}
	\frametitle{ General overview  }







\tikzstyle{decision} = [diamond, draw, fill=blue!50]
\tikzstyle{line} = [draw,  -stealth, thick]
\tikzstyle{lined} = [draw, bend right=45, thick, dashed]
\tikzstyle{elli}=[draw, ellipse,minimum height=2cm, minimum width=2cm,  text width=5em]
\tikzstyle{elli2}=[draw, ellipse,minimum height=4cm, minimum width=7cm,  text width=5em, text centered]

\tikzstyle{block} = [draw, rounded corners, rectangle, top color=white, bottom color=blue!50, text width=8em, text centered, minimum height=15mm, node distance=7em]
\tikzstyle{largeblock} = [draw, rounded corners, rectangle,   minimum height=5cm, minimum width=10cm,  node distance=1em]

\tikzstyle{block2} = [draw, rounded corners, rectangle, color=blue!50, text width=8em, text centered,  minimum height=15mm, node distance=7em]
\begin{wrapfigure}{r}{9cm}






\begin{tikzpicture}
\node [largeblock]  (leavitt) {};
\node at (-4cm,2cm) {$\mathcal U$};
\node at (0cm,1.3cm) {$S_{M_{(E,w)}}$};
\node at (-2cm,0cm) {$S_{\operatorname{con-sand}}$};
\node at (2cm,0cm) {$S_{M_E}$};

\node[elli, in=leavitt, xshift=-1cm, yshift=0 cm] (symbol) {};
\node[elli, in=leavitt, xshift=1cm, yshift=0 cm] (symbol) {};
\node[elli2, in=leavit] (symbol)   {$\scriptstyle{\oplus C_{n_j}}$}; 

%\bigoplus_{j=1}^t C_{n_j}



\end{tikzpicture}

\end{wrapfigure}



\end{frame}

%------slide-----------------------------------------------------------------%




















\begin{frame}
	\frametitle{ Sandpile models }

\tiny


$$    \xymatrix{
\!\!\!      &    \bullet^0   & \\
\bullet^0 & \bullet^4 \ar@{-}[l]  \ar@{-}[r] \ar@{-}[u] \ar@{-}[d]  & \bullet^0      \ \rightsquigarrow  \\
 & \bullet^0  &  \\
}       \pause      \xymatrix{
\!\!\!      &    \bullet^1   & \\
  \ \ \bullet^1 & \bullet^0 \ar@{-}[l]  \ar@{-}[r] \ar@{-}[u] \ar@{-}[d]  & \bullet^1    \   \rightsquigarrow  \\
& \bullet^1  & \\
}        \pause 
  \xymatrix{
\!\!\!      &    \bullet^0   & \\
  \ \ \bullet^1 & \bullet^1 \ar@{-}[l]  \ar@{-}[r] \ar@{-}[u] \ar@{-}[d]  & \bullet^1    \  \rightsquigarrow   \\
& \bullet^1  & \\
}          $$
$$   \pause  \xymatrix{
\!\!\!      &    \bullet^0   & \\
\rightsquigarrow \ \ \  \bullet^0 & \bullet^2 \ar@{-}[l]  \ar@{-}[r] \ar@{-}[u] \ar@{-}[d]  & \bullet^1     \ \rightsquigarrow   \cdots  \rightsquigarrow  \\
 & \bullet^1  &  \\
}      \pause       \xymatrix{
\!\!\!      &    \bullet^0   & \\
  \ \ \bullet^0 & \bullet^4 \ar@{-}[l]  \ar@{-}[r] \ar@{-}[u] \ar@{-}[d]  & \bullet^0    \   \rightsquigarrow \\
& \bullet^0  & \\
}          $$






\normalsize



%$$\xymatrix{
%\!\!\!    \bullet^s   &   \bullet^x \ar@{.}[l]  \ar@{.}@/_{5pt}/ [l]  \ar@/_{10pt}/ [l]_{f_1}  \ar@/^{10pt}/ [l]^{f_k}     \ar@{.}@(l,d) \ar@(ur,dr)^{e_{1}} \ar@(r,d)^{e_{2}} \ar@(dr,dl)^{e_{3}} 
%\ar@{.}@(l,u) \ar@(u,r)^{e_n}  \ar@{.}@(ul,ur)& 
%}$$

\end{frame}





