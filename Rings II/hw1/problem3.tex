Let $K$ be any field. Let $F$ be the graph \ \ \  \ 
$\xymatrix{ \bullet \ar@(ul,dl)  & \bullet \ar[l]}$. Explicitly write down a K-algebra isomorphism 
$M_2(K[x,x^{-1}])\cong L_K(F)$.\\

\begin{solution}\renewcommand{\qedsymbol}{}\ \\
    Let $K$ and $F$ be as above, and $L_K(F)$ be the Leavitt algebra of $F$. Let $G=M_2(K[x,x^{-1}])$.
    Denote the edges from left to right as $f$ and $e$ and the vertices from left to right as $v$ and
    $u$. Consider the mapping $\phi:L_K(F)\rightarrow G$ given by

    $$\phi(u)=\left(\begin{array}{cc} 1 & 0 \\ 0 & 0 \end{array}\right)$$

    $$\phi(e)=\left(\begin{array}{cc} 0 & 1 \\ 0 & 0 \end{array}\right)$$

    $$\phi(e^*)=\left(\begin{array}{cc} 0 & 0 \\ 1 & 0 \end{array}\right)$$

    $$\phi(v)=\left(\begin{array}{cc} 0 & 0 \\ 0 & 1 \end{array}\right)$$

    $$\phi(f)=\left(\begin{array}{cc} 0 & 0 \\ 0 & x \end{array}\right)$$

    $$\phi(f^*)=\left(\begin{array}{cc} 0 & 0 \\ 0 & x^{-1} \end{array}\right)$$

    Now, we will check that the relations are satisfied. Notice that:

    $$\phi(u)\phi(u)-\phi(u)=\left(\begin{array}{cc} 1 & 0 \\ 0 & 0 \end{array}\right)^2-
    \left(\begin{array}{cc} 1 & 0 \\ 0 & 0 \end{array}\right)$$
    $$=\left(\begin{array}{cc} 1 & 0 \\ 0 & 0 \end{array}\right)-
    \left(\begin{array}{cc} 1 & 0 \\ 0 & 0 \end{array}\right)=0$$

    $$\phi(v)\phi(v)-\phi(v)=\left(\begin{array}{cc} 0 & 0 \\ 0 & 1 \end{array}\right)^2-
    \left(\begin{array}{cc} 0 & 0 \\ 0 & 1 \end{array}\right)$$
    $$=\left(\begin{array}{cc} 0 & 0 \\ 0 & 1 \end{array}\right)-
    \left(\begin{array}{cc} 0 & 0 \\ 0 & 1 \end{array}\right)=0$$

    $$\phi(u)\phi(v)=\left(\begin{array}{cc} 1 & 0 \\ 0 & 0 \end{array}\right)
    \left(\begin{array}{cc} 0 & 0 \\ 0 & 1 \end{array}\right)=0$$
    $$=\left(\begin{array}{cc} 0 & 0 \\ 0 & 1 \end{array}\right)
    \left(\begin{array}{cc} 1 & 0 \\ 0 & 0 \end{array}\right)=\phi(v)\phi(u)$$

    and

    $$\phi(u)+\phi(v)=\left(\begin{array}{cc} 1 & 0 \\ 0 & 0 \end{array}\right)+
    \left(\begin{array}{cc} 0 & 0 \\ 0 & 1 \end{array}\right)$$
    $$=\left(\begin{array}{cc} 1 & 0 \\ 0 & 1 \end{array}\right)=I_2$$

    So, we have that the mappings of the vertices are also idempotent, orthogonal, and sum to the
    identity. Now observe that:

    $$\phi(v)\phi(f)=\left(\begin{array}{cc} 0 & 0 \\ 0 & 1 \end{array}\right)
    \left(\begin{array}{cc} 0 & 0 \\ 0 & x \end{array}\right)$$
    $$=\left(\begin{array}{cc} 0 & 0 \\ 0 & x \end{array}\right)$$
    $$=\left(\begin{array}{cc} 0 & 0 \\ 0 & x \end{array}\right)
    \left(\begin{array}{cc} 0 & 0 \\ 0 & 1 \end{array}\right)=\phi(f)\phi(v)$$

    $$\phi(v)\phi(f^*)=\left(\begin{array}{cc} 0 & 0 \\ 0 & 1 \end{array}\right)
    \left(\begin{array}{cc} 0 & 0 \\ 0 & x^{-1} \end{array}\right)$$
    $$=\left(\begin{array}{cc} 0 & 0 \\ 0 & x^{-1} \end{array}\right)$$
    $$=\left(\begin{array}{cc} 0 & 0 \\ 0 & x^{-1} \end{array}\right)
    \left(\begin{array}{cc} 0 & 0 \\ 0 & 1 \end{array}\right)=\phi(f^*)\phi(v)$$

    $$\phi(f)\phi(f^*)=\left(\begin{array}{cc} 0 & 0 \\ 0 & x \end{array}\right)
    \left(\begin{array}{cc} 0 & 0 \\ 0 & x^{-1} \end{array}\right)$$
    $$=\left(\begin{array}{cc} 0 & 0 \\ 0 & 1 \end{array}\right)=\phi(v)$$

    $$\phi(f^*)\phi(f)=\left(\begin{array}{cc} 0 & 0 \\ 0 & x^{-1} \end{array}\right)
    \left(\begin{array}{cc} 0 & 0 \\ 0 & x \end{array}\right)$$
    $$=\left(\begin{array}{cc} 0 & 0 \\ 0 & 1 \end{array}\right)=\phi(v)$$

    $$\phi(u)\phi(e)=\left(\begin{array}{cc} 1 & 0 \\ 0 & 0 \end{array}\right)
    \left(\begin{array}{cc} 0 & 1 \\ 0 & 0 \end{array}\right)$$
    $$=\left(\begin{array}{cc} 0 & 1 \\ 0 & 0 \end{array}\right)=\phi(e)$$

    $$\phi(v)\phi(e^*)=\left(\begin{array}{cc} 0 & 0 \\ 0 & 1 \end{array}\right)
    \left(\begin{array}{cc} 0 & 0 \\ 1 & 0 \end{array}\right)$$
    $$=\left(\begin{array}{cc} 0 & 0 \\ 1 & 0 \end{array}\right)=\phi(e^*)$$

    $$\phi(e)\phi(u)=\left(\begin{array}{cc} 0 & 1 \\ 0 & 0 \end{array}\right)
    \left(\begin{array}{cc} 1 & 0 \\ 0 & 0 \end{array}\right)=0$$

    $$\phi(e^*)\phi(v)=\left(\begin{array}{cc} 0 & 0 \\ 1 & 0 \end{array}\right)
    \left(\begin{array}{cc} 0 & 0 \\ 0 & 1 \end{array}\right)=0$$

    $$\phi(e)\phi(e^*)=\left(\begin{array}{cc} 0 & 1 \\ 0 & 0 \end{array}\right)
    \left(\begin{array}{cc} 0 & 0 \\ 1 & 0 \end{array}\right)$$
    $$=\left(\begin{array}{cc} 1 & 0 \\ 0 & 0 \end{array}\right)=\phi(u)$$

    and

    $$\phi(e^*)\phi(e)=\left(\begin{array}{cc} 0 & 0 \\ 1 & 0 \end{array}\right)
    \left(\begin{array}{cc} 0 & 1 \\ 0 & 0 \end{array}\right)$$
    $$=\left(\begin{array}{cc} 0 & 0 \\ 0 & 1 \end{array}\right)=\phi(v)$$

    So, all of the relations described in the Leavitt algebra are satisfied by the mapping and thus,
    this map over the free algebra mod the relations is the unique desired homomorphism.

    Now, consider the mapping $\psi:G\rightarrow L_K(F)$ given by:

    $$\psi(\left(\begin{array}{cc} 1 & 0 \\ 0 & 0 \end{array}\right))=u$$

    $$\psi(\left(\begin{array}{cc} 0 & 1 \\ 0 & 0 \end{array}\right))=e$$

    $$\psi(\left(\begin{array}{cc} 0 & 0 \\ 1 & 0 \end{array}\right))=e^*$$

    $$\psi(\left(\begin{array}{cc} 0 & 0 \\ 0 & 1 \end{array}\right))=v$$

    $$\psi(\left(\begin{array}{cc} 0 & 0 \\ 0 & x \end{array}\right))=f$$

    $$\psi(\left(\begin{array}{cc} 0 & 0 \\ 0 & x^{-1} \end{array}\right))=f^*$$

    Notice that, with almost identical but reversed logic, $\psi$ is also a homomorphism since it
    satisfies the relations over the free algebra mod the aforementioned relations.

    Finally, note that for all $x\in G$, 
    $$\phi(\psi(x))=x$$

    and for all $x\in L_K(F)$

    $$\psi(\phi(x))=x$$

    Hence, each of the derived homomorphisms has an inverse, namely the other. That is, they are
    both bijective. Therefore, they are isomorphisms yielding $M_2(K[x,x^{-1}])\cong L_K(F)$ as desired.
\end{solution}