\documentclass{beamer}
\usetheme{Warsaw}

\usepackage{amsfonts}
\usepackage[mathcal]{eucal}
\usepackage{eufrak}
\usepackage{amssymb}
\usepackage{amsmath}
\usepackage{mathrsfs}

\title{Equivalent Codes From Finite Fields}

\author{Stephanie Amber Klumpe}

\institute{University of Colorado at Colorado Springs}

\date{May 1, 2023}

\begin{document}

\begin{frame}
    \titlepage
\end{frame}

\begin{frame}{Encoding}
    
    Imagine you wanted to send the message "Hi" to someone, but you were concerned about the quality
    or security of the data line you were using.

    \medskip

    \pause

    You would likely want to encrypt it, but how would you do that if you wanted to make sure that the
    receiver got the whole message even if errors were present?

    \bigskip

    \pause

    We can use redundancy!

\end{frame}

\begin{frame}{Redundancy}
    
    This is the process of adding extra information to codewords that the recipient knows will be there.

    \bigskip

    These are called check bits and allow for the recipient to find errors after decrypting because they
    know what should be at the end.

    \bigskip

    \pause

    Redundancy turns the binary version of "Hi":
    
    $$01001000\;\;01101001$$
    
    into
    
    $$0100101\;\;1000110\;\;0110110\;\;1001001$$

\end{frame}

\begin{frame}{Richard Hamming}
    
    This is the brain child of Richard Hamming. He was a mathematician in the mid 1900s and did
    extensive research in the field of mathematics at Bell Labs.

    \bigskip

    Many things are attributed to his name such as Hamming code, Hamming numbers, and Hamming distance.
    He also spearheaded the area in coding theory called error correcting code(\textbf{ECC}).

    \bigskip

    \pause

    \textbf{ECC} is used in much of telecom today including computers, cell towers, and satellite system.

\end{frame}

\begin{frame}{Hamming Code}
    
    With his work on \textbf{ECC}, he devised a particular version known as the $[7,4]$ code. This uses
    the block matrix

    $$G=\left[\begin{array}{ccccccc}
        1 & 0 & 0 & 0 & 1 & 1 & 0 \\
        0 & 1 & 0 & 0 & 1 & 0 & 1 \\
        0 & 0 & 1 & 0 & 0 & 1 & 1 \\
        0 & 0 & 0 & 1 & 1 & 1 & 1
    \end{array}\right]$$

    to take packets of four and encode them into packets of seven. The $[7,4]$ code allows for detection
    and correction of single bit errors or just detection of double bit errors. This is the code that we
    used on the word "Hi" above.

\end{frame}

\begin{frame}{Hamming Code}
    
    \textbf{Example:} Let's look at the letter "H". In binary, it is given by
    "01001000". Notice that this binary codeword cannot be put through matrix multiplication in the
    Hamming code. However, each packet of four can be. So,

    $$(0 1 0 0)G=0100101\;\;\text{and}\;\; (1 0 0 0)G=1000110$$

    \bigskip

    \pause

    The process for checking for errors comes down to checking if bits in certain positions line up with
    the intended data.

\end{frame}

\begin{frame}{Linear Codes}
    
    What we saw above is an example of a linear code. These are \textbf{ECC} for which any linear
    combination of codewords is also a codeword.

    \bigskip

    That is to say, linear codes are finite abelian groups.

\end{frame}

\begin{frame}{Alphabets}

    We are used to using the english alphabet or the binary system to make codes.
    
    \medskip

    \textbf{Example:} "Hello world" and "Rovvy gybvn" are the same in the english alphabet as one is
    just a Ceaser cipher of the other.

    \bigskip

    \pause

    Thanks to work done by ring theorists and coding theorists, we can also use rings and modules.

    \bigskip

    \pause

    This then leads to the natural mathematical question of when codes in these new alphabets can be
    considered equivalent.
    
\end{frame}

\begin{frame}{Maps and Equivalencies}
    
    In many areas of mathematics, we are interested in maps that preserve certain properties.

    \medskip

    In algebra we look at homomorphisms, in differential geometry we look at diffeomorphisms, in
    topology we look at homeomorphisms, etc.

    \bigskip

    \pause

    For us to even look at our desired morphsims, we first need what is known as the Hamming weight.

   
    % I make note that we will not be using the distance part, but it does come up with respect to these
    % definitions since isometries preserve distance.

\end{frame}

\begin{frame}{Hamming Weight}
    
    The Hamming weight of a string of letters is the number of nonzero elements from the alphabet and is
    denoted $wt(x)$ for each codeword $x\in C$.

    \medskip

    \pause

    \textbf{Example:} Consider the letter 'i' from above. Then, $x=01101001$ and so

    $$wt(01101001)=4$$

\end{frame}

\begin{frame}{Maps and Equivalencies}
    
    In coding and information theory, we look at morphisms that preserve the norms of vector spaces.
    These are called linear isometries.

    \bigskip

    In our particular case, the linear isometries $f:C\to C'$ will preserve the Hamming weight
    and the Hamming distance.

    \bigskip

    \pause

    \textbf{Example:} If $C=01011001$ and $C'=10100110$. Then the map $f:C\to C'$ defined by

    $$f(0)=1\;\;\;f(1)=0$$
    
    is a linear isometry between the two codes.

\end{frame}

\begin{frame}{Maps and Equivalencies}

    Two codes $C,C'$ are equivalent when there exists a linear isometry $f$ between them.

    \bigskip

    \pause
    
    Another kind of morphism of interest in coding theory is the monomial transformation.\\
    These are module epimorphisms $T:C\to C'$ between codes such that

    $$T(x_1,\ldots,x_n)=(x_{\sigma(1)}u_1,\ldots,x_{\sigma(n)}u_n)$$
    
    where $\sigma$ is a permutation of the set $\{1,\ldots,n\}$ and $u_i\in R$ are units for
    $1\leq i\leq n$.

    \bigskip

    \pause
    
    We also say that two codes $C,C'$ are equivalent if there exists a monomial transformation $T$
    between them.

\end{frame}

\begin{frame}{The Big Question}
    
    Can we extend linear isometries to monomial transformations in a general setting, thus making the
    ideas above the same?

\end{frame}

\begin{frame}
    
    \begin{center}
        No!\footnote{Not in general.}
    \end{center}

\end{frame}

\begin{frame}{Finite Structures}
    
    However, if we happen to be in a finite field, then we can make these sorts of claims.

    \medskip

    \pause

    \textbf{Example:} Consider the codes $C=\{0000,0101,0010,0111\}$ and $C'=\{0000,1100,0001,1101\}$ in
    $\mathbb{Z}_2=\{0,1\}$.

    \medskip

    \pause

    Using the permutation $(2\;4\;1\;3)$ on the letters of each of the codewords, we have that $C$
    transforms to $C'$!
    
    \bigskip

    \pause

    It turns out that any linear isometry between $C$ and $C'$ extends to a monomial transformation.\\

    \pause

    Thus, the ideas are the same!

\end{frame}

\begin{frame}{Florence Jessie MacWilliams}
    
    This is not a coincidence, and we have the mathematician Florence Jessie MacWilliams to thank for
    this realization!

    \medskip
    
    \pause

    MacWilliams was a mathematician in the mid 1900's. She too, worked at Bell Labs and did research,
    but she did not get her doctorate until sometime working there. MacWilliams made many contributions
    to the algebra and combinatorics of coding theory. She even wrote a book titled "The Theory of
    Error-Correcting Codes"
    
    \bigskip

    She got her PhD at Harvard where in she proved the result that we will be looking at.

\end{frame}

\begin{frame}{MacWilliams Extension Theorem}

    MacWilliams showed that any two codes in a finite field that have a linear isometry between them,
    this is then a transformation.

    \bigskip
    
    \textbf{MacWilliams Extension Theorem:} Every Hamming weight preserving linear isometry $f$ of
    linear codes $C,C'$ over a finite field $\mathbb{F}$ extends to a monomial transformation $T$ on the
    ambient space $\mathbb{F}^n$.

\end{frame}

\begin{frame}{Generalizations}
    
    One of the next questions could be can we generalize this any further?

    \bigskip

    \pause

    Yes! We have similar extension theorems over other certain finite structures.

\end{frame}

\begin{frame}{Wood's Theorem}
    
    Jay A. Wood proved in 1999 that linear isometries between codes extend to transformations in
    finite Frobenius rings.

    \bigskip

    A finite ring $R$ such that $Soc(_RR)\cong\;_R(R/J(R))$ is a Frobenius ring. Similarly, a finite
    ring $R$ such that $Soc(R_R)\cong(R/J(R))_R$ is a Frobenius ring.

    \bigskip

    \pause

    \textbf{Example:} Any semisimple ring is Frobenius. So, the ring $M_2(\mathbb{Z}_7)$ is Frobenius.

\end{frame}

\begin{frame}{Wood's Theorem}

    \textbf{Theorem:} Any linear isometry $f$ between codes $C,C'$ over a finite Frobenius ring $R$ can
    be extended to a monomial Transformation $T$.

    \bigskip

    \pause
    
    Since then, he and others including Hai Quang Dinh and Sergio R. L\'{o}pez-Permouth have proven
    the converse.

    \bigskip

    \pause

    \textbf{Theorem:} A finite ring $R$ is Frobenius if and only if every linear isometry $f$ between
    codes $C,C'$ can be extended to a monomial transformation $T$.

\end{frame}

\begin{frame}{Current and Future Work}

    With this generalization from finite fields to Frobenius rings, work is still being done to go even
    further.

    \bigskip

    That is, ring theorists are trying to find more general structures for which we can extend
    Macwilliams' Extension Theorem.
    
\end{frame}

\begin{frame}{Other Structures}
    
    For example, could we extend Macwilliams' Extension Theorem to Quasi-Frobenius rings?

    \bigskip

    What about arbitrary finite ring?

    \bigskip

    \pause

    For now, those questions are still unanswered.

\end{frame}

\begin{frame}

    \begin{center}
        Questions?
    \end{center}

\end{frame}

\begin{frame}{Am Ende}

    \begin{center}
        Vielen Dank!
    \end{center}

\end{frame}

\end{document}