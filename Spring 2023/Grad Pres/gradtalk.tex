\documentclass{beamer}
\usetheme{Warsaw}

\usepackage{amsfonts}
\usepackage[mathcal]{eucal}
\usepackage{eufrak}
\usepackage{amssymb}
\usepackage{amsmath}
\usepackage{mathrsfs}

\title{Equivalent Codes From Finite Fields}

\author{Stephanie Amber Klumpe}

\institute{University of Colorado at Colorado Springs}

\date{April 27, 2023}



\begin{document}

\begin{frame}
    \titlepage
\end{frame}

\begin{frame}
    
    Imagine you wanted to send the message "Hi" to someone, but you were concerned about the quality
    or secuity of the data line you were using.\\
    You would likely want to encrypt it, but how would you do that if you wanted to make sure that the
    reciever got the whole message even if errors were present?\\
    We can use redundancy to turn "01001000 01101001" into "0100101 1000110 0110110 1001001".

\end{frame}

\begin{frame}
    
    This is the brain child of Richard Hamming. He spearheaded this area in coding theory called error
    correcting code(ECC). We used his $[7,4]$ code on the world "Hi". This EEC uses the block matrix
    $$G=\left[\begin{array}{ccccccc}
        1 & 0 & 0 & 0 & 1 & 1 & 0 \\
        0 & 1 & 0 & 0 & 1 & 0 & 1 \\
        0 & 0 & 1 & 0 & 0 & 1 & 1 \\
        0 & 0 & 0 & 1 & 1 & 1 & 1
    \end{array}\right]$$
    ECC is used in much of telecom today including computers, cell towers, annd satilite system.\\\

\end{frame}

\begin{frame}
    
    \textbf{Ex:} Let's look at the letter "H". In binary, it is given by "01001000". Notice that this
    binary letter cannot be put through matrix multiplication in the Hamming code. However, each packet
    of four can be. So,
    $$(0 1 0 0)G=0100101\;\;\text{and}\;\; (1 0 0 0)G=1000110$$
    The process for checking for errors comes down to checking if bits in certain positions line up with
    the intended data.

\end{frame}

\begin{frame}

    Historically, the alphabet itself, the binary system, the hexadecimal system and others have been
    used to create codes. Since then, the field of coding theory has made use of more general fields,
    rings, and modules as the setting to create codes.\\
    This then leads to the natural mathematical question of when codes can be consider the same.\\ % edit the next para to more accurately reflect what we want to talk about
    Just as "Ich bin ein berliner" and "I am a doughnut" or "Hello world" and "Rovvy gybvn" are the
    same phrase, given certain codes and conditions, we can say similar things in different alphabets.
    
\end{frame}

\begin{frame}
    
    In many areas of mathematics, we are concerned with maps that preserve certain properties.
    Homomorphisms preserve algebraic properties across algebraic structures, isometries preserve
    distance across metric spaces, and homeomorphisms preserve topological properties across topologies.
    The same idea applies here, in that we want to know when certian maps preserve the Hamming weight of
    a code. We call these particular morphisms linear isometries. % different from metric space ones as these work on finite dimensional vector spaces

\end{frame}

\begin{frame}
    
    The Hamming weight is the number of letters in a codeword that are not the $0$ element from the
    alphabet. So, looking back at the binary representation of "H", "01001000" would have a Hamming
    weight of 2. We denote this weight by $wt(x)$ for $x\in C$.\\
    Another example would be $wt(0,1,2,2,4,0,3,1)=6$ on the alphabet $\{0,1,2,3,4\}$

\end{frame}

\begin{frame}
    
    Once we have these linear isometries, we can then talk about the left monomial transformations.
    These are surjective module homomorphisms $T$ between codes $C,C'$ such that
    $T(x_1,\ldots,x_n)=(x_{\sigma(1)}u_1,\ldots,x_{\sigma(n)}u_n)$ where $\sigma$ is a permutation of
    the set $\{1,\ldots,n\}$ and $u_i\in R$ are units for $1\leq i\leq n$.

\end{frame}

\begin{frame}
    
    The mathematician Florence Jessie MacWilliams was able to prove that every onto linear isometry of
    codes is actually a left monomial transformation between codes when $R$ is a finite field.

\end{frame}

\end{document}

%\begin{frame}
%\end{frame}

% serious reordering of the later 7 frmaes would do nicely.
% build up to the Macwilliams result. Flow through her life a bit, but emphesize the work.
% follow up with the more general results by woods and sergio