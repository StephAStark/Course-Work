\documentclass{beamer}
\usetheme{Warsaw}

\usepackage{amsfonts}
\usepackage[mathcal]{eucal}
\usepackage{eufrak}
\usepackage{amssymb}
\usepackage{amsmath}
\usepackage{mathrsfs}

\title{Equivalent Codes From Finite Fields}

\author{Stephanie Amber Klumpe}

\institute{University of Colorado at Colorado Springs}

\date{April 27, 2023}

\begin{document}

\begin{frame}
    \titlepage
\end{frame}

% \begin{frame}{Abstract}
    
%     Equivalency of codes in regards to encryption becomes a different question when we remove ourselves from
%     the typical alphabet. Historically, people would take messages and use maps on the alphabet to translate
%     the messages into equivalent statements that look different on the surface. In the context of fields,
%     this is where MacWilliams did her work and showed that, under certain circumstances, we can still show
%     certain codes are equivalent even if they look different. We will explore her work and show with
%     examples how this can be done, and then look to what lies beyond MacWilliams' initial work.

% \end{frame}

\begin{frame}{Encoding}
    
    Imagine you wanted to send the message "Hi" to someone, but you were concerned about the quality
    or secuity of the data line you were using.

    \medskip

    \pause

    You would likely want to encrypt it, but how would you do that if you wanted to make sure that the
    reciever got the whole message even if errors were present?

    \medskip

    \pause

    We can use redundancy to turn
    
    $$01001000\;\;01101001$$
    
    into
    
    $$0100101\;\;1000110\;\;0110110\;\;1001001$$

\end{frame}

\begin{frame}{Richard Hamming}
    
    This is the brain child of Richard Hamming. He spearheaded this area in coding theory called error
    correcting code(\textbf{ECC}). We used his $[7,4]$ code on the world "Hi". This \textbf{ECC} uses
    the block matrix:

    $$G=\left[\begin{array}{ccccccc}
        1 & 0 & 0 & 0 & 1 & 1 & 0 \\
        0 & 1 & 0 & 0 & 1 & 0 & 1 \\
        0 & 0 & 1 & 0 & 0 & 1 & 1 \\
        0 & 0 & 0 & 1 & 1 & 1 & 1
    \end{array}\right]$$

    \bigskip

    \pause

    \textbf{ECC} is used in much of telecom today including computers, cell towers, annd satellite system.

\end{frame}

\begin{frame}{Hamming Code}
    
    \textbf{Example:} Let's look at the letter "H". In binary, it is given by
    "01001000". Notice that this binary letter cannot be put through matrix multiplication in the
    Hamming code. However, each packet of four can be. So,

    $$(0 1 0 0)G=0100101\;\;\text{and}\;\; (1 0 0 0)G=1000110$$

    \bigskip

    \pause

    The process for checking for errors comes down to checking if bits in certain positions line up with
    the intended data.

\end{frame}

\begin{frame}{Alphabets}

    We are used to using the english alphabet or the binary system to make codes. Interestingly, we can
    use the elements of a ring or module as our alphabet.

    \bigskip

    \pause

    This then leads to the natural mathematical question of when codes can be consider the same.

    \medskip

    Similar to "Hello world" and "Rovvy gybvn" being the same codes under a mapping, given certain codes
    and conditions, we can say those codes are in some way equivalent.
    
\end{frame}

\begin{frame}{Maps and Equivalencies}
    
    In many areas of mathematics, we are interested in maps that preserve certain properties.

    \medskip

    In algebra we look at homomorphisms, in differential geometry we look at diffeomorphisms, in
    topology we look at homeomorphisms, etc.

    \bigskip

    \pause

    The same idea applies here, in that we look for maps between codes $f:C\to C'$ such that $f$
    preserves the Hamming weight of a code. We call these particular morphisms linear isometries. 

\end{frame}

\begin{frame}{Hamming Weight}
    
    The Hamming weight, denoted $wt(x)$ is the number of letters in a codeword that are not the $0$
    element from the alphabet. So, looking back at the binary representation of "H", "01001000" would
    have a Hamming weight of
    
    $$wt(0,1,0,0,1,0,0,0)$$

    \bigskip

    \pause

    \textbf{Example:} Consider $01224031$. Then,
    
    $$wt(0,1,2,2,4,0,3,1)=6$$
    
    on the alphabet $\{0,1,2,3,4\}$

\end{frame}

\begin{frame}{Maps and Equivalencies}
    
    Another kind of morphism of interest in coding theory is the monomial transformation.\\
    These are surjective module homomorphisms $T:C\to C'$ between codes such that

    $$T(x_1,\ldots,x_n)=(x_{\sigma(1)}u_1,\ldots,x_{\sigma(n)}u_n)$$
    
    where $\sigma$ is a permutation of the set $\{1,\ldots,n\}$ and $u_i\in R$ are units for
    $1\leq i\leq n$.

\end{frame}

\begin{frame}{The Big Question}
    
    Since $f$ and $T$ are both morphisms between codes, the next natural question to ask might be if $f$
    is onto, are they the same? Or if $T$ preserves the Hamming weight are the same?

\end{frame}

\begin{frame}
    
    \begin{center}
        No!\footnote{Not in general.}
    \end{center}

\end{frame}

\begin{frame}{Finite Structures}
    
    Something magical happens however, if we happen to be in a finite field.

    \medskip

    Consider again the binary system $\mathbb{Z}_2=\{0,1\}$. Consider the codes
    $C=\{0000,0101,0010,0111\}$ and $C'=\{0000,1100,0001,1101\}$

    \medskip

    Using the permutation $(2\;4\;1\;3)$, we have that $C$ transforms to $C'$!
    
    \bigskip

    \pause

    It turns out that any linear isometry between $C$ and $C'$ extends to a monomial transformation.\\

    \pause

    Thus, the codes are essentially the same!

\end{frame}

\begin{frame}{MacWilliams Extension Theorem}
    
    This is not a coincidence, and we have the mathematician Florence Jessie MacWilliams to thank for
    this realization!

    \medskip
    
    \pause

    MacWilliams was among the first women in the field of coding theory and for her dissertation, she showed
    that any two codes in a finite field that have a linear isometry between them, this is then a
    transformation.

\end{frame}

\begin{frame}{MacWilliams Extension Theorem}
    
    \textbf{MacWilliams Extension Theorem:} Every Hamming weight preserving linear isometry $f$ of
    linear codes over a finite field $\mathbb{F}$ extends to a monomial transformation $T$ on the
    ambient space $\mathbb{F}^n$.

\end{frame}

\begin{frame}
    
    This allows people to check the uniquness of thier codes up to isometry.

    \medskip

    It also shows why linear \textbf{ECC} work the way they do.

\end{frame}

\begin{frame}{Generalizations}
    
    One of the next question could be can we generalize this any further?

    \bigskip

    \pause

    Yes! We have similar extension theorems over certain finite structures.

\end{frame}

\begin{frame}{Wood's Theorem}
    
    Jay A. Wood proved in 1999 that linear isometries between codes extend to transformations in
    finite Frobenius rings.

    \bigskip

    A ring $R$ such that $Soc(_RR)\cong\;_R(R/J(R))$ is a Frobenius ring.

    \bigskip

    \pause

    \textbf{Example:} Any semisimple ring is Frobenius. So, the ring $M_2(\mathbb{Z}_7)$ is Frobenius.

\end{frame}

\begin{frame}{Wood's Theorem}
    
    Since then, he and others including Hai Quang Dinh and Sergio R. L\'{o}pez-Permouth have proven
    the converse.

    \bigskip

    \pause

    \textbf{Theorem:} Codes in a finite Frobenius are equivalent if and only if every linear isometry
    between them extends to a monomial transformation.

    \bigskip

    \pause

    Work still continues in this field to further generalize these results.

\end{frame}

\begin{frame}

    \begin{center}
        Questions?
    \end{center}

\end{frame}

\begin{frame}{Am Ende}

    \begin{center}
        Vielen Dank!
    \end{center}

\end{frame}

\end{document}