\documentclass{beamer}
\usetheme{Warsaw}

\usepackage{amsfonts}
\usepackage[mathcal]{eucal}
\usepackage{eufrak}
\usepackage{amssymb}
\usepackage{amsmath}
\usepackage{mathrsfs}

\title{Equivalent Codes From Finite Fields}

\author{Stephanie Amber Klumpe}

\institute{University of Colorado at Colorado Springs}

\date{May 1, 2023}

\begin{document}

\begin{frame}
    \titlepage
\end{frame}

\begin{frame}{Encoding}
    
    Imagine you wanted to send the message "Hi" to someone, but you were concerned about the quality
    or secuity of the data line you were using.

    \medskip

    \pause

    You would likely want to encrypt it, but how would you do that if you wanted to make sure that the
    reciever got the whole message even if errors were present?

    \medskip

    \pause

    We can use redundancy to turn
    
    $$01001000\;\;01101001$$
    
    into
    
    $$0100101\;\;1000110\;\;0110110\;\;1001001$$

\end{frame}

\begin{frame}{Richard Hamming}
    
    This is the brain child of Richard Hamming. He was a mathematician in the mid 1900's and did
    extensive research in the field of mathematics at Bell Labs.

    \bigskip

    Many thins are attributed to his name such as Hamming code, Hamming numbers, and Hamming distance.
    He also spearheaded the area in coding theory called error correcting code(\textbf{ECC}).

    \bigskip

    \pause

    \textbf{ECC} is used in much of telecom today including computers, cell towers, annd satellite system.

\end{frame}

\begin{frame}{Hamming Code}
    
    With his work on \textbf{ECC}, he devised a particular version known as the $[7,4]$ code. This uses
    the block matrix

    $$G=\left[\begin{array}{ccccccc}
        1 & 0 & 0 & 0 & 1 & 1 & 0 \\
        0 & 1 & 0 & 0 & 1 & 0 & 1 \\
        0 & 0 & 1 & 0 & 0 & 1 & 1 \\
        0 & 0 & 0 & 1 & 1 & 1 & 1
    \end{array}\right]$$

    to take packets of four and encode them into packets of seven. This is the code that we used on the
    word "Hi" above.

\end{frame}

\begin{frame}{Hamming Code}
    
    \textbf{Example:} Let's look at the letter "H". In binary, it is given by
    "01001000". Notice that this binary codeword cannot be put through matrix multiplication in the
    Hamming code. However, each packet of four can be. So,

    $$(0 1 0 0)G=0100101\;\;\text{and}\;\; (1 0 0 0)G=1000110$$

    \bigskip

    \pause

    The process for checking for errors comes down to checking if bits in certain positions line up with
    the intended data.

\end{frame}

\begin{frame}{Linear Codes}
    
    What we saw above is an example of a linear code. These are \textbf{ECC} for which any linear
    combination of codewords is also a codeword.

    \bigskip

    That is to say, linear codes are finite abelian groups.

\end{frame}

\begin{frame}{Alphabets}

    We are used to using the english alphabet or the binary system to make codes. Thanks to work done
    by ring theorists and coding theorists, we can also use rings and modules.

    \bigskip

    \pause

    This then leads to the natural mathematical question of when codes can be consider the equivalent.

    \medskip

    Similar to "Hello world" and "Rovvy gybvn" being the same codes under a mapping, given certain codes
    and conditions, we can say those codes are in some way equivalent.
    
\end{frame}

\begin{frame}{Maps and Equivalencies}
    
    In many areas of mathematics, we are interested in maps that preserve certain properties.

    \medskip

    In algebra we look at homomorphisms, in differential geometry we look at diffeomorphisms, in
    topology we look at homeomorphisms, etc.

    \bigskip

    \pause

    We will look at, in part, linear isometries. These are maps between vector spaces that preserve the
    norms. In our particular case, the linear isometries $f:C\to C'$ will preserve the Hamming weight
    and the Hamming distance.
    % I make note that we will not be using the distance part, but it does come up with respect to these
    % definitions since isometries preserve distance.

\end{frame}

\begin{frame}{Hamming Weight}
    
    The Hamming weight of a string of letters is the number of nonzero elements from the alphabet and is
    denoted $wt(x)$ for each codeword $x\in C$.

    \medskip

    \pause

    \textbf{Example:} Consider the letter 'i' from above. Then, $x=01101001$ and so

    $$wt(01101001)=4$$

\end{frame}

\begin{frame}{Letters and Words}

    It is important to make the distinction between letters and codewords in this topic.
    
    \bigskip

    While 'i' might be a letter in its own right in the usual alphabet, it becomes an entire codeword
    in binary. This is due to the binary alphabet being $\{0,1\}$.

    \bigskip

    \pause

    So, we need to be aware of our alphabet and context.
    
\end{frame}

\begin{frame}{Maps and Equivalencies}
    
    Another kind of morphism of interest in coding theory is the monomial transformation.\\
    These are module epimorphisms $T:C\to C'$ between codes such that

    $$T(x_1,\ldots,x_n)=(x_{\sigma(1)}u_1,\ldots,x_{\sigma(n)}u_n)$$
    
    where $\sigma$ is a permutation of the set $\{1,\ldots,n\}$ and $u_i\in R$ are units for
    $1\leq i\leq n$.

\end{frame}

\begin{frame}{The Big Question}
    
    Can we extend linear isometries to monomial transformations in a general setting, thus making the
    codes in the maps equivalent?

\end{frame}

\begin{frame}
    
    \begin{center}
        No!\footnote{Not in general.}
    \end{center}

\end{frame}

\begin{frame}{Finite Structures}
    
    However, if we happen to be in a finite field, then we can make these sorts of claims.

    \medskip

    Consider again the binary system $\mathbb{Z}_2=\{0,1\}$. Consider the codes
    $C=\{0000,0101,0010,0111\}$ and $C'=\{0000,1100,0001,1101\}$

    \medskip

    Using the permutation $(2\;4\;1\;3)$, we have that $C$ transforms to $C'$!
    
    \bigskip

    \pause

    It turns out that any linear isometry between $C$ and $C'$ extends to a monomial transformation.\\

    \pause

    Thus, the codes are essentially the same!

\end{frame}

\begin{frame}{Florence Jessie MacWilliams}
    
    This is not a coincidence, and we have the mathematician Florence Jessie MacWilliams to thank for
    this realization!

    \medskip
    
    \pause

    MacWilliams was a mathematician in the mid 1900's. She too, worked at Bell Labs and did research,
    but she did not get her doctorate until sometime working there. MacWilliams made many controbutions
    to the algebra and combenatorics of coding theory. She even wrote a book titled "The Theory of
    Error-Correcting Codes"
    
    \bigskip

    She got her PhD at Harvard where in she proved the result that we will be looking at.

\end{frame}

\begin{frame}{MacWilliams Extension Theorem}

    MacWilliams showed that any two codes in a finite field that have a linear isometry between them,
    this is then a transformation.

    \bigskip
    
    \textbf{MacWilliams Extension Theorem:} Every Hamming weight preserving linear isometry $f$ of
    linear codes over a finite field $\mathbb{F}$ extends to a monomial transformation $T$ on the
    ambient space $\mathbb{F}^n$.

\end{frame}

\begin{frame}
    
    This allows people to check the uniquness of thier codes up to isometry.

    \medskip

    It also shows why linear \textbf{ECC} work the way they do.

\end{frame}

\begin{frame}{Generalizations}
    
    One of the next question could be can we generalize this any further?

    \bigskip

    \pause

    Yes! We have similar extension theorems over other certain finite structures.

\end{frame}

\begin{frame}{Wood's Theorem}
    
    Jay A. Wood proved in 1999 that linear isometries between codes extend to transformations in
    finite Frobenius rings.

    \bigskip

    A ring $R$ such that $Soc(_RR)\cong\;_R(R/J(R))$ is a Frobenius ring. There are other definitions
    of Frobenius rings as well.

    \bigskip

    \pause

    \textbf{Example:} Any semisimple ring is Frobenius. So, the ring $M_2(\mathbb{Z}_7)$ is Frobenius.

\end{frame}

\begin{frame}{Wood's Theorem}
    
    Since then, he and others including Hai Quang Dinh and Sergio R. L\'{o}pez-Permouth have proven
    the converse.

    \bigskip

    \pause

    \textbf{Theorem:} A finite ring is Frobenius if and only if every linear isometry between codes can
    be extended to a monomial transformation.

    \bigskip

    \pause

    Work still continues in this field to further generalize these results.

\end{frame}

\begin{frame}

    \begin{center}
        Questions?
    \end{center}

\end{frame}

\begin{frame}{Am Ende}

    \begin{center}
        Vielen Dank!
    \end{center}

\end{frame}

\end{document}