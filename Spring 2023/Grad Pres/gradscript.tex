\documentclass[12pt]{article}
\usepackage[margin=1in]{geometry}
\usepackage{amsmath}
\usepackage{amssymb}
\usepackage{amsthm}
\usepackage{accents}
\usepackage{graphicx}
\usepackage{listings}
\usepackage{tcolorbox}
\usepackage{lastpage}
\usepackage{fancyhdr}
\usepackage[framed,numbered,autolinebreaks,useliterate]{mcode}
\setlength{\oddsidemargin}{0in}
\setlength{\textwidth}{6.5in}
\setlength{\headheight}{40pt}
\setlength{\textheight}{9in}
\renewcommand \d{\displaystyle}
\pagestyle{fancy}

\newenvironment{solution}
  {\renewcommand\qedsymbol{$\blacksquare$}
  \begin{proof}[Solution]}
  {\end{proof}}
\renewcommand\qedsymbol{$\blacksquare$}

\newcommand{\ubar}[1]{\underaccent{\bar}{#1}}

% Style: no further than column 105 for readability.
%        ensure 'whitespace'
%        use \item[hw #] to enumerate wrt to the book problems

\title{How the Structures and Equivalencies of Rings and Modules Can Yield Equivalent Codes}
\author{Stephanie Amber Klumpe}
\date{April 27, 2023}

\begin{document}

\maketitle
\newpage

\begin{enumerate}
    \item[Abstract:] Equivalency of codes in regards to encryption becomes a different question when we remove ourselves from
the typical alphabet. Historically, people would take messages and use maps on the alphabet to translate
the messages into equivalent statements that look different on the surface. In the context of fields,
this is where MacWilliams did her work and showed that, under certain circumstances, we can still show
certain codes are equivalent even if they look different. We will explore her work and show with
examples how this can be done, and then look to what lies beyond MacWilliams' initial work. 
\end{enumerate}

Imagine you wanted to send a message through a comprimised or unreliable data channel. To ensure that
the desired recipiant got the message, you would want to encrypt your message, not only for security
reasons, but also to make sure that the whole message was delivered. If in particular, you were
concerned that some of the message would get lost in transmition, you could still encode it, however you
would likely want to do it in such a manner that any data lost wouldn't impact the message. One way, and
one of the most noteble ways is to use redundancy. Consider sending the message "Hi". Converting this to
binary, you would get "01001000 01101001". Using what is called Hamming code, we can encode this to
"0100101 1000110 0110110 1001001". This specific way of encoding a message was developed by Richard
Hamming, one of the pioneers of the field of Error Correction Codes and a well known mathematician in
the field of coding theory. For this example, we use his $[7,4]$ code to create the rendundancies. It
has this notation because it takes four bits of data and adds three extra bits in a special manner using
this matrix. The idea of ECC is used quite vastly in  the world of telecom from cell tower and satellite
systems, to computers and even game consols. This allows for the inevitable errors to happen with a way
of detecting them and also correcting some of them. This is becasue pieces of the encoded message can
get corrupted or go missing, and that is already accounted for from the extra three bits per four bit
piece. ECC is a bit of a misnomer as not all errors can be corrected, but with different methods, more
or less errors can be.\\

When it comes to encoding messages, we are used to using the our native alphabet, or even binary in some
cases, but we can actually change that up a bit. Consider for a moment the statements "Ich binn ein
berliner" and "I am a donut". These are the exact same statement. Alternitively, consider "Hello world"
and "Rovvy gybvn". These are also the same phrases. In both of these examples, we can get from one code
to another via a map between alphabets. In the latter case, it's a bit easier to define the map, as all
of the letters are shifted ten places to the right, so the map goes from our alphabet to itself. This is
the essence of detemining when codes are the same in the context of coding theory, even if they look
different on the surface. Like in algebra, we look at homomorphisms, and in topology we look at
homeomorphisms. We will be looking at in part what are
called linear isometries. Generally, these are maps in vector spaces that preserve distance.
For this, we will only concern ourselves with linear codes. These isometries over linear codes
preserve what is called the Hamming weight, again named after Richard Hamming. This weight is simply the
number of nonzero elements in a code with respect to the alphabet. For example the letter "H" in the
word "Hi" has the binary representation "01001000". So, applying the Hamming weight to it, we get
$wt(01001000)=2$ as there are only two nonzero elements in this string relative to the alphabet
$\{0,1\}$. This, however, is not quite enough for us to say that two codes are equivalent in general.
For that, we need a few more ideas.\\

This leads us naturally to monomial transformations. A monomial transformation $T:C\to C'$ is a module
epimorphism given by $T(x_1,\ldots,x_n)=(x_{\sigma(1)}u_1,\ldots,x_{\sigma(n)}u_n)$ where $\sigma$ is a
permutation of the set $\{1,\ldots,n\}$ and $u_i\in R$ are units for $1\leq i\leq n$. This then plays 
into some questions.
SInce these maps are both morphisms between codes, just one of the is surjective, can we say that
isometries extend to monomial transformations? Or if $T$ preserves the Hamming wieght of a code, are
they the same? In true mathematical fashion, the answer is no, not in general. However, some interesting
things happen if we look at specific alphabets.\\

What we are going to do, not just for name sake of this talk but also for ease of results, is look
specifically at finite fields. For example, consider $\mathbb{Z}_2$. From ring theory, we
have that this is a field since 2 is prime. In particular, it is a finite field of 2 elements. So
consider the codes $C=\{0000,0101,0010,0111\}$ and $C'=\{0000,1100,0001,1101\}$. Doing a bit of work, we
can see that each codeword in $C'$ is a permutaion of a codeword in $C$. This is not a couincidence.
For this work, we look at the mathematician FLorence Jessie MacWilliams. She was one of the first women
to publish in the field of coding theory. Specificall, for her doctoral research she proved that any
linear isometry between linear codes can be extended to a monomial transformation in a finite field!
This is known as the MacWilliams extension theorem. This was huge in this area of mathematics. Along
with the ECC's this allowed people creating codes to ensure that the codes they were making were unique
in some regard.

Now, another desire in pure mathematics is to generlize results that we have. So, can we generalize
MacWilliams result to different kinds of structures/alphabets that are not finite fields? As it turns
out, the answer is yes. Work is even still being done in this regard under a few different contexts.
In particular, we have the mathematician Jay A. Wood to look to. He showed that codes over every finite
Frobenius ring can satisfy an analogous version of MacWilliams theorem.

Since his work in this direction, himself and others have shown that the converse holds. When it comes
to these more recent works, we make more use of the generator matricies of linear codes such as the
Hamming code as well as other works done By Richard Hamming including the Hamming distance. Also, as
more work is done, the usage of noncommutative rings and modules comes into play much more heavily
as well as their properties.

\end{document}