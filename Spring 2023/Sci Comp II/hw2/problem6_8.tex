Let $D_N$ be the usual Chebyshev differentiation matrix. Show that the power $(D_N)^{N+1}$ is
identically equal to zero. Now try it on the computer for $N=5$ and $20$ and report the computed 2-norms
$||(D_5)^6||_2$ and $||(D_{20})^{21}||_2$. Discuss.\\\\

\begin{solution}\renewcommand{\qedsymbol}{}\ \\
    We know that the Chebyshev differentiation matrix, $D_N$, is a result of differentiating the
    $N^{th}$ degree Chebyshev polynomial. By the nature of the polynomials, we have that they are given
    by $\sum_{i=1}^Nc_ix^i$ where $c_i$ are the coefficients found by interpolating through the
    Chebyshev nodes. So, taking the derivative would give us an $(N-1)th$ degree polynomial. To find
    the second order differentiation matrix, we apply $D_N$ to $D_N$ to get $D_N^2$. That is, we take
    the second derivative of the Chebyshev polynomial. Continuing the process, we see that

    $$(\sum_{i=1}^Nc_ix^i)^{(N)}=c_N$$

    Hence,

    $$(\sum_{i=1}^Nc_ix^i)^{(N+1)}=(c_N)'=0$$

    So, applying $D_N$ to $(D_N)^N$ gives us $(D_N)^{N+1}=0$ as desired.

    Now, with the following code, we see that we do get a number very close to zero relative to Matlab's precision for
    $N=5$, however, we get an astronomically large number for $N=20$. This is due to the roundinjg and floating number
    capapbilities for the program. So while theoretically, the matrix to the 21st power would yield 0, that doesn't quite
    work in this particular instance of numerical computing.

\end{solution}

\newpage
\lstinputlisting{problem6_8.m}
\newpage 