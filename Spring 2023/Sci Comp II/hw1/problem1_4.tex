Run Program 1 to $N=2^{16}$ instead of $2^{12}$. What happens to the plot of the error vs. $N$? Why? Use
the MATLAB commands tic and toc to generate a plot of approximately how the computation time depends on
N. Is the dependence linear, quadratic, or cubic?\\\\

\begin{solution}\renewcommand{\qedsymbol}{}\ \\
    We can see from these two images that the plot of the error starts to behave differently at the end.
    This is due to the computations near the limit of the computational precision of the program. We can
    see that the error starts to go up as $N$ gets closer to this precision after the point of
    $N=2^{12}$.

    % \begin{figure}[htp]
    %     \centering
    %     \includegraphics[scale=0.18]{212.PNG}
    %     \caption{$N=2^{12}$}
    % \end{figure}
    % \begin{figure}[htp]
    %     \centering
    %     \includegraphics[scale=0.18]{216.PNG}
    %     \caption{$N=2^{16}$}
    % \end{figure}

    Now, as we see by the time graph, the dependance appears to be more linear. Even with the cluster of
    time on the left, the data seems to be well approximated by a linear function.

    % \begin{figure}[htp]
    %     \centering
    %     \includegraphics[scale=0.18]{tictoc4.PNG}
    %     \caption{Time vs $N$}
    % \end{figure}

\end{solution}

\newpage
\lstinputlisting{p1_4.m}
\newpage