We have seen that a discrete function $v$ can be spectrally differentiated by means of two complex
FFTs (one forward, one inverse). Explain how two distinct discrete functions $v$ and $w$ can be
spectrally differentiated at once by the same two complex FFTs, provided that $v$ and $w$ are real.\\\\

\begin{solution}\renewcommand{\qedsymbol}{}\ \\
    Let $v$ and $w$ be two real valued discrete functions. Then define the function $u\equiv v+iw$.
    Then, we have that $u$ is a complex function and we know that we can spectrally differentiate $u$ by
    the use of the FFT and its inverse. So, by applying the FFT to $u$, we get
    $\hat{u}=\hat{v}+i\hat{w}$. We can then spectrally differentiate $u$ and by consequence spectrally
    differentiate $v$ and $w$. Succeding this, we then apply the inverse FFT to $\hat{u}'$ and obtain
    $u'$ and therefore $v'$ and $w'$.

\end{solution}