The time step in Program 37 is specified by $\Delta t = 5/(N_x + N_y^2 )$. Study this discretization 
theoretically  and, if you like, numerically, and decide: Is this the right choice? Can you derive 
a more precise stability limit on $\Delta t$?\\

\begin{solution}\renewcommand{\qedsymbol}{}\ \\
    Taking a look at program 37, the implemented time step is not necessarily bad. Numerically speaking,
    it works well in terms of computation time as well as stability. Now, we know that
    
    $$\Delta t\leq\frac{6}{N_x}$$

    due to the length of the $x$ spacial dimension and the discretization using Fourier nodes, and

    $$\Delta t\leq\frac{8}{N_y^2}$$

    due to the discretization using CHebyshev nodes in $y$. So, theoretically we only need
    $\Delta t\leq\min\{\frac{6}{N_x},\frac{8}{N_y^2}\}$. Implementing this value in the p37 script, we
    still have stability. So, clearly the current $\Delta t$ value is sufficient. However, for clearity
    and conciseness, we can let $\Delta t=\frac{1}{(N_x+N_y)^2}$ as this value is less than the
    theoretical minimum.
\end{solution}