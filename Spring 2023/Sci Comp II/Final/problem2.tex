a. Modify the RBF interpolantion to include a polynomial correction. In 2D, this amounts to searching
the interpolant in the form

$$F(\bf{x}) = \sum_{j=1}^Nc_j\phi(||\bf{x}-\bf{x}_j||) + p(\bf{x})$$

where $p(x) = p(x, y) = \gamma_1 + \gamma_2x + \gamma_3y$. [See Fornberg (2016) Part 1 for details]\\

b. Solve the analogous problem in 3D, stating that given a 3D point cloud, find an implicit surface
that approximates these N points using RBF interpolants.\\

c. Revisit Exercise 1 by using the Machine Learning approach of the RBF Networks, to obtain a RBF best
fit approximation\\\\

\begin{solution}\renewcommand{\qedsymbol}{}\ \\

    a. For this problem, we will use the matrix setup in Fornberg. That is, we will calculate $A$ as we
    did in problem 1, then we will concatinate it with our $Xbig$ matrix, a ones vector, the transpose
    of these, as well as a $3\times3$ zero matrix in the lower left hand corner. Doing this along with
    adding three zeros to the end of our RHS vector allows us to solve the new system with a
    polynomial correction term. As such, we get that $\gamma_1=0.4085, \gamma_2=0.0051$, and
    $\gamma_3=-0.0009$ when using a perturbed elipse as our starting data.\\

    b. For part b, we will look at a torus. Here, we find the point cloud to be:

    \begin{center}
        \includegraphics[scale=0.5]{problem2bcloud.PNG}
    \end{center}

    So, we then compute the normal vector at each point to get:
    
    \begin{center}
        \includegraphics[scale=0.7]{problem2bnorm.PNG}
    \end{center}

    After that, we compute the interpolant as shwon in the code below.

\end{solution}