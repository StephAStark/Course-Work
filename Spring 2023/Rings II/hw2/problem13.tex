Give an example of a ring $R$, and a set of left $R$-modules $\{_RM_{\alpha}|\alpha\in I\}$, and a right
$R$-module $N_R$, for which the abelian groups $N\otimes_R(\prod_{\alpha\in I}M_{\alpha})$ and
$\prod_{\alpha\in I}(N\otimes_RM_{\alpha})$ are not isomorphic. Justify.\\\\

\begin{solution}\renewcommand{\qedsymbol}{}\ \\
    
    Consider $\mathbb{Z}, \mathbb{Q}$ and the $\mathbb{Z}$-modules $\{\mathbb{Z}2^n|n\in\mathbb{Z}^+\}$.
    Then,we see that $Q\otimes\mathbb{Z}2^n=0$ for all $n\in\mathbb{Z}^+$ since
    
    $$\frac{p}{q}\otimes x=\frac{p2^n}{q2^n}\otimes x=\frac{p}{q2^n}\otimes2^nx=$$
    $$\frac{p}{q2^n}\otimes0=0$$

    for all $x\in\mathbb{Z}2^n$. Thus, $\prod_{n\in\mathbb{Z}^+}(\mathbb{Q}\otimes\mathbb{Z}2^n)=0$.
    However, $\mathbb{Q}\otimes(\prod_{n\in\mathbb{Z}^+}\mathbb{Z}2^n)\neq0$. We can see this by taking
    $(1,\ldots,1)\in\prod_{n\in\mathbb{Z}^+}\mathbb{Z}2^n$. Then,
    
    $$\mathbb{Q}\otimes(1,\ldots,1)\cong\mathbb{Q}\otimes\mathbb{Z}\cong\mathbb{Q}$$

    and clearly, $\mathbb{Q}\ncong0$.\footnote{Sorry about the poor quality and lack of work on this
    assignment. I kind of lost my hope and drive knowing that I couldn't continue to the PhD program.
    Again, I'm really sorry and also thank you for all of your help during my time here.}

\end{solution}