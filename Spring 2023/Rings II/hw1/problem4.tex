Let $K$ be any field. Let $G_n$ be the graph

$$  \xymatrix{\bullet  \ar[r] & \bullet \ar[r] &  
{\bullet^v} \ar@(ur,dr) ^{y_1} \ar@(u,r) ^{y_2} \ar@(ul,ur) ^{y_3}
 \ar@{.} @(l,u) \ar@{.} @(dr,dl)
\ar@(r,d) ^{y_n} \ar@{}[l] }$$

Explicitly write down a $K$-algebra isomorphism ${\rm M}_3(L_K(R_n))\cong L_K(G_n)$ (where as usual
$R_n$ is the graph with one vertex and $n$ loops). \\\\

\begin{solution}\renewcommand{\qedsymbol}{}\ \\
    Let $K, G_n$ be as above, and let $F={\rm M}_3(L_K(R_n))$. Let $v$ and $y_i$ for $1\leq i\leq n$ be
    the vertex and edges denoted in the graph above. Further, let $t$ and $u$ be the other two vertices
    from left to right repsctively, and let $e$ and $f$ be the remaining two edges from left to right
    respectively. Consider the mapping $\gamma:L_K(G_n)\to F$ given by:

    $$\gamma(t)=\left(\begin{array}{ccc} 1 & 0 & 0 \\ 0 & 0 & 0 \\ 0 & 0 & 0 \end{array}\right)$$

    $$\gamma(u)=\left(\begin{array}{ccc} 0 & 0 & 0 \\ 0 & 1 & 0 \\ 0 & 0 & 0 \end{array}\right)$$

    $$\gamma(v)=\left(\begin{array}{ccc} 0 & 0 & 0 \\ 0 & 0 & 0 \\ 0 & 0 & 1 \end{array}\right)$$

    $$\gamma(e)=\left(\begin{array}{ccc} 0 & 1 & 0 \\ 0 & 0 & 0 \\ 0 & 0 & 0 \end{array}\right)$$

    $$\gamma(e^*)=\left(\begin{array}{ccc} 0 & 0 & 0 \\ 1 & 0 & 0 \\ 0 & 0 & 0 \end{array}\right)$$

    $$\gamma(f)=\left(\begin{array}{ccc} 0 & 0 & 0 \\ 0 & 0 & 1 \\ 0 & 0 & 0 \end{array}\right)$$

    $$\gamma(f^*)=\left(\begin{array}{ccc} 0 & 0 & 0 \\ 0 & 0 & 0 \\ 0 & 1 & 0 \end{array}\right)$$

    Now witness, without loss of generality in terms of vertices:

    $$\gamma(t)\gamma(t)-\gamma(t)
    =\left(\begin{array}{ccc} 1 & 0 & 0 \\ 0 & 0 & 0 \\ 0 & 0 & 0 \end{array}\right)^2
    -\left(\begin{array}{ccc} 1 & 0 & 0 \\ 0 & 0 & 0 \\ 0 & 0 & 0 \end{array}\right)$$
    $$=\left(\begin{array}{ccc} 1 & 0 & 0 \\ 0 & 0 & 0 \\ 0 & 0 & 0 \end{array}\right)-
    \left(\begin{array}{ccc} 1 & 0 & 0 \\ 0 & 0 & 0 \\ 0 & 0 & 0 \end{array}\right)=0$$

    That is, all vertice mappings are idempotent. Also:

    $$\gamma(t)\gamma(u)=\left(\begin{array}{ccc} 1 & 0 & 0 \\ 0 & 0 & 0 \\ 0 & 0 & 0 \end{array}\right)
    \left(\begin{array}{ccc} 0 & 0 & 0 \\ 0 & 1 & 0 \\ 0 & 0 & 0 \end{array}\right)=0$$

    So, the mappings of the vertices are also orthogonal. Next note that:

    $$\gamma(t)+\gamma(u)+\gamma(v)=
    \left(\begin{array}{ccc} 1 & 0 & 0 \\ 0 & 0 & 0 \\ 0 & 0 & 0 \end{array}\right)
    +\left(\begin{array}{ccc} 0 & 0 & 0 \\ 0 & 1 & 0 \\ 0 & 0 & 0 \end{array}\right)
    +\left(\begin{array}{ccc} 0 & 0 & 0 \\ 0 & 0 & 0 \\ 0 & 0 & 1 \end{array}\right)=I_3$$

    Hence, the sum of the mappings of the vertices is the identiy element. Now, notice

    $$\gamma(t)\gamma(e)=\left(\begin{array}{ccc} 1 & 0 & 0 \\ 0 & 0 & 0 \\ 0 & 0 & 0 \end{array}\right)
    \left(\begin{array}{ccc} 0 & 1 & 0 \\ 0 & 0 & 0 \\ 0 & 0 & 0 \end{array}\right)
    =\left(\begin{array}{ccc} 0 & 1 & 0 \\ 0 & 0 & 0 \\ 0 & 0 & 0 \end{array}\right)=\gamma(e)$$

    $$\gamma(e^*)\gamma(t)=
    \left(\begin{array}{ccc} 0 & 0 & 0 \\ 1 & 0 & 0 \\ 0 & 0 & 0 \end{array}\right)
    \left(\begin{array}{ccc} 1 & 0 & 0 \\ 0 & 0 & 0 \\ 0 & 0 & 0 \end{array}\right)
    =\left(\begin{array}{ccc} 0 & 0 & 0 \\ 1 & 0 & 0 \\ 0 & 0 & 0 \end{array}\right)=\gamma(e^*)$$

    $$\gamma(e)\gamma(u)=\left(\begin{array}{ccc} 0 & 1 & 0 \\ 0 & 0 & 0 \\ 0 & 0 & 0 \end{array}\right)
    \left(\begin{array}{ccc} 0 & 0 & 0 \\ 0 & 1 & 0 \\ 0 & 0 & 0 \end{array}\right)
    =\left(\begin{array}{ccc} 0 & 1 & 0 \\ 0 & 0 & 0 \\ 0 & 0 & 0 \end{array}\right)=\gamma(e)$$

    $$\gamma(u)\gamma(e^*)=
    \left(\begin{array}{ccc} 0 & 0 & 0 \\ 0 & 1 & 0 \\ 0 & 0 & 0 \end{array}\right)
    \left(\begin{array}{ccc} 0 & 0 & 0 \\ 1 & 0 & 0 \\ 0 & 0 & 0 \end{array}\right)
    =\left(\begin{array}{ccc} 0 & 0 & 0 \\ 1 & 0 & 0 \\ 0 & 0 & 0 \end{array}\right)=\gamma(e^*)$$

    $$\gamma(u)\gamma(f)=\left(\begin{array}{ccc} 0 & 0 & 0 \\ 0 & 1 & 0 \\ 0 & 0 & 0 \end{array}\right)
    \left(\begin{array}{ccc} 0 & 0 & 0 \\ 0 & 0 & 1 \\ 0 & 0 & 0 \end{array}\right)
    =\left(\begin{array}{ccc} 0 & 0 & 0 \\ 0 & 0 & 1 \\ 0 & 0 & 0 \end{array}\right)=\gamma(f)$$

    $$\gamma(f^*)\gamma(u)=
    \left(\begin{array}{ccc} 0 & 0 & 0 \\ 0 & 0 & 0 \\ 0 & 1 & 0 \end{array}\right)
    \left(\begin{array}{ccc} 0 & 0 & 0 \\ 0 & 1 & 0 \\ 0 & 0 & 0 \end{array}\right)
    =\left(\begin{array}{ccc} 0 & 0 & 0 \\ 0 & 0 & 0 \\ 0 & 1 & 0 \end{array}\right)=\gamma(f^*)$$

    $$\gamma(f)\gamma(v)=\left(\begin{array}{ccc} 0 & 0 & 0 \\ 0 & 0 & 1 \\ 0 & 0 & 0 \end{array}\right)
    \left(\begin{array}{ccc} 0 & 0 & 0 \\ 0 & 0 & 0 \\ 0 & 0 & 1 \end{array}\right)
    =\left(\begin{array}{ccc} 0 & 0 & 0 \\ 0 & 0 & 1 \\ 0 & 0 & 0 \end{array}\right)=\gamma(f)$$

    $$\gamma(v)\gamma(f^*)=
    \left(\begin{array}{ccc} 0 & 0 & 0 \\ 0 & 0 & 0 \\ 0 & 0 & 1 \end{array}\right)
    \left(\begin{array}{ccc} 0 & 0 & 0 \\ 0 & 0 & 0 \\ 0 & 1 & 0 \end{array}\right)
    =\left(\begin{array}{ccc} 0 & 0 & 0 \\ 0 & 0 & 0 \\ 0 & 1 & 0 \end{array}\right)=\gamma(f^*)$$

    $$\gamma(v)\gamma(y_i)=
    \left(\begin{array}{ccc} 0 & 0 & 0 \\ 0 & 0 & 0 \\ 0 & 0 & 1 \end{array}\right)
    \left(\begin{array}{ccc} 0 & 0 & 0 \\ 0 & 0 & 0 \\ 1 & 0 & 0 \end{array}\right)
    =\left(\begin{array}{ccc} 0 & 0 & 0 \\ 0 & 0 & 0 \\ 1 & 0 & 0 \end{array}\right)=\gamma(y_i)$$

    $$\gamma(e^*)\gamma(e)=
    \left(\begin{array}{ccc} 0 & 0 & 0 \\ 1 & 0 & 0 \\ 0 & 0 & 0 \end{array}\right)
    \left(\begin{array}{ccc} 0 & 1 & 0 \\ 0 & 0 & 0 \\ 0 & 0 & 0 \end{array}\right)
    =\left(\begin{array}{ccc} 0 & 0 & 0 \\ 0 & 1 & 0 \\ 0 & 0 & 0 \end{array}\right)=\gamma(u)$$

    $$\gamma(f)\gamma(f^*)=
    \left(\begin{array}{ccc} 0 & 0 & 0 \\ 0 & 0 & 1 \\ 0 & 0 & 0 \end{array}\right)
    \left(\begin{array}{ccc} 0 & 0 & 0 \\ 0 & 0 & 0 \\ 0 & 1 & 0 \end{array}\right)
    =\left(\begin{array}{ccc} 0 & 0 & 0 \\ 0 & 1 & 0 \\ 0 & 0 & 0 \end{array}\right)=\gamma(u)$$

    and

    $$\gamma(f^*)\gamma(f)=
    \left(\begin{array}{ccc} 0 & 0 & 0 \\ 0 & 0 & 0 \\ 0 & 1 & 0 \end{array}\right)
    \left(\begin{array}{ccc} 0 & 0 & 0 \\ 0 & 0 & 1 \\ 0 & 0 & 0 \end{array}\right)
    =\left(\begin{array}{ccc} 0 & 0 & 0 \\ 0 & 0 & 0 \\ 0 & 0 & 1 \end{array}\right)=\gamma(v)$$

    Finally, notice that for all elements in the image space, the product of any of the vertex mappings
    with any of the edge mappings, that we did not illustrate above, equals zero. So on the free algebra
    with the given relations, we have that $\gamma$ is a homomorphism.
    Next we define the map $\lambda:F\to L_K(G_n)$ by:

    $$\lambda(\left(\begin{array}{ccc} 1 & 0 & 0 \\ 0 & 0 & 0 \\ 0 & 0 & 0 \end{array}\right))=t$$

    $$\lambda(\left(\begin{array}{ccc} 0 & 0 & 0 \\ 0 & 1 & 0 \\ 0 & 0 & 0 \end{array}\right))=u$$

    $$\lambda(\left(\begin{array}{ccc} 0 & 0 & 0 \\ 0 & 0 & 0 \\ 0 & 0 & 1 \end{array}\right))=v$$

    $$\lambda(\left(\begin{array}{ccc} 0 & 1 & 0 \\ 0 & 0 & 0 \\ 0 & 0 & 0 \end{array}\right))=e$$

    $$\lambda(\left(\begin{array}{ccc} 0 & 0 & 0 \\ 1 & 0 & 0 \\ 0 & 0 & 0 \end{array}\right))=e^*$$

    $$\lambda(\left(\begin{array}{ccc} 0 & 0 & 0 \\ 0 & 0 & 1 \\ 0 & 0 & 0 \end{array}\right))=f$$

    $$\lambda(\left(\begin{array}{ccc} 0 & 0 & 0 \\ 0 & 0 & 0 \\ 0 & 1 & 0 \end{array}\right))=f^*$$

    As we see, $\lambda$ is naturally defined and is also a homomorphism by similar logic as above. Now,
    define $\eta$ to be the mapping from $\mathbb{M}_3(K)$ into $L_K(R_n)$. Then we can see that
    $\eta(\gamma(x))$ and $\lambda(\eta^{-1})$ gives the full desired homomorphisms. It
    also appears to be the case that $\lambda$ is the inverse of $\gamma$. Along this line, we see that,
    for all $x\in F$,

    $$\gamma(\lambda(x))=x$$

    and for all $x\in L_K(G_n)$

    $$\lambda(\gamma(x))=x$$

    Therefore, $\gamma$ and $\lambda$ are bijective homomorphisms and thus,
    ${\rm M}_3(L_K(R_n))\cong L_K(G_n)$ as desired.

\end{solution}