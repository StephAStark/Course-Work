For $i\geq 3$, let $C_n$ be the graph consisting of $n$ vertices $\{v_1, v_2, \hdots, v_n\}$,  and $2n$
edges $\{e_1, e_2, \hdots, e_n, f_1, f_2, \hdots, f_n\}$ for which 

$$s(e_i) = v_i,  \ \  r(e_i) = v_{i+1}, \ \  s(f_i) = v_i,  \ \ r(f_i) = v_{i-1},$$

where indices are interpreted ${\rm mod } \ n$. (So, for instance, $r(e_n) = e_1$, and $r(f_1) = v_n$.)
In words, $C_n$ is the graph with $n$ vertices, where each vertex emits two edges, one to both of its
neighboring vertices. In particular, $C_3$ is the graph $E$ of the previous problem. For $n=4$, $n=5$,
and $n=6$, describe $\mathcal{V}(L_K(C_n))$. (Do this by Smith normal form; you need NOT do this
directly as in Question 6a.) For two of these three, you will be able to use this description of
$\mathcal{V}(L_K(C_n))$ to identify $L_K(C_n)$ as a "known" $K$-algebra. (You might need to look at the
monoid $\mathcal{V}(L_K(C_n))$ directly in order to identify $[  L_K(C_n) ]$ within 
$\mathcal{V}(L_K(C_n))$.) Do that. Can you say anything about $L_K(C_n)$ for the third of these?\\\\ 

\begin{solution}\renewcommand{\qedsymbol}{}\ \\
    Using $C_3$ from a previous problem as a starting point, we have that the $A_{C_i}$ are given by

    $$A_{C_4}=\left(\begin{array}{cccc} 1 & 1 & 0 & 1 \\ 1 & 1 & 1 & 0 \\ 
                                        0 & 1 & 1 & 1 \\ 1 & 0 & 1 & 1 \end{array}\right)$$

    $$A_{C_5}=\left(\begin{array}{ccccc} 1 & 1 & 0 & 0 & 1 \\ 1 & 1 & 1 & 0 & 0 \\ 
         0 & 1 & 1 & 1 & 0 \\ 0 & 0 & 1 & 1 & 1 \\ 1 & 0 & 0 & 1 & 1 \end{array}\right)$$

    $$A_{C_6}=\left(\begin{array}{cccccc} 1 & 1 & 0 & 0 & 0 & 1 \\ 1 & 1 & 1 & 0 & 0 & 0 \\
                                          0 & 1 & 1 & 1 & 0 & 0 \\ 0 & 0 & 1 & 1 & 1 & 0 \\
                        0 & 0 & 0 & 1 & 1 & 1 \\ 1 & 0 & 0 & 0 & 1 & 1 \end{array}\right)$$

    Hence, the $I_n-A_{C_i}$ are given by

    $$I-A_{C_4}=\left(\begin{array}{cccc} 1 & -1 & 0 & -1 \\ -1 & 1 & -1 & 0 \\ 
                                        0 & -1 & 1 & -1 \\ -1 & 0 & -1 & 1 \end{array}\right)$$

    $$I-A_{C_5}=\left(\begin{array}{ccccc} 1 & -1 & 0 & 0 & -1 \\ -1 & 1 & -1 & 0 & 0 \\
         0 & -1 & 1 & -1 & 0 \\ 0 & 0 & -1 & 1 & -1 \\ -1 & 0 & 0 & -1 & 1 \end{array}\right)$$

    $$I-A_{C_6}=\left(\begin{array}{cccccc} 1 & -1 & 0 & 0 & 0 & -1 \\ -1 & 1 & -1 & 0 & 0 & 0 \\
                                           0 & -1 & 1 & -1 & 0 & 0 \\ 0 & 0 & -1 & 1 & -1 & 0 \\
                                           0 & 0 & 0 & -1 & 1 & -1 \\ -1 & 0 & 0 & 0 & -1 & 1 
                                        \end{array}\right)$$          

    So, using either the smithForm or the numbertheory Smith Form Calculator depending on whether the
    matrix in question is singular or not, we get the following Smith Normal Forms corresponding to
    $n=4$, $n=5$, and $n=6$ respectively:

    $$\left(\begin{array}{cccc} 1 & 0 & 0 & 0 \\ 0 & 1 & 0 & 0 \\ 
                                0 & 0 & 1 & 0 \\ 0 & 0 & 0 & 3 \end{array}\right)$$

    $$\left(\begin{array}{ccccc} 1 & 0 & 0 & 0 & 0 \\ 0 & 1 & 0 & 0 & 0 \\ 0 & 0 & 1 & 0 & 0 \\ 
                                 0 & 0 & 0 & 1 & 0 \\ 0 & 0 & 0 & 0 & 1 \end{array}\right)$$

    $$\left(\begin{array}{cccccc} 1 & 0 & 0 & 0 & 0 & 0 \\ 0 & 1 & 0 & 0 & 0 & 0 \\
                                  0 & 0 & 1 & 0 & 0 & 0 \\ 0 & 0 & 0 & 1 & 0 & 0 \\
                                  0 & 0 & 0 & 0 & 0 & 0 \\ 0 & 0 & 0 & 0 & 0 & 0 \end{array}\right)$$
\end{solution}

Now, by applying the theorme that was used for problems 8 and 9, we can see that the "known" algebras
are associated with $n=5$ and $n=6$. For $n=5$, we have $\mathbb{Z}_1=\{0\}$ which is just the trivial
algebra. For $n=6$ on the other hand, we have $\mathbb{Z}\times\mathbb{Z}$. Now for $n=4$, we know that
it will look like $\mathbb{Z}_3$, but we don't necessarily have an explicit representation as a Leavitt
algebra.

\newpage
\lstinputlisting{problem10.m}
\newpage