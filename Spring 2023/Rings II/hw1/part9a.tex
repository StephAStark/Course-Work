Compute "directly" the monoid $M_F$. (First find a set of representatives of the equivalence classes in
$M_F$, then prove that these equivalence classes are distinct.)\\\\

\begin{solution}\renewcommand{\qedsymbol}{}\ \\
    Similar to problem 8, the grpah of $F$ has three vertices, and hence we will be dealing with
    $(\mathbb{Z}^{\geq0})^3$. So, $v\mapsto(1,0,0), u\mapsto(0,1,0),$ and $z\mapsto(0,0,1)$. We will mod
    out by $v\mapsto v+u, u\mapsto v+z,$ and $z\mapsto u+z$. That is, we will mod out by the relations:

    $$\mathcal{R}_1:(1,0,0)=(1,0,0)+(0,1,0)=(1,1,0)$$

    $$\mathcal{R}_2:(0,1,0)=(1,0,0)+(0,0,1)=(1,0,1)$$

    and

    $$\mathcal{R}_3:(0,0,1)=(0,1,0)+(0,0,1)=(0,1,1)$$

    So, using these relations, we get:

    $$(1,0,0)\to(1,1,0)\to(1,0,0)+(0,1,0)\to(1,1,0)+(0,1,0)\to(1,2,0)\to$$
    $$(1,0,0)+(0,2,0)\to(1,1,0)+(0,2,0)\to(1,3,0)$$

    and by repeating this process $n$ times for $n\in\mathbb{Z}^+$, we get that

    $$(1,0,0)\to(1,n,0)$$

    Thus, the second component is just 0. Similarly, we have that $(0,0,1)\to(0,n,1)$. Hence, we have
    that

    $$(0,0,1)\to(n,0,n)+(0,0,1)\to(n,0,n)+(0,n,1)\to(2n,0,2n+1)$$

    and continuing this process gives us that the first component is also $0$ and the third component is
    any integer. So, we can conclude that $M_F$ contains the equivalence classes $[(0,0,n)]_n$ for
    $n\in\mathbb{Z}$. It is not to difficult to see that each of the of these equivalence classes is
    unique using invariant weights of 1, and therefore, $M_F=\mathbb{Z}$.

\end{solution}