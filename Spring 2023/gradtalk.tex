\documentclass{beamer}
\usetheme{Warsaw}


\title{If You Like it, You Should of...}

\subtitle{OR: The Applications of Noncommutative Ring Theory}

\author{Stephanie Amber Klumpe}

\institute{University of Colorado at Colorado Springs}

\date{April 27, 2023}


\begin{document}

\begin{frame}
    \titlepage
\end{frame}

\begin{frame}
    \frametitle{Our Journey}
    \tableofcontents
\end{frame}

\section{Section 1}
\subsection{Some Ground Work}

\begin{frame}
\frametitle{What is all This?}

To continue down this road of ring theory, we need to establish a few definitions.

\end{frame}

\begin{frame}

\textbf{Ring:} A ring $R$ is a set with two binary operations $(+)$ and $(*)$ such that the group
$(R,+)$ is commutative, $(*)$ is associative and $(+),(*)$ distribute.\\\

\textbf{Module:} A module $M$ is a commutative group over a ring $R$ such that things hold. We make the
distinction between left $(_RM)$ and right $(M_R)$ modules.\\\

\textbf{Ideal:} An ideal $I$ is a a thing relative to a ring $R$.

\end{frame}

\begin{frame}

With these in place, let's continue to the noncommutative realm.    

\end{frame}

\end{document}

%\begin{frame}
%\end{frame}