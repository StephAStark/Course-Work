\documentclass[12pt]{article}
\usepackage{amsmath}
\usepackage{amssymb}
\usepackage{amsthm}
\usepackage{accents}
\usepackage{graphicx}
\setlength{\oddsidemargin}{0in}
\setlength{\textwidth}{6.5in}
\setlength{\topmargin}{-.55in}
\setlength{\textheight}{9in}
\pagestyle{empty}
\renewcommand \d{\displaystyle}
\begin{document}
\noindent Dallas Klumpe

\noindent Math 5820

\noindent HW 13

12.2.2. Do these data suggest that the direction of the Earth’s magnetic field shifted over the time period spanned by the eruptions? Let $\alpha=0.05$.\\
We are testing $H_0:\mu_1=\mu_2=\mu_3$ against $H_1:$not all$\mu_i$ are equal. Well, given the data, we have that $n=n_1+n_2+n_3=3+3+3=9$, $\bar{y}_1=59.4333, \bar{y}_2=55.9667, \bar{y}_3=51.7,$ and $\bar{y}=\frac19\sum_{i=1}^33(\bar{y}_i)=55.7$. Hence, we have $SSTR=90.02667$ and $SSE=17.6733$. So, our $F=15.28178$. Now, $F_{\alpha}=5.14$. Clearly $F>F_{\alpha}$, so we reject the magnetic field being the same over the years. That is, we can say with $95\%$ confidence that the Earth's magnetic field has shifted between the eruptions.\\[20pt]

12.2.5.  Based on the results shown in the following table, is it conceivable that the four tribes were contemporaries of one another? Let $\alpha=0.01$.\\
We are testing $H_0:\mu_1=\mu_2=\mu_3=\mu_4$. Now, $n=12$, $\bar{y}_1=983.3333, \bar{y}_2=950, \bar{y}_3=1466.6667, \bar{y}_4=1100$, and $\bar{y}=1125$. So, $SSTR=504166.6667$ and $SSE=363333.3333$. Therefore, we have that $F=3.7003$. Now, $F_{\alpha}=7.59$. Since $F<F_{\alpha}$, we fail to reject $H_0$. So, we can say with $99\%$ confidence that the tribes were contemporaries.\\[20pt]

12.2.7. Fill in the entries missing from the following ANOVA table.(Bold text was the given values)
\begin{center}
\begin{tabular}{ c|   c   c   c   c  }
Source & df & SS & MS & F\\
\hline
Treatment & $\textbf{4}$ & 271.36 & 67.68 & $\textbf{6.40}$\\
Error & 10 & 106 & $\textbf{10.60}$ & \\
Total & 14 & $\textbf{377.36}$ &  & 
\end{tabular}
\end{center}



\end{document}