\documentclass[12pt]{article}
\usepackage{amsmath}
\usepackage{amssymb}
\usepackage{amsthm}
\usepackage{accents}
\usepackage{graphicx}
\setlength{\oddsidemargin}{0in}
\setlength{\textwidth}{6.5in}
\setlength{\topmargin}{-.55in}
\setlength{\textheight}{9in}
\pagestyle{empty}
\renewcommand \d{\displaystyle}
\begin{document}
\noindent Dallas Klumpe

\noindent Math 5820

\noindent HW 20

11.3.16.a. Construct a $95\%$ confidence interval for $E(Y|14)$.\\
Well, $\hat{y}=-0.104+0.988(14)=13.728$. Also, the data gives us $\bar{x}=15.0389$, $\sum_{i=1}^{18}x_i=270.7$, $\sum_{i=1}^{18}x_i^2=4170.27$. So, $(14-15.0389)^2=1.0793$ and $s_{xx}=99.2398$. Now, $t_{\alpha/2}=2.1199$. Hence, we have the confidence interval $[13.728-0.202\sqrt{\frac{1}{18}+\frac{1.0793}{99.2398}}, 13.728+0.202\sqrt{\frac{1}{18}+\frac{1.0793}{99.2398}}]=[13.6759, 13.7801]$\\
b. Construct a $95\%$ prediction interval for the volume of a child weighing 14 kg.\\
Using the same information as above, we have the interval $[13.728-0.202\sqrt{1+\frac{1}{18}+\frac{1.0793}{99.2398}}, 13.728+0.202\sqrt{1+\frac{1}{18}+\frac{1.0793}{99.2398}}]=[13.5194, 13.9366]$.\\[20pt]

11.4.1. Let $X$ and $Y$ have the joint pdf $f_{X,Y}(x,y)=\left \{ \begin{array}{cc} \frac{x+2y}{22} & \text{for}  (x,y)=(1,1), (1,3), (2,1), (2,3) \\ 0 & \text{otherwise} \end{array} \right.$. Find Cov$(X,Y)$ and $\rho(X,Y)$.\\
Well, $E(XY)=(\frac{3}{22}+3*\frac{7}{22}+2*\frac{4}{22}+6*\frac{8}{22})=\frac{40}{11}, E(X)=(\frac{3}{22}+\frac{7}{22}+2*\frac{4}{22}+2*\frac{8}{22})=\frac{17}{11}$, and $E(Y)=(\frac{3}{22}+3*\frac{7}{22}+\frac{4}{22}+3*\frac{8}{22})=\frac{26}{11}$. Hence, Cov$(X,Y)=E(XY)-E(X)E(Y)=\frac{40}{11}-\frac{17}{11}\frac{26}{11}=\frac{-2}{121}$. Now, $E(X^2)=(\frac{3}{22}+\frac{7}{22}+4*\frac{4}{22}+4*\frac{8}{22})=\frac{29}{11}$ and $E(Y^2)=(\frac{3}{22}+9*\frac{7}{22}+\frac{4}{22}+9*\frac{8}{22})=\frac{71}{11}$. Therefore, Var$(X)=\frac{29}{11}-(\frac{17}{11})^2=\frac{30}{121}$ and Var$(Y)=\frac{71}{11}-(\frac{26}{11})^2=\frac{105}{121}$. Thus, $\rho(X,Y)=\frac{\frac{-2}{121}}{\sqrt{\frac{30}{121}}\sqrt{\frac{105}{121}}}=\frac{\frac{-2}{121}}{\frac{\sqrt{3150}}{121}}=\frac{-2}{\sqrt{3150}}=\frac{-2}{15\sqrt{14}}=-0.0356$.\\[20pt]

11.4.2. Suppose that $X$ and $Y$ have the joint pdf $f_{X,Y}(x,y)=x+y, 0<x<1, 0<y<1$. Find $\rho(X,Y)$.\\
Well, $f_X(x,y)=\int_0^1x+y dy=x+\frac12$ and $f_Y(x,y)=\int_0^1x+y dx=y+\frac12$. So, $E(X)=\int_0^1x(x+\frac12)=\int_0^1x^2\frac{x}{2}=\frac13+\frac14=\frac{7}{12}$. Since $f_X$ and $f_Y$ are symmetric, we have that $E(Y)=\frac{7}{12}$. Now, $E(X^2)=\int_0^1x^2(x+\frac12)=\int_0^1x^3+\frac{x^2}{2}=\frac14+\frac16=\frac{5}{12}$. Again by symmetry, we have that $E(Y^2)=\frac{5}{12}$. Hence, Var$(X)=$Var$(Y)=\frac{5}{12}-(\frac{7}{12})^2=\frac{11}{144}$. Finally, we have that $E(XY)=\int_0^1\int_0^1xy(x+y) dydx=\int_0^1\int_0^1x^2y+xy^2 dydx=\int_0^1\frac{x^2}{2}+\frac{x}{3}dx=\frac16+\frac16=\frac13$. Therefore, Cov$(X,Y)=\frac13-\frac{7}{12}\frac{7}{12}=\frac{-1}{144}$. Thus, $\rho(X,Y)=\frac{\frac{-1}{144}}{\sqrt{\frac{11}{144}}\sqrt{\frac{11}{144}}}=\frac{-1}{11}$\\[20pt]

11.4.7.a. For random variables $X$ and $Y$, show that Cov$(X+Y, X-Y)=$Var$(X)-$Var$(Y)$.\\
Let $X$ and $Y$ be random varaibales. Well, Cov$(X+Y, X-Y)=E((X+Y)(X-Y))-E(X+Y)E(X-Y)=E(X^2-Y^2)-(E(X)+E(Y))(E(X)-E(Y))=E(X^2)-E(Y^2)-((E(X))^2-(E(Y))^2)=E(X^2)-(E(X))^2-E(Y^2)+(E(Y))^2=$Var$(X)-$Var$(Y)$ as desired.




\end{document}