\documentclass[12pt]{article}
\usepackage{amsmath}
\usepackage{amssymb}
\usepackage{amsthm}
\usepackage{accents}
\usepackage{graphicx}
\setlength{\oddsidemargin}{0in}
\setlength{\textwidth}{6.5in}
\setlength{\topmargin}{-.55in}
\setlength{\textheight}{9in}
\pagestyle{empty}
\renewcommand \d{\displaystyle}
\begin{document}
\noindent Dallas Klumpe

\noindent Math 5820

\noindent HW 14

12.3.3. Use the data to do an F-test, a Tukey test, and a Fisher test$(\alpha=0.05)$.\\
To start, we are testing $H_0:\mu_1=\mu_2=\mu_3$. The data gives us that $n=18, \bar{y}_1=273.8333, \bar{y}_2=204.5, \bar{y}_3=396.6667$, and $\bar{y}=291.6667$. Hence, $SSTR=113646.333$ and $SSE=146753.667$. Then we get that $F=5.80802$. Now, $F_{\alpha}=3.68$. Since $F>F_{\alpha}$, we reject $H_0$. So, not all $\mu_i's$ are equal. So, we will apply Tukey's test. Now, since all smaple sizes are the same, $r=n_1=6$. From above, we have that $MSE=\frac{SSE}{n-k}=\frac{146753.667}{15}=9783.5778$. Next, $rk-k=15$. So, $D=\frac{3.67}{\sqrt{6}}=1.4983$. Then, each of our confidence intervals is $[(\bar{y}_i-\bar{y}_j)-1.4983\sqrt{9783.5778}, (\bar{y}_i-\bar{y}_j)+1.4983\sqrt{9783.5778}]$. This gives us
\begin{center}
\begin{tabular}{ c   c   c  }
$\mu_1-\mu_2$ & [-78.8665, 217.5331] & fail to reject\\
$\mu_1-\mu_3$ & [-271.0335, 25.3661] & fail to reject\\
$\mu_2-\mu_3$ & [-340.3665, -43.9669] & reject
\end{tabular}
\end{center}
For Fisher's test, we have $t_{\alpha/2}=2.132$. Also, each $n_i=6$. Now, we have: 
\begin{center}
\begin{tabular}{ c   c   c  }
$\mu_1-\mu_2$ & [-52.4185, 191.0851] & fail to reject\\
$\mu_1-\mu_3$ & [-244.5852, -1.0815] & reject\\
$\mu_2-\mu_3$ & [-313.9185, -70.4148] & reject
\end{tabular}
\end{center}



\end{document}