\documentclass[12pt]{article}
\usepackage{amsmath}
\usepackage{amssymb}
\usepackage{amsthm}
\usepackage{accents}
\usepackage{graphicx}
\setlength{\oddsidemargin}{0in}
\setlength{\textwidth}{6.5in}
\setlength{\topmargin}{-.55in}
\setlength{\textheight}{9in}
\pagestyle{empty}
\renewcommand \d{\displaystyle}
\begin{document}
\noindent Dallas Klumpe

\noindent Math 5820

\noindent HW 12

9.5.3. Construct two $99\%$ confidence intervals for $\mu_X-\mu_Y$ using the data of Case Study 9.2.3, first assuming the variances are equal, and then assuming they are not.\\
First assume that the variances are equal for the datat of case study 9.2.3. The case study gicves us that $\bar{x}=18.6$ and $\bar{y}=21.9$. Also by the data, we have that $s_p=\sqrt{\frac{115.9929(11)+35.7604(11)}{22}}=8.7107$. Hence our $99\%$ confidence interval is $[(18.6-21.9)-2.8188(8.7107)(\sqrt{\frac{1}{12}+\frac{1}{12}}), (18.6-21.9)+2.8188(8.7107)(\sqrt{\frac{1}{12}+\frac{1}{12}})]=[-13.324, 6.724]$. Next, assume that the variances are unequal. Then, our degrees of freedom is $\nu=\frac{(\frac{115.9929}{35.7604}+\frac{12}{12})^2}{\frac{10.521}{11}+\frac{1}{11}}=17.19$. That is, we have 17 degrees of freedom. So, our $99\%$ confidence interval is $[-3.3-(2.8982)\sqrt{\frac{115.9929}{12}+\frac{35.7604}{12}}, -3.3+(2.8982)\sqrt{\frac{115.9929}{12}+\frac{35.7604}{12}}]=[-13.6064, 7.0064]$.\\[20pt]

9.5.6. Construct a $95\%$ confidence interval for $\frac{\sigma_X^2}{\sigma_Y^2}$ based on the data in Case Study 9.2.1. The hypothesis test referred to tacitly assumed that the variances were equal. Does that agree with your confidence interval? Explain.\\
Well, the sample variances given in the case study are $s_X^2=0.0002103$ and $s_Y^2=0.0000955$. Hence, the $95\%$ confidence interval is $[\frac{0.0002103}{0.0000955}(0.238), \frac{0.0002103}{0.0000955}(4.82)]=[0.5241, 10.6141]$. Now, we are $95\%$ confident that the true proportion of the variances is in this interval. Since 1 lies within this confidence interval, we can say, with $95\%$ confidence, that the variances are equal. So, we can reasonably say that the variances are equal.\\[20pt]

9.5.12. Find a $95\%$ confidence interval for the difference in complication rates. Is there evidence that the new sutures are better?\\
Well, we have that $n=77$ and $m=73$. So, the sample complication rates are $\hat{p}_x=\frac{9}{77}=0.1169$ and $\hat{p}_y=\frac{6}{73}=0.0822$. Then, our confidence interval is $[(0.1169-0.0822)-1.96\sqrt{\frac{.1169(.8831)}{77}+\frac{.0822(.9178)}{73}}, (0.1169-0.0822)+1.96\sqrt{\frac{.1169(.8831)}{77}+\frac{.0822(.9178)}{73}}]=[-0.0608, 0.1302]$. Since we have that the interval is of positive numbers, we can say that we are $95\%$ confident that the true difference in complication rates $p_x-p_y$ is in the interval. So, we can reasonably say that the new sutures are better in that they have a lower complication rate.




\end{document}