\input{settings}

\begin{document}

\lhead{Stephanie Klumpe}

\chead{Problem Set 6}

\rhead{MATH 5510}

\cfoot{\thepage\ of \pageref{LastPage}}

 \begin{enumerate}

   \item (\#2 in 6.5) Prove the general form of Theorem 6.5.3: If $A_{\gamma}$ is a connected subspace of
$X_{\tau}$ for every $\gamma\in \Lambda$ and if $\cap_{\gamma\in\Lambda} A_{\gamma}\neq \emptyset$, then
$\cup_{\gamma\in\Lambda}A_\gamma$ is connected as a subspace of $X$.\\\\

\begin{solution}\renewcommand{\qedsymbol}{}\ \\
    Let $X_{\tau}$ be a topological space and let $\{A_{\gamma}\}_{\gamma\in\Lambda}$ be a collection of
    connected subspaces of $X_{\tau}$. Let $\cap_{\gamma\in\Lambda}A_{\gamma}\neq\emptyset$. Assume that
    $\cup_{\gamma\in\Lambda} A_{\gamma}$ is disconnected. Then
    $\cup_{\gamma\in\Lambda}A_{\gamma}=U\cup V$ where $U$ and $V$ are disjoint, nonempty
    $\tau_{\cup_{\gamma\in\Lambda} A_{\gamma}}-$open sets. Take
    $A_{\gamma}\in\cup_{\gamma\in\Lambda} A_{\gamma}$, and consider $U\cap A_{\gamma}$ and
    $V\cap A_{\gamma}$. Since $U$ and $V$ are disjoint, we have that
    $(U\cap A_{\gamma})\cap(V\cap A_{\gamma})=(U\cap V)\cap A_{\gamma}=\emptyset$. Since $U$ and $V$ are
    open, $U\cap A_{\gamma}$ and $V\cap A_{\gamma}$ are open in the subspace topology by definition.
    Finally we can see that
    $(U\cap A_{\gamma})\cup(V\cap A_{\gamma})=(U\cup V)\cap A_{\gamma}=X\cap A_{\gamma}=A_{\gamma}$. So,
    if $U\cap A_{\gamma}, V\cap A_{\gamma}\neq\emptyset$, then $A_{\gamma}$ is disconnected as a
    subspace, which is a contradiction. So, without loss of generality, say
    $U\cap A_{\gamma}=\emptyset$. Since $\gamma\in\Lambda$ was arbitrary, we have that
    $U\cap A_{\gamma}=\emptyset$ or $V\cap A_{\gamma}=\emptyset$. So either $U\cap A_{\gamma}=\emptyset$
    for all $\gamma\in\lambda$, in which case $U=\emptyset$ which is a contradiction, or
    $U\cap A_{\gamma}=\emptyset$ for $\gamma\in\Gamma\subset\Lambda$ and $V\cap A_{\gamma}=\emptyset$
    for $\gamma\in\Lambda\setminus\Gamma$. For this latter case, we contradict
    $\cap_{\gamma\in\Lambda} A_{\gamma}\neq \emptyset$. Thus, $\cup_{\gamma\in\Lambda} A_{\gamma}$ is
    connected as desired.

\end{solution}
   \pagebreak
   \item (\#2 in 4.2) Show that for $\mathcal{B} = \{(r_1,r_2)| r_1,r_2 \in \mathbb{Q}, r_1 < r_2\}$, we have
$\tau_{\mathcal{B}} = \mathcal{U}$, the usual topology on $\mathbb{R}$. (In order to show that two
topologies are equal, use double containment).\\\\

\begin{solution}\renewcommand{\qedsymbol}{}\ \\
    Let $\mathcal{B}$ and $\mathcal{U}$ be as given. Now, $\tau_{\mathcal{B}}=\{U\subseteq\mathbb{R}|$
    if $x\in U$ then there exists $B\in\mathcal{B}$ such that $x\in B\subseteq U\}$. First, let
    $X\in\tau_{\mathcal{B}}$. If $X=\emptyset$ or $X=\mathbb{R}$, then $X\in \mathcal{U}$. So, let $X$
    be a proper nonempty subset of $\mathbb{R}$. So $X$ is such that if $x\in X$, then there exists
    $(r_1,r_2)\in\mathcal{B}$ for some $r_1<r_2\in\mathbb{Q}$ such that $x\in(r_1,r_2)\subseteq X$. So,
    $X\in\mathcal{U}$ by definition of the usual topology. Now, let $X\in\mathcal{U}$. If $X=\emptyset$
    or $X=\mathbb{R}$, then $X\in\tau_{\mathcal{B}}$. So, let $X\subset\mathbb{R}$ be such that
    $X\neq\emptyset$. So, by definition, if $x\in X$ then there exists $(a,b)$ for some
    $a<b\in\mathbb{R}$ such that $x\in(a,b)\subseteq X$. Now, by the density of the rationals, there
    exist $r_1,r_2\in\mathbb{Q}$ such that $a\leq r_1<x<r_2\leq b$. So, there exists $(r_1,r_2)$ for
    some $r_1,r_2\in\mathbb{Q}$ with $r_1<r_2$ and such that $x\in(r_1,r_2)\subseteq X$. Therefore,
    $X\in\tau_{\mathcal{B}}$. Thus, $\tau_{\mathcal{B}} = \mathcal{U}$ as desired.

\end{solution}
   \pagebreak
   \item a. Let $S$ be a nonempty set and suppose $\mathcal{P}$ is a partition of $S$. Define the relation $R$ on
$S$ by $R=\{(x,y)\in S\times S:x,y\in P$ for some $P\in\mathcal{P}\}$. Prove $R$ is an equivalence
relation.\\

b. Let $G$ be a group. Define $R$ on $G$ by $R=\{(x,y)\in G\times G:y=gxg^{-1}$ for some $g\in G\}$.
Prove $R$ is an eqivalence relation.\\\\

\begin{solution}\renewcommand{\qedsymbol}{}\ \\
    Let $S, \mathcal{P}$, and $R$ be as above. Now, let $s\in S$. Then, $s\in P$ for some
    $P\in\mathcal{P}$. Hence $(s,s)\in R$ and so $R$ is reflexive. Now, let $x,y\in S$ and $(x,y)\in R$.
    Then, $x,y\in P$ for some $P\in\mathcal{P}$. So, $y,x\in P$ and hence $(y,x)\in R$. Thus $R$ is
    symmetric. Finally, let $x,y,z\in S$, $(x,y)\in R$, and $(y,z)\in R$. Then $x,y\in P_1$ and
    $y,z\in P_2$ for some $P_1,P_2\in\mathcal{P}$. Since $\mathcal{P}$ is a partition of $S$, and
    $P_1\cap P_2\neq\varnothing$, $P_1=P_2$. Thus $x,y,z\in P_1$ and so $(x,z)\in R$. Therefore $R$ is
    transitive and an equivalance relation.\\

    Let $G$ and $R$ be as given. Let $x\in G$. Since $G$ is a group, $e\in G$. Now, $x=exe^{-1}$. Thus
    $(x,x)\in R$ and so $R$ is reflexive. Now, let $x,y\in G$ and $(x,y)\in R$. Then $y=gxg^{-1}$ for
    some $g\in G$. So, $x=g^{-1}yg=g^{-1}y(g^{-1})^{-1}$. Hence $(y,x)\in R$ and $R$ is symmetric.
    Finally let $x,y,z\in G$, $(x,y)\in R$, and $(y,z)\in R$. Then $y=gxg^{-1}$ and $z=hyh^{-1}$ for
    some $g,h\in G$. So, $z=hgxg^{-1}h^{-1}=hgx(hg)^{-1}$, and thus $(x,z)\in R$ since $G$ is a group
    and closed under the operation. Therefore $R$ is transitive and an equivalence relation as desired.

\end{solution}
   \pagebreak
   \item Let $a_n=(\frac{4+2(-1)^n}{5})^n$.

\begin{enumerate}

    \item Determine the $\lim\sup$ and $\lim\inf$ of the sequence via the ratio and the root test.\\\\

\begin{solution}\renewcommand{\qedsymbol}{}\ \\
    Well,
    
    $$\lim\sup(a_n)^{\frac1n}=\lim\sup((\frac{4+2(-1)^n}{5})^n)^{\frac1n}=$$
    $$\lim\sup\frac{4+2(-1)^n}{5}=\lim\frac65=\frac65$$
    
    Also,
    
    $$\lim\inf(a_n)^{\frac1n}=\lim\inf((\frac{4+2(-1)^n}{5})^n)^{\frac1n}=$$
    $$\lim\inf\frac{4+2(-1)^n}{5}=\lim\frac25=\frac25$$
    
    Now,
    
    $$\lim\sup|\frac{a_{n+1}}{a_n}|=$$
    $$\lim\sup|\frac{(4+2(-1)^{n+1})^{n+1}}{5^{n+1}}\times\frac{5^n}{(4+2(-1)^n)^n}|=$$
    $$\lim\sup\frac{4+2(-1)^{n+1}}{5(4+2(-1)^n)}=\lim\frac{6^{n+1}}{5(2^n)}=\infty$$
    
    Finally, we have
    
    $$\lim\inf|\frac{a_{n+1}}{a_n}|=\lim\inf\frac{4+2(-1)^{n+1}}{5(4+2(-1)^n)}=$$
    $$\lim\frac{2^{n+1}}{5(6^n)}=0$$

\end{solution}
    \pagebreak
    \item Based on the work in part a, does the series converge or diverge? What about the series
$\sum_{i=1}^{\infty}(-1)^na_n$?\\\\

\begin{solution}\renewcommand{\qedsymbol}{}\ \\
    The series $\sum_{n=1}^{\infty} a_n$ does not converge by both the ratio and the root test as
    calculated above. Also, the series $\sum_{n=1}^{\infty} (-1)^na_n$ diverges by the alternating
    series test.

\end{solution}
    \pagebreak
    \item Determine the interval of convergence of the power series based off the sequence $a_n$.\\\\

\begin{solution}\renewcommand{\qedsymbol}{}\ \\
    The power series $\sum_{n=1}^{\infty} a_nx^n$ has a radius of convergence of $\frac56$ from above.
    Now both $\sum_{n=1}^{\infty}a_n(\frac56)^n$ and $\sum_{n=1}^{\infty}a_n(\frac{-5}{6})^n$ diverge to
    infinity, so the exact interval of convergence is $(-\frac56,\frac56)$.

\end{solution}

  \end{enumerate}
   \pagebreak
   \item (\# 7a in 6.2) A space $X_{\tau}$ is said to be \textit{totally disconnected} if every subspace of $X$
with more than one element is disconnected (in the subspace topology). Show that every discrete space is
totally disconnected.\\

\begin{solution}\renewcommand{\qedsymbol}{}\ \\
    Let $X$ be a set with the discrete topology. If $X$ itself has only one element, then
    $X_{\mathcal{D}}$ is vacuously totally disconnected. So, let $Card(X)\geq2$. Now, let $U$ be a
    subspace of $X_{\mathcal{D}}$ such that $Card(U)\geq2$. Then, consider the sets $V\subset U$ such
    that $V\neq\emptyset$ and $U\setminus V$. Hence $V, U\setminus V\neq\emptyset$,
    $V\cap(U\setminus V)=\emptyset$, and $V\cup(U\setminus V)=U$. Now, $V,(U\setminus V)$ are both open
    in the subspace topology, since $V$ and $(U\setminus V)$ are both open in $X_{\mathcal{D}}$, so $U$
    is disconnected. Since $U$ was an arbitrary subspace with $Card(U)\geq2$, $X_{\mathcal{D}}$ is
    totally disconnected. Since $X_{\mathcal{D}}$ was also arbitrary, we have that every discrete space
    is totally disconnected as desired.

\end{solution}
   \pagebreak
   \item (\#7 in 7.3) Prove the Closed Graph Theorem: If $f: X_{\tau} \to Y_{\nu}$ is continuous, with $Y$ both
compact and Hausdorff, then the graph

$$G_f = \{(x,y) \in X\times Y | y=f(x)\}$$

is closed in $X_{\tau}\times Y_{\nu}$.\\\\

\begin{solution}\renewcommand{\qedsymbol}{}\ \\
    Let $X_{\tau}$ and $Y_{\nu}$ be topological spaces with $Y_{\nu}$ compact and Hausdorff and let
    $f:X_{\tau}\rightarrow Y_{\nu}$ be continuous. We will show that $(X\times Y)\setminus G_f$ is open.
    So, let $(x,y)\in(X\times Y)\setminus G_f$. So, for this $(x,y)$, we have that $y\neq f(x)$. Since
    $Y$ is Hausdorff, there exist disjoint $\nu-$open sets $U,V$ such that $y\in U$ and $f(x)\in V$. Let
    $M_{f(x)}$ be an open neighborhood of $f(x)$. Since $f$ is continuous, there exists a $\tau$-open
    neighborhood, $N_x$, of $x$ such that $N_x\subseteq f^{-1}(M_{f(x)})$. Since
    $y\neq f(x), U\subset Y$ and $(x,y)\in N_x\times U\subset(X\times Y)\setminus G_f$, and since
    $(x,y)\in(X\times Y)\setminus G_f$ was arbitrary, we have that $(X\times Y)\setminus G_f$ is open.
    Thus, $G_f$ is closed in $X\times Y$ as desired.

\end{solution}

 \end{enumerate}

\end{document}

%\item Prove that for $A$ and $B$ any subsets of a topological space $X_{\tau}$,\\ 
% $\text{Bdy($A\cup B$)}\subseteq \text{Bdy($A$)}\cup\text{Bdy($B$)}$.

%\item (\#4 in 4.5) Let $A\subseteq X_{\tau}$ and let $f:X_{\tau} \to Y_{\nu}$ be continuous. If $x$ is
% a limit point of $A$, must $f(x)$ be a limit point of $f(A)\subseteq Y$? Explain.

%Now let $U$ be an open set in $X_{P(i)}$ and let $p_i$ be the projection map
% $\prod X_{P(i)}\rightarrow X_{P(i)}$. So, $(p_{i}\circ f)^{-1}(U)=f^{-1}(p_{i}^{-1}(U))$. Well,
% $p_{i}$ is continuous for all $i\in\mathbb{N}$, so $p_{i}^{-1}(U)$ is open in $\prod X_{P(i)}$. Then
% for every $(x_{P(1)},x_{P(2)},\ldots)\in p_i^{-1}(U)$, there exists
% $V_{P(1)}\times V_{P(2)}\times\cdots\in\tau_{P(1)}\times\tau_{P(2)}\times\cdots$ such that
% $(x_{P(1)}, x_{P(2)},\ldots)\in V_{P(1)}\times V_{P(2)}\subseteq p_i^{-1}(U)$ where
% $\tau_{P(1)}\times\tau_{P(2)}\times\cdots$ is the product topology and $\tau_{P(i)}$ is the respective
% topology on $X_{P(i)}$. So, $f^{-1}(p_{i}^{-1}(U))$ is open in $\prod X_{i}$ since for every
% $(x_1, x_2,\ldots)\in f^{-1}(p_{i}^{-1}(U))$ we have that
% $(x_1, x_2,\ldots)\in V_1\times V_2\times\cdots\subseteq f^{-1}(p_{i}^{-1}(U))$ where
% $\prod V_{P(i)}$ is the permutation of $\prod V_i$, and so $\prod V_i\in\prod\tau_i$. Since
% $i\in\mathbb{N}$ was arbitrary, we have that $p_{i}\circ f$ is continuous for every $i$. Therefore
% $f$ is continuous. Now, let $U$ be open in $\prod X_{i}$. So, for every $(x_1, x_2,\ldots)\in U$,
% there exists $V_1\times V_2\times\cdots\in\tau_1\times\tau_2\times\cdots$ such that
% $(x_1, x_2,\ldots)\in V_1\times V_2\times\cdots\subseteq U$ where $\prod \tau_i$ is the product
% topology and $\tau_i$ is the topology on $X_i$ respectively. So, for every
% $(x_{P(1)}, x_{P(2)},\ldots)\in f(U)$, we have
% $(x_{P(1)}, x_{P(2)},\ldots)\in V_{P(1)}\times V_{P(2)}\times\cdots\subseteq f(U)$. Therefore, $f(U)$
% is open in $\prod X_{P(i)}$. Thus, $f$ is an open map. Therefore $f$ is a homeomorphism and so
% $X_{P(i)}\cong\prod X_{P(i)}$. 