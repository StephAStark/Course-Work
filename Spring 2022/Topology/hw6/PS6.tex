\documentclass[12pt]{article}
\usepackage[margin=1in]{geometry}
\usepackage{amsmath}
\usepackage{amssymb}
\usepackage{amsthm}
\usepackage{accents}
\usepackage{graphicx}
\usepackage{listings}
\usepackage{tcolorbox}
\usepackage{lastpage}
\usepackage{fancyhdr}
\usepackage[framed,numbered,autolinebreaks,useliterate]{mcode}
\setlength{\oddsidemargin}{0in}
\setlength{\textwidth}{6.5in}
\setlength{\headheight}{40pt}
\setlength{\textheight}{9in}
\renewcommand \d{\displaystyle}
\pagestyle{fancy}

\newenvironment{solution}
  {\renewcommand\qedsymbol{$\blacksquare$}
  \begin{proof}[Solution]}
  {\end{proof}}
\renewcommand\qedsymbol{$\blacksquare$}

\newcommand{\ubar}[1]{\underaccent{\bar}{#1}}

% Style: no further than column 105 for readability.
%        ensure 'whitespace'
%        use \item[hw #] to enumerate wrt to the book problems

\begin{document}

\lhead{Stephanie Klumpe}

\chead{Problem Set 6}

\rhead{MATH 5510}

\cfoot{\thepage\ of \pageref{LastPage}}

 \begin{enumerate}

   \item Describe and implement a fourth-order Runge-Kutta and Fourier method for the Burger equation with
periodic boundary conditions:

$$u_t = \epsilon u_{xx} + uu_x,\;\;x\in(-\pi,\pi),\;\;u(x,0) = e^{-10\sin^2(x/2)}$$

with $\epsilon = 0.03$ and the simulation running up to $t = 1$.
   \pagebreak
   \item Consider Fisher's equation:

$$u_t = u_{xx} + u(1 - u)$$

(i) The only traveling "front" wasve $u(x,t) = \Phi(x - ct)$ with $\Phi(-\infty)=1$ and $\Phi(\infty)=0$
is given by:

$$u(x,t) = [1+exp(\frac{x}{\sqrt{6}} - \frac56t)]^{-2}$$

Verify this assertion by deriving the ODE that $\Phi$ has to satisfy and then use a numerical method to
solve it.[Also try analytically!]\\

(ii) Since $u(x,t)\to0$ (resp. 1) as $x\to\infty$ (resp. $-\infty$), we can approximate the traveling
wave given in (i) in $(-L,L)$ wehre $L$ is large enough so that the wave front does not reach the
boundary $x=L$, by imposing the boundary conditions

$$u(-L,t) = 1,\;\;u(L,t) = 0$$

and taking the initial value as $u(x,0)$. Design and implement a Chebyshev collocation method to
approximate the solution of the PDE and study the convergence if the approximate solution as $N$ is
increasing.
   \pagebreak
   \item If $A$ is a subspace of $X_{\tau}$, the inclusion map $j:A\hookrightarrow X$ is continuous.\\\\


\begin{solution}\renewcommand{\qedsymbol}{}\ \\
    Let $A$ be a subspace of $X_{\tau}$. Let $V$ be a $\tau-$open subset of $X$. Assume first that
    $V\cap A=V$. Then $j^{-1}(V)\subseteq A$ and
    $j^{-1}(V)=j^{-1}(V\cap A)=j^{-1}(V)\cap j^{-1}(A)=V\cap A=V$ by definition of the inclusion map.
    Thus $V$ is $\tau_{A}-$open by definition of the subspace topology. Next assume that
    $V\cap A=\emptyset$. Then, $j^{-1}(V)=\emptyset$ by the definition of the inclusion map. Hence
    $j^{-1}(V)$ is open since $\tau_{A}$ is a topology. Finally assume that $V\cap A\neq V$ and
    $V\cap A\neq\emptyset$. Then we can take $V=V_1\cup V_2$ such that $V_1\cap V_2=\emptyset$,
    $V_1\cap A=V_1$, and $V_2\cap A=\emptyset$. Then
    $j^{-1}(V)=j^{-1}(V_1\cup V_2)=j^{-1}(V_1)\cup j^{-1}(V_2)=V_1$ be definition of the inclusion map.
    Since $V_1=V_1\cap A$, $j^{-1}(V)$ is $\tau_{A}-$open. Thus, $j:A\hookrightarrow X$ is continuous.

\end{solution}
   \pagebreak
   \item (\#7 in 4.3) Prove that if $A$ is any subset of the indiscrete space $X_{\mathcal{I}}$, with
$\text{Card($A$)}\geq 2$, then $A' = X$.\\\\

\begin{solution}\renewcommand{\qedsymbol}{}\ \\
    Let $A\subseteq X_{\mathcal{I}}$ with Card$(A)\geq2$. We have that $A'\subseteq X$ by definition.
    So, let $x\in X$. Since $\mathcal{I}=\{X,\emptyset\}$, every neighborhood of $x$ is $X$ itself.
    Since $A\subseteq X_{\mathcal{I}}$ with Card$(A)\geq2$, $(X\setminus\{x\})\cap A\neq\emptyset$.
    Therefore $x\in A'$. Thus, $X=A'$.

\end{solution}
   \pagebreak
   \item (\#8 a, b in 2.2) Let $f:X\to Y$ and $g:Y\to Z$ be any functions.
Prove that if $f$ is one-to-one and $g$ is one-to-one, then $g\circ f:X\to Z$ is one-to-one. Is the
converse true?\\
If $g$ is onto and $f$ is onto, then is $g\circ f$ always onto? Is the converse true?\\

\begin{solution}\renewcommand{\qedsymbol}{}\ \\
    Let $f$ and $g$ be one-to-one functions as given above with $X, Y,$ and $Z$ as sets. Now, assume
    that $g(f(x_1))=g(f(x_2))$ for some $x_1, x_2\in X$. Since $g$ is injective, $f(x_1)=f(x_2)$. Also,
    since $f$ is injective, $x_1=x_2$ and thus $g\circ f:X\rightarrow Z$ is injective as desired. The
    converse is not true however. Consider $f:\mathbb{R}\rightarrow[0,\infty)$ and
    $g:[0,\infty)\rightarrow[0,\infty)$ given by $f(x)=x^2$ and $g(x)=\sqrt{x}$. Clearly
    $g(f(x))=x$ is one-to-one, but $f(x)$ is not.\\

    Let $f$ and $g$ be as given with sets $X, Y$, and $Z$. Assume that $f$ and $g$ are onto. Now, let
    $z\in Z$. Since $g$ is onto, there exists $y\in Y$ such that $g(y)=z$. Now, since $f$ is onto, there
    exists $x\in X$ such that $f(x)=y$. Thus, there exists $x\in X$ such that $g(f(x))=z$ and so
    $g\circ f:X\rightarrow Z$ is onto as desired. The converse is not always true. Consider
    $f:[0,\infty)\rightarrow\mathbb{R}$ and $g:\mathbb{R}\rightarrow[0,\infty)$ defined by
    $f(x)=\sqrt[3]{x}$ and $g(x)=x^2$. So clearly $g(f(x))=x^{\frac23}$ is onto, but $f$ is not.

\end{solution}
   \pagebreak
   \item (\#3 in 2.5) Verify that the set $\{1,4,7,10,\ldots\}$ is infinite, by Definition 2.5.2.\\\\

\begin{solution}\renewcommand{\qedsymbol}{}\ \\
    Let $A=\{1,4,7,10,\ldots\}$. Define $B=\{4,7,10,13,\ldots\}\subset A$. Also define
    $f:A\rightarrow B$ by $f(x)=x+3$. Now, assume that $f(x_1)=f(x_2)$. So, $x_1+3=x_2+3$, whence
    $x_1=x_2$. Hence $f$ is injective. Now $y\in Y$ such that $y=x+3$ for some $x\in X$. Therefore,
    $x=y-3$. So, $f(x)=f(y-3)=(y-3)+3=y$. Since $y$ was arbitrary, we have that $f$ is also onto. So,
    $f$ is bijective. Thus, $A=B$ and we have that $A$ is equal to a proper subset of itself. Thus by
    definition, $A$ is infinite as desired.

\end{solution}

 \end{enumerate}

\end{document}

%\item Prove that for $A$ and $B$ any subsets of a topological space $X_{\tau}$,\\ 
% $\text{Bdy($A\cup B$)}\subseteq \text{Bdy($A$)}\cup\text{Bdy($B$)}$.

%\item (\#4 in 4.5) Let $A\subseteq X_{\tau}$ and let $f:X_{\tau} \to Y_{\nu}$ be continuous. If $x$ is
% a limit point of $A$, must $f(x)$ be a limit point of $f(A)\subseteq Y$? Explain.

%Now let $U$ be an open set in $X_{P(i)}$ and let $p_i$ be the projection map
% $\prod X_{P(i)}\rightarrow X_{P(i)}$. So, $(p_{i}\circ f)^{-1}(U)=f^{-1}(p_{i}^{-1}(U))$. Well,
% $p_{i}$ is continuous for all $i\in\mathbb{N}$, so $p_{i}^{-1}(U)$ is open in $\prod X_{P(i)}$. Then
% for every $(x_{P(1)},x_{P(2)},\ldots)\in p_i^{-1}(U)$, there exists
% $V_{P(1)}\times V_{P(2)}\times\cdots\in\tau_{P(1)}\times\tau_{P(2)}\times\cdots$ such that
% $(x_{P(1)}, x_{P(2)},\ldots)\in V_{P(1)}\times V_{P(2)}\subseteq p_i^{-1}(U)$ where
% $\tau_{P(1)}\times\tau_{P(2)}\times\cdots$ is the product topology and $\tau_{P(i)}$ is the respective
% topology on $X_{P(i)}$. So, $f^{-1}(p_{i}^{-1}(U))$ is open in $\prod X_{i}$ since for every
% $(x_1, x_2,\ldots)\in f^{-1}(p_{i}^{-1}(U))$ we have that
% $(x_1, x_2,\ldots)\in V_1\times V_2\times\cdots\subseteq f^{-1}(p_{i}^{-1}(U))$ where
% $\prod V_{P(i)}$ is the permutation of $\prod V_i$, and so $\prod V_i\in\prod\tau_i$. Since
% $i\in\mathbb{N}$ was arbitrary, we have that $p_{i}\circ f$ is continuous for every $i$. Therefore
% $f$ is continuous. Now, let $U$ be open in $\prod X_{i}$. So, for every $(x_1, x_2,\ldots)\in U$,
% there exists $V_1\times V_2\times\cdots\in\tau_1\times\tau_2\times\cdots$ such that
% $(x_1, x_2,\ldots)\in V_1\times V_2\times\cdots\subseteq U$ where $\prod \tau_i$ is the product
% topology and $\tau_i$ is the topology on $X_i$ respectively. So, for every
% $(x_{P(1)}, x_{P(2)},\ldots)\in f(U)$, we have
% $(x_{P(1)}, x_{P(2)},\ldots)\in V_{P(1)}\times V_{P(2)}\times\cdots\subseteq f(U)$. Therefore, $f(U)$
% is open in $\prod X_{P(i)}$. Thus, $f$ is an open map. Therefore $f$ is a homeomorphism and so
% $X_{P(i)}\cong\prod X_{P(i)}$. 