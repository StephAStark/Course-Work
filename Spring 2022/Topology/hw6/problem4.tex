(\#5 in 5.2) Prove Theorem 5.2.3: If $X_{\tau}$ and $Y_{\sigma}$ are any topological spaces, with
basepoints $x_0\in X$ and $y_0\in Y$, then the inclusion maps

$$i_X:X_{\tau}\hookrightarrow X_{\tau}\times Y_{\sigma}$$

and

$$i_Y:Y_{\sigma}\hookrightarrow X_{\tau}\times Y_{\sigma}$$

are both continuous, where $X_{\tau}\times Y_{\sigma}$ denotes the Cartesian product endowed with the
product topology.\\\\

\begin{solution}\renewcommand{\qedsymbol}{}\ \\
Let $X_{\tau}, Y_{\sigma}, x_0, y_0, i_X$, and $i_Y$ be as stated above. Now we need only show that
$p_X\circ i_X, p_Y\circ i_X, p_X\circ i_Y$ and $p_Y\circ i_Y$ are continuous. We will start with $i_X$.
Let $U$ be an open set in $X_{\tau}$. Now, $(p_x\circ i_x)^{-1}(U)=i_X^{-1}(p_X^{-1}(U))$. Since $p_X$
is continuous, we have that $p_X^{-1}(U)$ is open in $X_{\tau}\times Y_{\sigma}$. Now,
$p_X^{-1}(U)=\{(x,y)\in X_{\tau}\times Y_{\sigma}|p_X((x,y))=x\in U\}=U\times Y$. So,
$i_X^{-1}(p_X^{-1}(U))=\{x\in X|i_X(x)=(x,y_0)\in p_X^{-1}(U)\}=U$, so $p_X\circ i_X$ is continuous.
Now, let $U$ be an open set in $Y_{\sigma}$. Well, $(p_Y\circ i_x)^{-1}(U)=i_X^{-1}(p_Y^{-1}(U))$ and
$p_Y^{-1}(U)=\{(x,y)\in X_{\tau}\times Y_{\sigma}|p_Y((x,y))=y\in U\}=X\times U$ is open in
$X_{\tau}\times Y_{\sigma}$ since $p_Y$ is continuous. Now $i_X^{-1}(p_Y^{-1}(U))=\emptyset$ if
$y_0\notin U$ since for all $x\in X$, $i_X(x)=(x,y_0)\notin p_Y^{-1}(U)$. If, on the other hand,
$y_0\in U$, then $i_X^{-1}(p_Y^{-1}(U))=X$ since for all $x\in X$, $i_X(x)=(x,y_0)\in p_Y^{-1}(U)$. In
either case, $i_X^{-1}(p_Y^{-1}(U))$ is open in $X_{\tau}$, so $p_Y\circ i_X$ is continuous. Therefore,
by one of the theorem 5.2.2, $i_X$ is contiuous. Now we will show $i_Y$ is continuous. Let $U$ be an
open set in $Y_{\sigma}$. Then $(p_Y\circ i_Y)^{-1}(U)=i_Y^{-1}(p_Y^{-1}(U))$ and $p_Y^{-1}(U)$ is
open in $X_{\tau}\times Y_{\sigma}$ since $p_Y$ is continuous. So,
$i_Y^{-1}(p_Y^{-1}(U))=\{y\in Y|i_Y(y)=(x_0,y)\in p_Y^{-1}(U)\}=U$ so $p_Y\circ i_Y$ is continuous. Now
let $U$ be open in $X_{\tau}$. We have that $(p_X\circ i_Y)^{-1}(U)=i_Y^{-1}(p_X^{-1}(U))$ where
$p_X^{-1}(U)$ is open in $X_{\tau}\times Y_{\sigma}$ since $p_X$ is continuous. Now
$i_Y^{-1}(p_X^{-1}(U))=\emptyset$ if $x_0\notin U$ since for all
$y\in Y, i_Y(y)=(x_0,y)\notin p_X^{-1}(U)$. On the other hand $i_Y^{-1}(p_X^{-1}(U))=Y$ if $x_0\in U$
since for all $y\in Y, i_Y(y)=(x_0,y)\in p_X^{-1}(U)$. So, $(p_Y\circ i_Y)$ and by the same theorem as
above, $i_Y$ is also continuous. So, $i_X$ and $i_Y$ are both continuous.

\end{solution}