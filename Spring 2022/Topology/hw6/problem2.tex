(\#6 in 4.4) Let $X_{\tau}$ be a topological space and let $A\subseteq X$.\\
Show that the sets $\text{Int}(A), \text{Bdy}(A)$ and $\text{Ext}(A)$ are pairwise disjoint.\\\\

\begin{solution}\renewcommand{\qedsymbol}{}\ \\
    Let $X_{\tau}$ and $A$ be as given. We need to show that
    $\text{Int}(A)\cap\text{Bdy}(A)=\emptyset, \text{Int}(A)\cap\text{Ext}(A)=\emptyset,$ and
    $\text{Bdy}(A)\cap\text{Ext}(A)=\emptyset$. Well, $\text{Int}(A)\cap\text{Bdy}(A)=\emptyset$ since
    for every $x\in$Bdy$(A)$, no neighborhood of $x$, $N_x$, is a subset of $A$, and for every
    $x\in$Int$(A)$, there exists a neighborhood of $x, N_x$, such that $N_x\subseteq A$. Therefore, no
    x is in $\text{Int}(A)$ and $\text{Bdy}(A)$. Similarly, $\text{Bdy}(A)\cap\text{Ext}(A)=\emptyset$
    since for every $x\in$Bdy$(A)$, no neighborhood of $x, N_x$, is a subset of $X\setminus A$, and for
    every $x\in$Ext$(A)$, there exists a neighborhood of $x, N_x$ such that
    $N_x\subseteq(X\setminus A)$. Therefore, no $x$ can be in $\text{Bdy}(A)$ and $\text{Ext}(A)$.
    Finally, let $x\in$Int$(A)$. Then there exists a neighborhood of $x, N_x$, such that
    $N_x\subseteq A$. Hence, $x\notin$Int$(X\setminus A)$. Similarly, if $x\in$Ext$(A)$, there exists a
    neighborhood of $x, N_x$, such that $N_x\subseteq(X\setminus A)$ and so $x\notin$Int$(A)$. Thus,
    Int$(A)\cap$Ext$(A)=\emptyset$. Therefore $\text{Int}(A), \text{Bdy}(A)$ and $\text{Ext}(A)$ are
    pairwise disjoint.

\end{solution}