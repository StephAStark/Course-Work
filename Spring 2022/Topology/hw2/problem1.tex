(\#6 in 2.5) Show that the countable union of countable sets is countable. (Hint: consider a table
similar to the one used to show that the set of rational numbers is countable).\\\\


\begin{solution}\renewcommand{\qedsymbol}{}\ \\
    Let $A_n$ be a countable set for all $n\in\mathbb{N}$. Then $\bigcup_{n\in\mathbb{N}}A_n$ is a
    countable union of countable sets. We will show that $\bigcup_{n\in\mathbb{N}}A_n$ is countable.
    Now, consider the following table displaying $\bigcup_{n\in\mathbb{N}}A_n$:

        \begin{center}
            \begin{tabular}{ c|   c   c   c   c   c   c   }
                $A_1$ & $a_{11}$ & $a_{12}$ & $a_{13}$ & $a_{14}$ & $a_{15}$ & $\dots$ \\ 
                $A_2$ & $a_{21}$ & $a_{22}$ & $a_{23}$ & $a_{24}$ & $a_{25}$ & $\dots$ \\  
                $A_3$ & $a_{31}$ & $a_{32}$ & $a_{33}$ & $a_{34}$ & $a_{35}$ & $\dots$ \\
                $A_4$ & $a_{41}$ & $a_{42}$ & $a_{43}$ & $a_{44}$ & $a_{45}$ & $\dots$ \\
                $A_5$ & $a_{51}$ & $a_{52}$ & $a_{53}$ & $a_{54}$ & $a_{55}$ & $\dots$ \\
                $A_6$ & $a_{61}$ & $a_{62}$ & $a_{63}$ & $a_{64}$ & $a_{65}$ & $\dots$ \\
                $\vdots$ & $\vdots$ & $\vdots$ & $\vdots$ & $\vdots$ & $\vdots$ & $\ddots$ \\
            \end{tabular}
        \end{center}

    We can write the elements of each $A_i$ for $i\in\mathbb{N}$ in this table since each $A_i$ is
    countable, and so we can define onto mappings $f_i:\mathbb{N}\rightarrow A_i$ by $f_i(j)=a_{ij}$.
    This is beacuase each $A_i$ is countable, so we can list their elements along the horizontal, and
    since the collection of all $A_i$ is also countable, we can list the $A_i$ along the vertical. Now,
    consider the function $f:\bigcup_{n\in\mathbb{N}}A_n\rightarrow\mathbb{N}$ defined by
    $f(a_{mn})=\frac12(m+n-2)(m+n-1)+m$. Now, this is the same function from the proof of the positive
    rationals being countable. Since each element is indexed, we see that each element of each set has a
    unique representation given by the element's index. Therefore, $f$ is a one to one function since
    the order of $m$ and $n$ matter to the function's output. That is, if $a_{m_1n_1}\neq a_{m_2n_2}$,
    then $f(a_{m_1n_1})\neq f(a_{m_2n_2})$. So we have that $f^{-1}$ exists. Now, let
    $y=a_{mn}\in\bigcup_{n\in\mathbb{N}}A_n$ and $x=\frac12(m+n-2)(m+n-1)+m$. Then $x\in\mathbb{N}$.
    Since $f^{-1}$ exists, we have that $f^{-1}(x)=f^{-1}(\frac12(m+n-2)(m+n-1)+m)=a_{mn}=y$. Since $y$
    was arbitrary, we have that $f^{-1}$ is an onto function from $\mathbb{N}$ to
    $\bigcup_{n\in\mathbb{N}}A_n$. Since there exists an onto function from $\mathbb{N}$ to
    $\bigcup_{n\in\mathbb{N}}A_n$, we have that $\bigcup_{n\in\mathbb{N}}A_n$ is countable as desired.        

\end{solution}