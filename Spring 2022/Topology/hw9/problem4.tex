(\#5 in 2.2) Prove \textbf{Theorem 2.2.5}: For $A\subseteq X$ and $f:X\to Y$ any function, we have
$A\subseteq f^{-1}(f(A))$. If, in addition, $f$ is one-to-one, then $A=f^{-1}(f(A))$.\\\\

\begin{solution}\renewcommand{\qedsymbol}{}\ \\
    Let $A\subseteq X$ and $f:X\to Y$ be any function. Assume $x\in A$. Then $f(x)=y$ for some $y\in Y$.
    So by definition of image, we have that $f(x)\in f(A)$. Then by definition of inverse image, we have
    that $x\in f^{-1}(f(A))$, and so $A\subseteq f^{-1}(f(A))$ as desired. Now, assume that $f$ is
    injective. Also, let $x\in f^{-1}(f(A))$. So, $f(x)\in f(A)$. Therefore, by definition, there is an
    $a\in A$ such that $f(x)=f(a)$. Now, since $f$ is one-to-one, we have that $x=a$. Hence $x\in A$.
    So, by the preiviuos proof, we have that $A=f^{-1}(f(A))$ as was to be done.

\end{solution}