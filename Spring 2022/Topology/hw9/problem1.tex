(\#2 in 6.5) Prove the general form of Theorem 6.5.3: If $A_{\gamma}$ is a connected subspace of
$X_{\tau}$ for every $\gamma\in \Lambda$ and if $\cap_{\gamma\in\Lambda} A_{\gamma}\neq \emptyset$, then
$\cup_{\gamma\in\Lambda}A_\gamma$ is connected as a subspace of $X$.\\\\

\begin{solution}\renewcommand{\qedsymbol}{}\ \\
    Let $X_{\tau}$ be a topological space and let $\{A_{\gamma}\}_{\gamma\in\Lambda}$ be a collection of
    connected subspaces of $X_{\tau}$. Let $\cap_{\gamma\in\Lambda}A_{\gamma}\neq\emptyset$. Assume that
    $\cup_{\gamma\in\Lambda} A_{\gamma}$ is disconnected. Then
    $\cup_{\gamma\in\Lambda}A_{\gamma}=U\cup V$ where $U$ and $V$ are disjoint, nonempty
    $\tau_{\cup_{\gamma\in\Lambda} A_{\gamma}}-$open sets. Take
    $A_{\gamma}\in\cup_{\gamma\in\Lambda} A_{\gamma}$, and consider $U\cap A_{\gamma}$ and
    $V\cap A_{\gamma}$. Since $U$ and $V$ are disjoint, we have that
    $(U\cap A_{\gamma})\cap(V\cap A_{\gamma})=(U\cap V)\cap A_{\gamma}=\emptyset$. Since $U$ and $V$ are
    open, $U\cap A_{\gamma}$ and $V\cap A_{\gamma}$ are open in the subspace topology by definition.
    Finally we can see that
    $(U\cap A_{\gamma})\cup(V\cap A_{\gamma})=(U\cup V)\cap A_{\gamma}=X\cap A_{\gamma}=A_{\gamma}$. So,
    if $U\cap A_{\gamma}, V\cap A_{\gamma}\neq\emptyset$, then $A_{\gamma}$ is disconnected as a
    subspace, which is a contradiction. So, without loss of generality, say
    $U\cap A_{\gamma}=\emptyset$. Since $\gamma\in\Lambda$ was arbitrary, we have that
    $U\cap A_{\gamma}=\emptyset$ or $V\cap A_{\gamma}=\emptyset$. So either $U\cap A_{\gamma}=\emptyset$
    for all $\gamma\in\lambda$, in which case $U=\emptyset$ which is a contradiction, or
    $U\cap A_{\gamma}=\emptyset$ for $\gamma\in\Gamma\subset\Lambda$ and $V\cap A_{\gamma}=\emptyset$
    for $\gamma\in\Lambda\setminus\Gamma$. For this latter case, we contradict
    $\cap_{\gamma\in\Lambda} A_{\gamma}\neq \emptyset$. Thus, $\cup_{\gamma\in\Lambda} A_{\gamma}$ is
    connected as desired.

\end{solution}