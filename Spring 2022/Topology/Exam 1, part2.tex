\documentclass[12pt]{article}
\usepackage[margin=1in]{geometry} 
\usepackage{amsmath}
\usepackage{amssymb}
\usepackage{amsthm}
\usepackage{accents}
\usepackage{graphicx}

\setlength{\oddsidemargin}{0in}
\setlength{\textwidth}{6.5in}
\setlength{\topmargin}{-.55in}
\setlength{\textheight}{9in}
\pagestyle{empty}

\begin{document}
\noindent Math 5510

\noindent Topology

\noindent Stephanie Klumpe

\vspace{.2in}
\begin{center}
Exam One, Part Two
\end{center}

\noindent \textbf{Instructions:} You may use our class text and class notes on this exam, but you may not use any other print or electronic resources. Also, please do not discuss the problems with anyone but me. Please submit your completed exam by Friday, April 1 at 5:00 pm. Good luck!

\begin{enumerate}
\item Let $X = \{1,2,3,4,5\}$ with topology $\tau=\{\emptyset, X, \{1\}, \{3, 4\}, \{1, 3, 4\}\}$, and let $Y = \{A,B\}$ with topology $\sigma=\{\emptyset, Y, \{A\}\}$. 
\begin{enumerate}
\item How many functions are there from $X$ to $Y$? Justify your answer.\\
There are 32 functions from $X$ to $Y$. To start, we can take every element of $X$ and map it to $A\in Y$, $f(i)=A$ for $i\in X$. Similarly, we can map every element to $B\in Y$, $f(i)=b$ for $i\in X$. Next we can take just one element of $X$ and map it to $A$ and map the rest to $B$, like $f:X\rightarrow Y$ defined by $f(1)=A, f(2)=B, f(3)=B, f(4)=B, f(5)=B$. This is just the combination 5 choose 1. Similarly, we can take one element of $X$ and map it to $B$ and then map the rest to $A$, such as $f:X\rightarrow Y$ defined by $f(1)=B, f(2)=A, f(3)=A, f(4)=A, f(5)=A$. Again, this is just the combination 5 choose 1. So both have 5 combinations. Next, we can take two elements of $X$ and map them to $A$ and map the rest to $B$ like $f:X\rightarrow Y$ defined by $f(1)=A, f(2)=A, f(3)=B, f(4)=B, f(5)=B$. On the other hand, we can map two elements to $B$ and the other three to $A$ such as $f:X\rightarrow Y$ defined by $f(1)=B, f(2)=B, f(3)=A, f(4)=A, f(5)=A$. Both of these are 5 choose 2 combinations, or ten each. So, in total, we have $2+10+20=32$ functions. Any others would be repeats of these combinations.
\item List all the continuous functions from $X$ to $Y$. Show that each of the functions you list is in fact continuous.\\
There are four continuous functions from $X_{\tau}$ to $Y_{\sigma}$. First is $f:X_{\tau}\rightarrow Y_{\sigma}$ defined by $f(1)=f(2)=f(3)=f(4)=f(5)=A$. Then we see that $f^{-1}(\emptyset)=\emptyset, f^{-1}(A)=X$, and $f^{-1}(Y)=\emptyset$. Since all of the preimages are open sets in $X_{\tau}$, $f$ is continuous. Next is $f:X_{\tau}\rightarrow Y_{\sigma}$ defined by $f(1)=A$, and $f(2)=f(3)=f(4)=f(5)=B$. Then, $f^{-1}(\emptyset)=\emptyset, f^{-1}(A)=\{1\},$ and $f^{-1}(Y)=X$. Again, all preimages are $\tau$-open, so $f$ is continuous. Third is $f:X_{\tau}\rightarrow Y_{\sigma}$ given by $f(1)=f(2)=f(5)=B$, and $f(3)=f(4)=A$. So, we get that $f^{-1}(\emptyset)=\emptyset, f^{-1}(A)=\{3, 4\},$ and $f^{-1}(Y)=X$. Still, all preimages are open in $X$, which means $f$ is continuous. Finally we have $f:X_{\tau}\rightarrow Y_{\sigma}$ defined as $f(1)=f(3)=f(4)=A$, and $f(2)=f(5)=B$. Hence $f^{-1}(\emptyset)=\emptyset, f^{-1}(A)=\{1, 3, 4\},$ and $f^{-1}(Y)=X$. We still see that all preimages are open in $X$, so $f$ is continuous. Therefore all four functions are continuous.
\item Are there any continuous functions from $Y$ to $X$? Explain.\\
Yes, there is at least one continuous function from $Y$ to $X$. Consider $f:Y_{\sigma}\rightarrow X_{\tau}$ defined by $f(A)=2$ and $f(B)=5$. Then, we see that $f$ is well defined, and that $f^{-1}(\emptyset)=f^{-1}(\{1\})=f^{-1}(\{3, 4\})=f^{-1}(\{1,3,4\})=f^{-1}(X)=\emptyset$ which is open in $Y$. So, we see that there exists at least one continuous function from $Y$ to $X$.
\end{enumerate}

\item Show that the interval $(0,\infty)$ is homeomorphic to $\mathbb{R}$.\\

Consider the function $f:(0,\infty)\rightarrow\mathbb{R}$ defined by $f(x)=\ln(x)$. First we will show that $f$ is bijective. So, let $x_1,x_2\in(0,\infty)$ and assume that $f(x_1)=f(x_2)$. Then $\ln x_1=\ln x_2$. Hence $e^{\ln x_1}=e^{\ln x_2}$. Therefore $x_1=x_2$ and $f$ is injective. Now, let $y\in\mathbb{R}$ and let $x=e^y$. Since $e^y>0$ for all $y\in\mathbb{R}$, $x\in(0,\infty)$. So, $f(x)=\ln x=\ln e^y=y$. Thus, $f$ is onto and so $f$ is bijective. Clearly, we have that $f^{-1}(x)=e^x$. Now, let $U$ be an open set in $\mathbb{R}$. So, $U$ is the union of open intervals in $\mathbb{R}$. Then $f^{-1}(U)=e^(U)$ is also the union of open intervals in $(0,\infty)$ since $e^x$ is an increasing function. So, $f$ is continuous. Now, let $U$ be an open set in $(0,\infty)$. So, $U$ is the union of open intervals. Then, $f(U)=\ln U$ is the union of open intervals in $\mathbb{R}$ since $\ln x$ is also an increasing function. So, $f(U)$ is open and hence $f$ is an open map. Therefore, we have that $f$ is a homeomorphism between $(0,\infty)$ and $\mathbb{R}$. Thus, $(0,\infty)$ is homeomorphic to $\mathbb{R}$ as desired.\\

\item Let $X_{\tau}$ be a topological space with $\mathcal{B}$ a basis for $\tau$. If $Y\subseteq X$, define $\mathcal{B}_Y = \{B\cap Y|B\in \mathcal{B}\}$.

Prove that $\mathcal{B}_Y$ is a basis for the subspace topology on $Y$ by first proving that $\mathcal{B}_Y$ satisfies the two conditions for a basis of \textit{some} topology on $Y$, then showing that the topology generated by $\mathcal{B}_Y$ is the subspace topology on $Y$.\\

Let $X_{\tau}, \mathcal{B}, Y,$ and $\mathcal{B}_Y$ be as given above. Since $\mathcal{B}$ is a basis for $\tau$, we know that $\bigcup\mathcal{B}=X$. That is, $\mathcal{B}$ covers $X$. Now since $Y\subseteq X$, we have that $\bigcup\mathcal{B}_Y=Y$ and so $\mathcal{B}_Y$ covers $Y$. Now, take $B_1,B_2\in\mathcal{B}_Y$. Then $B_1=Y\cap B_{X_1}$ and $B_2=Y\cap B_{X_2}$ for some $B_{X_1}, B_{X_2}\in\mathcal{B}$. Then $B_1\cap B_2=(Y\cap B_{X_1})\cap(Y\cap B_{X_2})=Y\cap(B_{X_1}\cap B_{X_2})$. Since $B_{X_1},B_{X_2}\in\mathcal{B}$, there exists $B_{X_3}\in\mathcal{B}$ such that $B_{X_3}\subseteq B_{X_1}\cap B_{X_2}$. Now, take $B_3=Y\cap B_{X_3}$. Well, we have that $B_3Y\cap B_{X_3}\subseteq Y\cap(B_{X_1}\cap B_{X_2})=B_1\cap B_2$. Hence, there exists $B_3\in\mathcal{B}_Y$ such that $B_3\subseteq B_1\cap B_2$. So, $\mathcal{B}_Y$ is the basis for some topology on $Y$. So, the topology generated by $\mathcal{B}_Y$ is $\tau_{\mathcal{B}_Y}=\{U\subseteq Y|$ if $x\in U$ then there exists a $B\in\mathcal{B}_Y$ such that $x\in B\subseteq U\}$. Also, $\tau_Y=\{U\subseteq Y|U=Y\cap V$such that $V$ is $\tau$-open$\}$. Now, let $U\in\tau_{\mathcal{B}_Y}$, and let $x\in U$. Then there exists $B\in\mathcal{B}_Y$ such that $x\in B\subseteq U$. Since $B\in\mathcal{B}_Y$, $B=B_X\cap Y$ for some $B_X\in\mathcal{B}$. Since $B_X$ is $\tau-$open by definition, we have that $B\in\tau_Y$. Since $x$ and $U$ were arbitrary, $\tau_{\mathcal{B}_Y}\subseteq\tau_Y$. Now, let $U\in\tau_Y$. Then $U=Y\cap V$ where $V$ is $\tau-$open. Since $V$ is $\tau-$open, there exists $B_X\in\mathcal{B}$ such that $B_X\subseteq V$. So, $Y\cap B_X\subseteq U=Y\cap V$. So, there exists $B=\in\mathcal{B}_Y$ such that $B\subseteq U$. Hence $U\in\tau_{\mathcal{B}_Y}$. So, $\tau_{\mathcal{B}_Y}=\tau_Y$ as desired.

\item Let $X_{\tau}$ and $Y_{\sigma}$ be topological spaces and suppose that $X=A\cup B$. Let $f: X_{\tau}\to Y_{\sigma}$ be a function such that $f|_A$ and $f|_B$ are both continuous (where $A$ and $B$ are considered as subspaces of $X$).

Prove that if $A$ and $B$ are both open or both closed, then $f$ is continuous.\\

Let $X_{\tau}$ and $Y_{\sigma}$ be as given. Let $A$ and $B$ be subspaces of $X$ such that $X=A\cup B$ and let $f: X_{\tau}\to Y_{\sigma}$ be as give. Assume that $A$ and $B$ are open. Let $U\subseteq Y$ be $\sigma-$open. First consider $f^{-1}(U)\subseteq A$. Then $f^{-1}(U)=f^{-1}|_A(U)$. Since $f|_A$ is continuous, $f^{-1}|_A(U)$ is $\tau_A-$open. So, $f^{-1}|_A(U)=A\cap V$ for some $\tau-$open $V$. We assume that $A$ is open, so we have that $f^{-1}|_A(U)$ is $\tau-$open. Next consider $f^{-1}(U)\subseteq B$. So, $f^{-1}(U)=f^{-1}|_B(U)$. Since $f|_B$ is continuous, $f^{-1}|_B(U)$ is $\tau_B-$open. Hence $f^{-1}|_B(U)=B\cap V$ for some $\tau-$open $V$. Since $B$ is open, we have that $f^{-1}|_B(U)$ is open in $X$ as well. Next consider $f^{-1}(U)\subseteq A\cap B$. If $A\cap B\neq\emptyset$, then we can use either of the two previous cases. If $A\cap B=\emptyset$, then we have that $f^{-1}(U)=\emptyset$ which is open in $X_\tau$. Finally, assume that $f^{-1}(U)\subseteq A\cup B$. Then $f^{-1}(U)=f^{-1}|_A(U)\cup f^{-1}|_B(U)=(A\cap V)\cup(B\cap W)$ for some $\tau-$open sets $V, W$ since $f|_A$ and $f|_B$ are continuous. Hence, we have that $f^{-1}(U)$ is open in $X$ since $A$ and $B$ are both open. Thus, $f$ is continuous.\\\\
Now assume that both $A$ and $B$ are closed. Let $U\subset Y$ be $\sigma-$closed. Now consider $f^{-1}(U)\subseteq A$. Then again, we have that $f^{-1}(U)=f^{-1}|_A(U)$. Since $f|_A$ is continuous, $f^{-1}|_A(U)$ is $\tau_A-$closed. So, $f^{-1}|_A(U)=A\cap K$ where $K$ is some $\tau-$closed subset of $X$. Since $A$ is also closed, we have that $f^{-1}(U)$ is also closed. Next consider $f^{-1}(U)\subseteq B$. So, we see that $f^{-1}(U)=f^{-1}|_B(U)$. Since $f|_B$ is continuous, $f^{-1}|_B(U)$ is $\tau_B-$closed. Hence $f^{-1}|_B(U)=B\cap K$ for some $\tau-$closed $K\subseteq X$. Again, we have that $f^{-1}(U)$ is alos closed since $B$ is also closed by assumption. Now consider $f^{-1}(U)\subseteq A\cap B$. If $A\cap B\neq\emptyset$, then we can use either of the two previous cases, to get that $f^{-1}(U)$ is closed in $X$. If $A\cap B=\emptyset$, then we have that $f^{-1}(U)=\emptyset$ which is closed in $X_\tau$. Finally, consider the case where $f^{-1}(U)\subseteq A\cup B$. Then $f^{-1}(U)=f^{-1}|_A(U)\cup f^{-1}|_B(U)$. Then we have that $f^{-1}(U)=(A\cap K)\cup (B\cap V)$ for some $\tau-$closed sets $K, V$ since $f|_A$ and $f|_B$ are continuous. Thus, $f^{-1}(U)$ is closed since $A$ and $B$ are closed. Therefore, we have that $f$ is again continuous. So, if $A$ and $B$ are both open or both closed, then $f$ is continuous as desired.

\item Recall that a topological space is called second-countable if its topology has a countable basis.
\begin{enumerate}
\item Let $X$ be a countably infinite set, and let $\mathcal{FC}$ be the finite complement topology. Prove that $X_{\mathcal{FC}}$ is second-countable.

Hint: The collection $\mathcal{FC}$ is a basis for itself.\\

Before we begin, we will show that every subset of a countable set is countable. So, let $A$ be a countable set and let $B\subseteq A$. If $A$ is finite, then $B$ is finite, so $B$ is then countable. So assume that $A$ is countably infinite, and that $B$ is also infinite since $B$ would be countable otherwise. Since $A$ is countable, there exists $f:\mathbb{N}\rightarrow A$ such that $f$ is onto. So the elemtents of $A$ are indexed by $\mathbb{N}$. Now, let $y\in B$. Since $a\in B$, $y=x_i$ for some $i\in\mathbb{N}$. Now, consider $b\in B\setminus\{a\}$. Then $b=x_j$ for some $j\in\mathbb{N}$ such that $i\neq j$. Continuing in this manner, we see that every element of $B$ is some indexed element of $A$. So, by ordering the elements of $B$ using the idicies, and then shifting the indicies accordingly, we have that there exists an onto function $f:\mathbb{N}\rightarrow B$. Hence, $B$ is countable by definition. Since $B\subseteq A$ was arbitrary, every subset of a countable set is countable.\\
Let $X$ be as given. Now, since every topology is a basis for itself, consider $\mathcal{FC}$. The sets of $\mathcal{FC}$ are $V=X\setminus\{x_{j_1},\ldots, x_{j_n}|x_i\in X$ for $1\leq i\leq n,j\in\mathbb{N}\}$ for $n\in\mathbb{N}$ such that $n\geq1$. Clearly, $V\subseteq X$ for each $n\in\mathbb{N}$, so $V$ is countable. So, $\{V\}_i=\bigcup_{j\in\mathbb{N}}\{X\setminus\{x_{j_1},\ldots, x_{j_n}|x_i\in X$ for $1\leq i\leq n, j\in\mathbb{N}\}\}$ is countable for each $n\in\mathbb{N}$ since $\{V\}_i$ is a countable union of subsets of $X$. Hence, $\bigcup_{i=1}^n\{V\}_i=\mathcal{FC}$ is the countable union of countable sets. Therefore $\mathcal{FC}$ is countable. Thus, we have that $X_{\mathcal{FC}}$ is second countable as desired.\\

\item Let $X$ be an uncountable set. Prove that $X_{\mathcal{FC}}$ is not second countable.

Hint: Show that no countable collection of subsets of $X$ can form a basis for $X$.\\

Let $X$ be as given. Assume by way of contradiction that there exists a countable basis for $\mathcal{FC}$, call it $B$. That is, there is a countable collection of subsets of $X$ that forms a basis. If all $b\in B$ are themselves countable, then $\bigcup B=X$ is countable, which means $X$ is countable, which is a contradiction. So, assume not all $b\in B$ are countable. Since $B$ is a basis, we have that all $V\in\mathcal{FC}$ are unions of basis sets. So, consider $Y=\bigcup_{V\in\mathcal{FC}}(X\setminus V)$. Then $Y=\bigcup_{V\in\mathcal{FC}}(X\setminus V)=\bigcup_{b\in B}(X\setminus(\cup b))$. Now, since $B$ is countable, $Y$ is the countable union of finite sets. Hence, $Y$ is countable. So, let $x\in X\setminus Y$ and take $V=X\setminus\{x\}$. Then we see that $V\in\mathcal{FC}$ since $X\setminus V=\{x\}$ which is clearly finite. However, we see that $x\notin\bigcup_{b\in B}b$, so $x$ is not in any union of basis sets. Therefore, $B$ is not a basis since $\bigcup B\neq X$. Thus, $X_{\mathcal{FC}}$ is not second countable.
\end{enumerate}
\end{enumerate}

\end{document}
