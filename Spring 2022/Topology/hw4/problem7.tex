Let $A$ be the subspace $(-\infty,-1)\cup [0,\infty)$ of $\mathbb{R}$ with the usual topology, and
consider the map $f:A\to \mathbb{R}$ defined as follows:
\[f(x)=\left\{ \begin{array}{ll}
                  x+1 & \mbox{$x<-1$}\\
                  x      & \mbox{$x\geq 0$}
                  \end{array}
          \right. \]
Show that $f$ is bijective and continuous. Is $f$ a homeomorphism?\\\\

\begin{solution}\renewcommand{\qedsymbol}{}\ \\
    Let $A$ and $f$ be as given above. Now, let $x_1,x_2\in A$ and assume that $f(x_1)=f(x_2)$. Without
    loss of generality, first let $x_1<-1$ and $x_2\geq0$. Then $f(x_1)=x_1+1<0$ and $f(x_2)=x_2\geq0$
    contradicting $f(x_1)=f(x_2)$. So, either $x_1+1=x_2+1$ which yields $x_1=x_2$, or $x_1=x_2$. Hence,
    we have that $f$ is injective. Now, let $y\in\mathbb{R}$. Assume first that $y\geq0$. Take $x=y$.
    Then $f(x)=x=y$. Now assume that $y<0$. So, take $x=y-1$. Then $x<-1$ and  $f(x)=x+1=(y-1)+1=y$, and
    therefore $f$ is onto as well. Thus, $f$ is bijective.\\Now, let $V$ be an open set in
    $\mathbb{R}_{\mathcal{U}}$. Assume that $V\subseteq(0,\infty)$. Then $f^{-1}(V)=A\cap V=V$ which is
    $\tau_{A}-$open by definition. Now assume that $V\subseteq(-\infty,-1)$. Then $f^{-1}(V)=A\cap V=V$
    and is thus again $\tau_{A}-$open. Finally assume that $V=(a,b)$ where $a<-1<0<b$. Then
    $f^{-1}(V)=(a,-1)\cup[0,b)=A\cap V$. So, $f^{-1}(V)$ is open in the subspace $A$. Therefore $f$ is
    continuous.\\
    Now, $f$ is not a homeomorphism. We can see that $f^{-1}:\mathbb{R}\rightarrow A$ is defined by 
    \[f^{-1}(x)=\left\{ \begin{array}{ll}
        x-1 & \mbox{$x<0$}\\
        x      & \mbox{$x\geq 0$}
        \end{array}
                \right. \]
    Consider the interval $W=[0,1)\in A$ which is open in $A$ since we can write $W$ as
    $W=A\cap(-0.5,1)$ where $(-0.5,1)$ is open in $\mathbb{R}_{\mathcal{U}}$. Then $f(W)=[0,1)$, which
    is not open in $\mathbb{R}_{\mathcal{U}}$. Thus $f^{-1}$ is not continuous and so $f$ is not a
    homeomorphism as claimed.

\end{solution}