On a previous homework, you listed all the possible topologies on a set with three elements.
However, some of the resulting topological spaces are homeomorphic. Which are homeomorphic? Divide the
set of 29 topological spaces into homeomorphism classes, and be sure to justify your choices. There are
9 homeomorphism classes in total.\\
To justify your choices, explain why the spaces within each class are homeomorphic to each other. Give
the functions explicitly and explain why they are homeomorphisms.
(You can be somewhat loose with that explanation).\\
Also explain why spaces in two different classes are not homeomorphic.\\\\

\begin{solution}\renewcommand{\qedsymbol}{}\ \\
    Let $X$ be a set with three elelments given by $\{a,b,c\}$. First, we will label all 29 topologies.
    Let
    
    $$\tau_1=\mathcal{P}(X), \tau_2=\{X,\emptyset\}, \tau_3=\{\emptyset, \{a\}, X\},
    \tau_4=\{\emptyset, \{b\}, X\}, \tau_5=\{\emptyset, \{c\}, X\}, \tau_6=\{\emptyset, \{a,b\}, X\}$$
    $$\tau_7=\{\emptyset, \{a,c\}, X\}, \tau_8=\{\emptyset, \{b,c\}, X\},
    \tau_9=\{\emptyset, \{a\}, \{a,b\}, X\}, \tau_{10}=\{\emptyset, \{a\}, \{a,c\}, X\}$$
    $$\tau_{11}=\{\emptyset, \{b\}, \{a,b\}, X\}, \tau_{12}=\{\emptyset, \{b\}, \{b,c\}, X\},
    \tau_{13}=\{\emptyset, \{c\}, \{a,c\}, X\}, \tau_{14}=\{\emptyset, \{c\}, \{b,c\}, X\}$$
    $$\tau_{15}=\{\emptyset, \{a\}, \{b,c\}, X\}, \tau_{16}=\{\emptyset, \{b\}, \{a,c\}, X\},
    \tau_{17}=\{\emptyset, \{c\}, \{a,b\}, X\}$$
    $$\tau_{18}=\{\emptyset, \{a\}, \{b\}, \{a,b\}, X\},
    \tau_{19}=\{\emptyset, \{a\}, \{c\}, \{a,c\}, X\},
    \tau_{20}=\{\emptyset, \{b\}, \{c\}, \{b,c\}, X\}$$
    $$\tau_{21}=\{\emptyset, \{a\}, \{a,b\}, \{a,c\}, X\},
    \tau_{22}\{\emptyset, \{b\}, \{a,b\}, \{b,c\}, X\},
    \tau_{23}=\{\emptyset, \{c\}, \{a,c\}, \{b,c\}, X\}$$
    $$\tau_{24}=\{\emptyset, \{a\}, \{b\}, \{a,b\}, \{a,c\}, X\},
    \tau_{25}=\{\emptyset, \{a\}, \{b\}, \{a,b\}, \{b,c\}, X\}$$
    $$\tau_{26}=\{\emptyset, \{a\}, \{c\}, \{a,b\}, \{a,c\}, X\},
    \tau_{27}=\{\emptyset, \{a\}, \{c\}, \{a,c\}, \{b,c\}, X\}$$
    $$\tau_{28}=\{\emptyset, \{b\}, \{c\}, \{a,b\}, \{b,c\}, X\},
    \tau_{29}=\{\emptyset, \{b\}, \{c\}, \{a,c\}, \{b,c\}, X\}$$

    To start, $X_{\tau_1}$ is homeomorphic only to itself under the identitiy map. Since there is no
    other topology on $X$ with 8 elements, there is no other topological space homeomorphic to
    $X_{\tau_1}$. Since the identity map is mapping $X_{\tau_1}$ to itself, it is a homeomorphism since
    it is clearly bijective and is continuous and open by a theorem since $\tau_1$ and $\tau_1$ are the
    same size. Also, $X_{\tau_2}$ is homeomorphic only to itself by the identity map as well by a
    similar argument. Now, $X_{\tau_3}, X_{\tau_4},$ and $X_{\tau_5}$ are all homeomorphic. Consider the
    function $f:X_{\tau_3}\rightarrow X_{\tau_4}$ defined by $f(a)=b, f(b)=a,$ and $f(c)=c$. This is
    clearly bijective since all elements are mapped to and no element is mapped to more than one. Also,
    we can see that $\{a\}\mapsto\{b\}, \{b\}\mapsto\{a\}, \emptyset\mapsto\emptyset$, and by definition
    of the function, $X\mapsto X$. So, we have that $f$ is continuous and open since open sets are
    mapped to open sets by both $f$ and $f^{-1}$. Hence $f$ is a homeomorphism. By changing the function
    based on the elements in the domain and the codomain, we can clearly see that all of them are
    homeomorphic. Now, the topological spaces $X_{\tau_6}, X_{\tau_7}$, and $X_{\tau_8}$ are also
    homeomorphic. Take the function $f:X_{\tau_6}\rightarrow X_{\tau_7}$ definied by $f(a)=a, f(b)=c$,
    and $f(c)=b$. Then, we see that by similar logic as above, $f$ is bijective. We can now see that
    $\{a,b\}\mapsto\{a,c\}, \{a,c\}\mapsto\{a,b\}, \emptyset\mapsto\emptyset$, and again by the
    definition of our function, $X\mapsto X$. So, $f$ is a homeomorphism, and if we change the domain
    and codomain, we need only adjust the function slightly to preserve the homeomorphic property. Next
    we have $X_{\tau_9}, X_{\tau_{10}}, X_{\tau_{11}}, X_{\tau_{12}}, X_{\tau_{13}}, X_{\tau_{14}}$ all
    being homeomorphic. Take the function $f:X_{\tau_9}\rightarrow X_{\tau_{10}}$ defined by
    $f(a)=a, f(b)=c,$ and $f(c)=b$. By similar logic as the last function we defined above, $f$ is a
    homeomorphism since
    $\{a\}\mapsto\{a\}, \{a,b\}\mapsto\{a,c\}, \{a,c\}\mapsto\{a,b\}, \emptyset\mapsto\emptyset$, and
    $X\mapsto X$. Since we can change the function only slightly to match the different domains and
    codomians, we have that they are all homeomorphic. Now, the spaces $X_{\tau_{15}}, X_{\tau_{16}}$,
    and $X_{\tau_{17}}$ are homeomorphic as well. Let's consider the function
    $f:X_{\tau_{15}}\rightarrow X_{\tau_{16}}$ defined by $f(a)=b, f(b)=a,$ and $f(c)=c$. By repeating
    the same logic as the first function we defined, we have that $f$ is again a homeomorphism since
    $\{a\}\mapsto\{b\}, \{b\}\mapsto\{a\}, \{b,c\}\mapsto\{a,c\}, \{a,c\}\mapsto\{b,c\},
    \emptyset\mapsto\emptyset$, and $X\mapsto X$. So, again by editing the function to match the domain
    and the codomain, we have that they are all homeomorphic. We also have that
    $X_{\tau_{18}}, X_{\tau_{19}}$, and $X_{\tau_{20}}$ are all homeomorphic topological spaces. Take
    the function $f:X_{\tau_{18}}\rightarrow X_{\tau_{19}}$ defined by $f(a)=a, f(b)=c$, and $f(c)=b$.
    This is similar to one of the functions that we defined above, so we already have that $f$ is a
    bijection. Now,
    $\{a\}\mapsto\{a\}, \{b\}\mapsto\{c\}, \{c\}\mapsto\{b\}, \{a,b\}\mapsto\{a,c\},
    \{a,c\}\mapsto\{a,b\}, \emptyset\mapsto\emptyset$, and $X\mapsto X$. So, we have that $f$ is open
    and continuous since open sets are mapped to open sets, so $f$ is a homeomorphism. Then by changing
    the function depending on the element of the spaces, we have that they are all homeomorphic. Next,
    the spaces $X_{\tau_{21}}, X_{\tau_{22}}, X_{\tau_{23}}$ are homeomorphic too. Consider the function
    $f:X_{\tau_{21}}\rightarrow X_{\tau_{22}}$ defined by $f(a)=b, f(b)=c$, and  $f(c)=c,$. This
    function is similar to the other function that we defined earlier, and so we have that it is a
    bijection and we see that it is open and continuous since
    $\{a\}\mapsto\{b\}, \{b\}\mapsto\{a\}, \{a,b\}\mapsto\{a,b\}, \{a,c\}\mapsto\{b,c\},
    \{a,b\}\mapsto\{b,c\}, \emptyset\mapsto\emptyset$, and $X\mapsto X$. Finally, we have that
    $X_{\tau_{24}}, X_{\tau_{25}}, X_{\tau_{26}}, X_{\tau_{27}}, X_{\tau_{28}}$, and $X_{\tau_{29}}$ are
    all homeomorphic. To see this, take the function $f:X_{\tau_{24}}\rightarrow X_{\tau_{26}}$ defined
    by $f(a)=a, f(b)=c,$ and $f(c)=b$. We again have already defined this function for smaller
    topological spaces, but the argument is the same, so we have that $f$ is still a homeomorphism since
    $\{a\}\mapsto\{a\}, \{b\}\mapsto\{c\}, \{c\}\mapsto\{b\}, \{a,b\}\mapsto\{a,c\},
    \{a,c\}\mapsto\{a,b\}, \emptyset\mapsto\emptyset$, and $X\mapsto X$.\\

    Now clealy, no two topological spaces that are of different sizes canbe homeomorphic as there would
    be no bijective continuous map between the two. Now, for any two spaces with the same number of
    elements across two classes, we see that there would be no well defined continuous mapping between
    any two. For example, $X_{\tau_3}$ cannot be homeomorphic to $X_{\tau_6}$ since there is no way for
    $\{a\}$ to map to $\{a,b\}$ and vice versa.

\end{solution}