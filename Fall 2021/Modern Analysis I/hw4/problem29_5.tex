Let $f$ be defined on $\mathbb{R}$, and suppose $|f(x)-f(y)|\leq(x-y)^2$ for all $x,y\in\mathbb{R}$.
Prove $f$ is a constant function.\\\\

\begin{solution}\renewcommand{\qedsymbol}{}\ \\
    Let $f$ be as given above and let $x,y\in\mathbb{R}$. Then we can rewrite this as
    $|f(x)-f(y)|\leq|x-y||x-y|$. Hence, $|\frac{f(x)-f(y)}{x-y}|\leq|x-y|$. Well,
    
    $$|f'(y)|=\lim_{x\rightarrow y}|\frac{f(x)-f(y)}{x-y}|\leq\lim_{x\rightarrow y}|x-y|=|y-y|=0$$
    
    Thus, $f'(y)=0$ and since $y$ was arbitrary, we have that $f'(x)=0$ for all $x\in\mathbb{R}$. So, by
    corollary 29.4 we have that $f(x)$ is a constant function on $\mathbb{R}$.

\end{solution}