a. Show $\sum_{n=0}^{\infty}(-1)^nx^{2n}=\frac{1}{1+x^2}$ for $x\in(-1,1)$.\\

b. Show $arctan(x)=\sum_{n=0}^{\infty}\frac{(-1)^n}{2n+1}x^{2n+1}$ for $x\in(-1,1)$.\\

c. Show the equality in b also holds for $x=1$. Use this to find a nice formula for $\pi$.\\

d.What happens at $x=-1$?\\

\begin{solution}\renewcommand{\qedsymbol}{}\ \\
    Let $x\in(-1,1)$. Set $y=-x^2$. So, $-y=x^2$. Hence,
    
    $$\sum_{n=0}^{\infty}(-1)^nx^{2n}=\sum_{n=0}^{\infty}(-1)^n(-y)^{n}$$
    $$=\sum_{n=0}^{\infty}(y)^n=\frac{1}{1-y}=\frac{1}{1+x^2}$$
    
    as desired.\\

    Let $x\in(-1,1)$. Well,
    
    $$arctan(x)=\int_{0}^{x}\frac{1}{1+t^2}\,dt=\int_{0}^{x}\sum_{n=0}^{\infty}(-1)^nt^{2n}\,dt$$
    $$=\sum_{n=0}^{\infty}\frac{(-1)^n}{2n+1}x^{2n+1}$$

    Set $x=1$. So, the series becomes $\sum_{n=0}^{\infty}\frac{(-1)^n}{2n+1}$. This converges by the
    alternating series test, so we get that the equality holds at x=1 by Abel's Theorem. Well,
    $arctan(1)=\frac{\pi}{4}=\sum_{n=0}^{\infty}\frac{(-1)^n}{2n+1}$. Thus, we get that
    $\pi=4\times\sum_{n=0}^{\infty}\frac{(-1)^n}{2n+1}$.\\

    Define $x=-1$. Then the series becomes
    
    $$\sum_{n=0}^{\infty}\frac{(-1)^n}{2n+1}(-1)^{2n+1}=\sum_{n=0}^{\infty}\frac{(-1)^n}{2n+1}(-1)$$
    
    since $2n+1$ is odd for all $n$ and $(-1)^m=-1$ if $m$ is odd. So, we get
    
    $$(-1)\sum_{n=0}^{\infty}\frac{(-1)^n}{2n+1}=-\frac{\pi}{4}=acrtan(-1)$$
    
    Therefore the equality holds for $x=-1$ as well.\\

\end{solution}