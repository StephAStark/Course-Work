a. If $f$ and $g$ are continuous functions on $[a,b]$ and $g(t)\geq0$ for all $t\in[a,b]$, prove there
exists $x\in(a,b)$ such that $\int_a^bf(t)g(t)dt=f(x)\int_a^bg(t)dt$.\\

b. Show Theorem 33.9 is a special case of part a.\\

c. Does the conclusion of part a hold if $[a,b]=[-1,1]$ and $f(t)=g(t)=t$ for all $t$?\\

\begin{solution}\renewcommand{\qedsymbol}{}\ \\
    Well, if $g(t)=0$, then we see that $0=\int_a^bf(t)*0dt=f(x)\int_a^b0dt$ for all $x\in(a,b)$. So,
    assume now that $g(t)>0$ for all $t\in[a,b]$. Now, since $f$ is continuous, there exists a maximum
    $M$ and a minimum $m$ for whcih there is some $x_0,y_0\in[a,b]$ such that $f(x_0)=m$ and $f(y_0)=M$.
    So,
    
    $$\int_a^bf(x_0)g(t)dt\leq\int_a^bf(t)g(t)dt\leq\int_a^bf(y_0)g(t)dt$$
    
    and hence
    
    $$f(x_0)\int_a^bg(t)dt\leq\int_a^bf(t)g(t)dt\leq f(y_0)\int_a^bg(t)dt$$
    
    So, by the Intermediate Value Theorem, there exists $x\in(a,b)$ such that
    
    $$f(x)\int_a^bg(t)dt=\int_a^bf(t)g(t)dt$$
    
    as desired.\\

    Well, if we let $g(t)=1$, then by part a, we see there exists $x\in[a,b]$ such that
    
    $$\int_a^bf(t)*1dt=f(x)\int_a^b1dt=f(x)(b-a)$$
    
    Hence there exists $x\in[a,b]$ such that $f(x)=\frac{1}{b-a}\int_a^bf$ as desired.\\

    No, the conclusion doesn't hold. We would get $\int_{-1}^1t^2dt=x\int_{-1}^1tdt$ which yields
    $\frac23=x*0=0$ which is not true. The conclusion doesn't hold beacsue the hypothesis is not met
    since $g(t)<0$ on $[-1,0)\subset[-1,1]$.

\end{solution}