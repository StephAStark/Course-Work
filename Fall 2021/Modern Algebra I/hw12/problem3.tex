a. Let $S$ be a nonempty set and suppose $\mathcal{P}$ is a partition of $S$. Define the relation $R$ on
$S$ by $R=\{(x,y)\in S\times S:x,y\in P$ for some $P\in\mathcal{P}\}$. Prove $R$ is an equivalence
relation.\\

b. Let $G$ be a group. Define $R$ on $G$ by $R=\{(x,y)\in G\times G:y=gxg^{-1}$ for some $g\in G\}$.
Prove $R$ is an eqivalence relation.\\\\

\begin{solution}\renewcommand{\qedsymbol}{}\ \\
    Let $S, \mathcal{P}$, and $R$ be as above. Now, let $s\in S$. Then, $s\in P$ for some
    $P\in\mathcal{P}$. Hence $(s,s)\in R$ and so $R$ is reflexive. Now, let $x,y\in S$ and $(x,y)\in R$.
    Then, $x,y\in P$ for some $P\in\mathcal{P}$. So, $y,x\in P$ and hence $(y,x)\in R$. Thus $R$ is
    symmetric. Finally, let $x,y,z\in S$, $(x,y)\in R$, and $(y,z)\in R$. Then $x,y\in P_1$ and
    $y,z\in P_2$ for some $P_1,P_2\in\mathcal{P}$. Since $\mathcal{P}$ is a partition of $S$, and
    $P_1\cap P_2\neq\varnothing$, $P_1=P_2$. Thus $x,y,z\in P_1$ and so $(x,z)\in R$. Therefore $R$ is
    transitive and an equivalance relation.\\

    Let $G$ and $R$ be as given. Let $x\in G$. Since $G$ is a group, $e\in G$. Now, $x=exe^{-1}$. Thus
    $(x,x)\in R$ and so $R$ is reflexive. Now, let $x,y\in G$ and $(x,y)\in R$. Then $y=gxg^{-1}$ for
    some $g\in G$. So, $x=g^{-1}yg=g^{-1}y(g^{-1})^{-1}$. Hence $(y,x)\in R$ and $R$ is symmetric.
    Finally let $x,y,z\in G$, $(x,y)\in R$, and $(y,z)\in R$. Then $y=gxg^{-1}$ and $z=hyh^{-1}$ for
    some $g,h\in G$. So, $z=hgxg^{-1}h^{-1}=hgx(hg)^{-1}$, and thus $(x,z)\in R$ since $G$ is a group
    and closed under the operation. Therefore $R$ is transitive and an equivalence relation as desired.

\end{solution}