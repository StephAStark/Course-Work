Let $(G,*)$ be a group and let $g\in G$ be fixed. Consider the function $f_g(x)=g*x$. Prove that $f_g$
is bijective.\\\\

\begin{solution}\renewcommand{\qedsymbol}{}\ \\
    Let $(G,*)$, $g$, and $f_g$ be as given. First let $x,y\in G$ and assume $f_g(x)=f_g(y)$. Then
    $g*x=g*y$. Since $G$ is a group, there exists $g^{-1}$ such that $g^{-1}*g=e$ where $e$ is the
    identity of $G$. So, multiplying both sides by $g^{-1}$ yields $g^{-1}*(g*x)=g^{-1}*(g*y)$. By
    associativity, $(g^{-1}*g)*x=(g^{-1}*g)*y$, hence $e*x=e*y$ and so $x=y$. Therefore $f_g$ is
    injective. Now, let $y\in G$. Again since $G$ is a group, there exists $g^{-1}$ such that
    $g*g^{-1}=e$. Let $x=g^{-1}*y$. Then
    
    $$f_g(x)=f_g(g^{-1}*y)=g*(g^{-1}*y)=(g*g^{-1})*y=e*y=y$$
    
    and thus $f_g$ is surjective. So, $f_g$ is bijective as desired.

\end{solution}