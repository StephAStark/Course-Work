By Lagrange's Theorem, the order of an element of a finite group is a factor of the order of the group\\

a. Suppose that $(G,*)$ is a group of order 4. Prove that every element of $G$ has order 1, 2, or 4.\\

b. Prove that if $(G,*)$ is a group of order 4 with an element of order 4, then
$(G,*)\cong(\mathbb{Z}\langle4\rangle,+)$.\\

c. Now suppose that $(G,*)$ is a group of order 4 with no element of order 4. Then by (a), every
nonidentity element of $G$ has order 2. Let $a$ and $b$ be distinct, non-identity elements of $G$. Prove
that $G=\{e,a,b,a*b\}$.\\

d. Suppose that $(G,*)$ is as in (c) above. Prove that
$G\cong\mathbb{Z}\langle2\rangle\times\mathbb{Z}\langle2\rangle$.\\\\

\begin{solution}\renewcommand{\qedsymbol}{}\ \\
    Well, let $g\in G$. Since all integer powers of $g$ are distinct, then $g$ has finite order $n\leq4$
    otherwise it would contradict $G$ having order 4. Well, if $g$ has order 1, then $g^1=g=e$. So,
    suppose that $g\neq e$. Now, assume by way of contradiction that $g$ has order 3. Then
    $\langle g\rangle$ also has order 3. Since $\langle g\rangle$ is a subgroup of $G$, we have a
    contradiction. So, all elements of $G$ have order 1, 2, or 4.\\

    Let $G$ be as given, and assume that $g\in G$ has order 4. Then, $G=\langle g\rangle$ is a finite
    cyclic group with 4 elements. Thus, $(G,*)\cong(\mathbb{Z}\langle4\rangle,+)$.\\

    Let $G$ be as given and assume that no element $g\in G$ has order 4. Now, let $a,b\in G$ be such
    that $a\neq b$ and $a,b\neq e$. Then by part a, we have that $a$ and $b$ both have order 2. Since
    $G$ is a group, $a*b\in G$, Now asssume by way of contradiction that $a*b$ has order 1. Then
    $a*b=e$. Since $b$ has order 2, we have that $a*b=b*b$ and by cancellation law, we have that $a=b$
    which contradicts $a\neq b$. Thus $a*b$ has order 2 by part a above.\\

    Let $G$ be as in part (c). Now, we have that $a*a=e$, so $a=a^{-1}$. Similarly, we see that
    $b=b^-1$. Since $(a*b)*(a*b)=e$, we have that $a*b=(a*b)^{-1}=b^{-1}*a^{-1}=b*a$, and so we see that
    $G$ is an abelian group. Now, let
    $f:G\rightarrow\mathbb{Z}\langle2\rangle\times\mathbb{Z}\langle2\rangle$ be defined by
    $f(e)=(\bar{0},\bar{0}), f(a)=(\bar{0},\bar{1}), f(b)=(\bar{1},\bar{0})$, and
    $f(a*b)=(\bar{1},\bar{1})$. Then, very clearly $f$ is bijective. So,
    
    $$f(e*e)=f(e)=(\bar{0},\bar{0})=(\bar{0},\bar{0})+(\bar{0},\bar{0})=f(e)+f(e)$$
    $$f(e*a)=f(a*e)=f(a)=(\bar{0},\bar{1})=(\bar{0},\bar{0})+(\bar{0},\bar{1})=f(e)+f(a)$$
    $$f(e*b)=f(b*e)=f(b)=(\bar{1},\bar{0})=(\bar{0},\bar{0})+(\bar{1},\bar{0})=f(e)+f(b)$$
    $$f(e*(a*b))=f((a*b)*e)=f(a*b)=f(b*a)=(\bar{1},\bar{1})$$
    $$=(\bar{0},\bar{1})+(\bar{1},\bar{0})=f(a)+f(b)$$
    $$f(a*a)=f(e)=(\bar{0},\bar{0})=(\bar{0},\bar{1})+(\bar{0},\bar{1})=f(a)+f(a)$$
    $$f(b*b)=f(e)=(\bar{0},\bar{0})=(\bar{1},\bar{0})+(\bar{1},\bar{0})=f(b)+f(b)$$
    $$f((a*b)*a)=f(a*(a*b))=f((a*a)*b)=f(e*b)=f(b)=(\bar{1},\bar{0})$$
    $$=(\bar{0},\bar{1})+(\bar{1},\bar{1})=f(a)+f(a*b)$$
    $$f(b*(a*b))=f((a*b)*b)=f(a*(b*b))=f(a*e)=f(a)=(\bar{0},\bar{1})$$
    $$=(\bar{1},\bar{1})+(\bar{1},\bar{0})=f(a*b)+f(b)$$
    
    and finally,
    
    $$f((a*b)*(a*b))=f(e)=(\bar{0},\bar{0})=(\bar{1},\bar{1})+(\bar{1},\bar{1})=f(a*b)+f(a*b)$$
    
    Therefore, $f$ is an isomorphism between $G$ and
    $\mathbb{Z}\langle2\rangle\times\mathbb{Z}\langle2\rangle$. Thus,
    $G\cong\mathbb{Z}\langle2\rangle\times\mathbb{Z}\langle2\rangle$.

\end{solution}