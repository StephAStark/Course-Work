Consider the following method for solving the heat equation $u_t=u_{xx}$:

$$U^{n+2}_i=U^n_i+\frac{2k}{h^2}(U^{n+1}_{i-1}-2U^{n+1}_i+U^{n+1}_{i+1})$$

a. Determine the order of accuracy of this method (in both space and time).
b. Suppose we take $k=\alpha h^2$ for some fixed $\alpha>0$ and refine the grid. For what values of
$\alpha$ (if any) will this method be Lax-Richtmyer stable and hence convergent?\\
Hint: Consider the MOL interpretation and the stability region of the time-discretization
being used.\\
c. Is this a useful method?\\

\begin{solution}\renewcommand{\qedsymbol}{}\ \\
    a. We will need to make use of Taylor series expansions. Note that the local truncation error comes
    out to be

    $$\tau=u(x,t+2k)-u(x,t)-\frac{2k}{h^2}(u(x-h,t+k)-2u(x,t+k)+u(x+h,t+k))$$

    Now, we will expand $u(x,t+2k), u(x-h,t+k), u(x,t+k)$, and $u(x+h,t+k)$. Observe that

    $$u(x,t+2k)=u(x,t)+2ku_t+2k^2u_{tt}+\frac86k^3u_{ttt}+\mathcal{O}(k^4)$$
    $$u(x-h,t+2k)=u(x,t)-hu_x+ku_t+\frac{h^2}{2}u_{xx}-hku_{xt}+\frac{k^2}{2}u_{tt}
    -\frac{h^3}{6}u_{xxx}$$
    $$+\frac12h^2ku_{xxt}-\frac12hk^2u_{xtt}+\frac{k^3}{6}u_{ttt}$$
    $$u(x,t+k)=u(x,t)+ku_t+\frac12k^2u_{tt}+\frac16k^3u_{ttt}+\mathcal{O}(k^4)$$

    and

    $$u(x-h,t+2k)=u(x,t)+hu_x+ku_t+\frac{h^2}{2}u_{xx}+hku_{xt}+\frac{k^2}{2}u_{tt}
    +\frac{h^3}{6}u_{xxx}$$
    $$+\frac12h^2ku_{xxt}+\frac12hk^2u_{xtt}+\frac{k^3}{6}u_{ttt}$$

    Hence,

    $$\tau=u_t+ku_{tt}+\frac23k^2u_{ttt}+\mathcal{O}(k^3)-\frac{1}{h^2}(h^2u_{xx}+h^2ku_{xxt}
    +\mathcal{O}(k^4)+\mathcal{O}(h^4))$$
    $$=\frac23k^2u_{ttt}+\mathcal{O}(h^2)+\mathcal{O}(k^3)$$

    So, we see that the above method is second order accurate in both time and space.

    b. Using the MOL, we get that $U_i'=\frac{1}{h^2}(U_{i-1}-2U_i+U_{i+1})$. From this, we can get to
    $U^{n+1}=AU^n+\frac{k}{h^2}G^n$ by discretizing time using the trapazoidal method where

    $$A=I+\frac{k}{h^2}\left(\begin{array}{cccc} -2 & 1 & 0 & \cdots \\ 1 & \ddots & \ddots & \vdots 
                                              \\ 0 & \cdots & 1 & -2 \end{array}\right)$$
                                              
    Using the spectral norm of $A$, we get that the above method is LR stable if and only if 
    $|1+\frac{k}{h^2}\lambda_i|\leq1$ where $\lambda_i$ is the $i^{th}$ eigenvalue of $A$. Since the
    eigenvalues of $A$ are $\lambda_i=2(\cos(i\pi h)-1)$ for $1\leq i\leq m$, we have that the method
    is LR stable if and only if $2\frac{k}{h^2}\leq1$. Since $k=\alpha h^2$, we need $2\alpha\leq1$ or
    $\alpha\leq\frac12$.\\

    c. Since we need $\alpha\leq\frac12$, this method is limited in its use.

\end{solution}