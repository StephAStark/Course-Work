a. The m-file heat\_CN.m solves the heat equation $u_t=\kappa u_{xx}$ using the Crank-Nicolson method.
Run this code, and by changing the number of grid points, confirm that it is second-order accurate.
(Observe how the error at some fixed time such as $T= 1$ behaves as $k$ and $h$ go to zero with a fixed
relation between $k$ and $h$, such as $k= 4h$.)\\
You might want to use the function error\_table.m to print out this table and estimate the order of
accuracy, and error\_loglog.m to produce a log-log plot of the error vs. $h$. See bvp\_2.m for an
example of how these are used.\\
b. Modify heat\_CN.m to produce a new m-file heat\_trbdf2.m that implements the TR-BDF2 method on the
same problem. Test it to confirm that it is also second order accurate. Explain how you determined the
proper boundary conditions in each stage of this Runge-Kutta method.\\

\begin{solution}\renewcommand{\qedsymbol}{}\ \\
    a. We can see from the output that this method is indeed second order accurate. See code below.

    b. See code below.

\end{solution}

\newpage
\lstinputlisting{heat_CN2.m}
\newpage
\lstinputlisting{heat_trbdf2.m}