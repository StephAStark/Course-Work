a. Find the function $G(x,\bar{x})$ solving $u''(x)=\delta(x-\bar{x}), u'(0)=0, u(1)=0$ and the
functions $G_0(x)$ solving $u''(x)=0, u'(0)=1, u(1)=0$ and $G_1(x)$ solving $u''(x)=0, u'(0)=0, 
u(1)=1$.\\
b. Using this as guidance, find the general formulas for the elements of the inverse of the matrix in
equation $(2.54)$. Write out the $5\times5$ matrices $A$ and $A^{-1}$ for the case $h=0.25$.\\

\begin{solution}\renewcommand{\qedsymbol}{}\ \\
    Since we want $G(x,\bar{x})$ to be the solution to $u''(x)=\delta(x-\bar{x})$ with $u'(0)=0, u(1)=0$
    we integrate $u''(x)$ to get
    
    $$G'(x,\bar{x})=\begin{cases} 0 & 0\leq x<\bar{x} \\ 1 & \bar{x}<x<\leq1 \end{cases}$$
    
    Since $0\leq\bar{x}\leq1$, we have that $u'(0)=0$. So, integrating again yields
    
    $$G(x,\bar{x})=\begin{cases} \bar{x}-1 & 0\leq x<\bar{x} \\ x-1 & \bar{x}<x\leq1 \end{cases}$$
    
    Here we see that $G(1,\bar{x})=u(1)=0$. Thus, we have found the solution to the boundry value
    problem. Now, to find $G_0(x)$, we will integrate $u''(x)=0$. This gives us $G_0'(x)=1$. Clearly,
    $G_0'(0)=u'(0)=1$. Integrating again gives us $G_0(x)=x+C$. Hence, $G_0(1)=1+C=0$. So, $G_0(x)=x-1$.
    So, we have found the solution to the boundry value problem $u''(x)=0, u'(0)=1, u(1)=0$. Finally, to
    solve $u''(x)=0, u'(0)=0, u(1)=1$, we will integrate $u''(x)=0$. To satisfy $u'(0)=0$, we get
    $G_1'(x)=0$. Integrating again yields $G_1(x)=C$. To satisfy $u(1)=1$, we have $G_1(x)=1$. So, we
    have found the solution to the final boundry problem.\\

    Now, the general formula for the elements of $A^{-1}$ is given by the Green's functions we found
    above. The formulas are
    
    $$B_{i0}=G_0(x_i)=x_i-1, B_{i m+1}=G_1(x_i)=1$$
    
    and
    
    $$B_{ij}=hG(x_i,x_j)=\begin{cases} hx_j-h & 1\leq i\leq j \\ hx_i-h & j\leq i\leq m \end{cases}$$
    
    where $i$ and $j$ are the rows and columns of the inverse matrix respctively, $m+1$ is the dimension
    of the matrix, and $x_i,x_j$ are the grid or stencil points used. Using this and $h=0.25$, we have
    that the first column of the $5\times5$ matrix is
    
    $$\left(\begin{array}{c} -1\\ -\frac{3}{4} \\ -\frac12 \\ -\frac14 \\ 0 \end{array}\right)$$
    
    Also, the last column is given by
    
    $$\left(\begin{array}{c} 1\\ 1 \\ 1 \\ 1 \\ 1 \end{array}\right)$$
    
    Now, the middle thre columns are given by
    
    $$\left(\begin{array}{c} \frac14\cdot\frac14-\frac14 
                          \\ \frac14\cdot\frac14-\frac14
                          \\ \frac14\cdot\frac12-\frac14
                          \\ \frac14\cdot\frac34-\frac14
                          \\ \frac14\cdot1-\frac14
      \end{array}\right)
    =\left(\begin{array}{c} -\frac{3}{16}\\ -\frac{3}{16} \\ -\frac18
                         \\ -\frac{1}{16} \\ 0 \end{array}\right),
     \left(\begin{array}{c} \frac14\cdot\frac12-\frac14\\ \frac14\cdot\frac12-\frac14
                         \\ \frac14\cdot\frac12-\frac14 \\ \frac14\cdot\frac34-\frac14
                         \\ \frac14\cdot1-\frac14
     \end{array}\right)
    =\left(\begin{array}{c} -\frac{1}{8}\\ -\frac{1}{8} \\ -\frac18 \\ -\frac{1}{16} \\ 0 
     \end{array}\right),
     \left(\begin{array}{c} \frac14\cdot\frac34-\frac14\\ \frac14\cdot\frac34-\frac14
                         \\ \frac14\cdot\frac34-\frac14 \\ \frac14\cdot\frac34-\frac14
                         \\ \frac14\cdot1-\frac14 
     \end{array}\right)
    =\left(\begin{array}{c} -\frac{1}{16}\\ -\frac{1}{16} \\ -\frac{1}{16} \\ -\frac{1}{16} \\ 0
     \end{array}\right)$$
     
     Hence, our inverse matrix is
     
     $$A^{-1}=\left(\begin{array}{ccccc} -1 & -\frac{3}{16} & -\frac18 & -\frac{1}{16}& 1 
                                      \\ -\frac{3}{4} & -\frac{3}{16} & -\frac18 & -\frac{1}{16} & 1
                                      \\ -\frac12 & -\frac18 & -\frac18 & -\frac{1}{16} & 1
                                      \\ -\frac14 & -\frac{1}{16} & -\frac{1}{16} & -\frac{1}{16} & 1
                                      \\ 0 & 0 & 0 & 0 & 1 \end{array}\right)$$
                                      
    By the book, we have that
    
    $$A=16\left(\begin{array}{ccccc} -0.25 & 0.25 & 0 & 0 & 0\\ 1 & -2 & 1 & 0 & 0\\ 0 & 1 & -2 & 1 & 0
                                  \\ 0 & 0 & 1 & -2 & 1\\ 0 & 0 & 0 & 0 & 0.0625 \end{array}\right)
       =\left(\begin{array}{ccccc} -4 & 4 & 0 & 0 & 0\\ 16 & -32 & 16 & 0 & 0\\ 0 & 16 & -32 & 16 & 0
                                \\ 0 & 0 & 16 & -32 & 16\\ 0 & 0 & 0 & 0 & 1 \end{array}\right)$$
                                
    and MATLAB gives us
    
    $$A^{-1}=\frac{1}{16}\left(\begin{array}{ccccc} -16 & -3 & -2 & -1 & 16\\ -12 & -3 & -2 & -2 & 16
        \\ -8 & -2 & -2 & -1 & 16\\ -4 & -1 & -1 & -1 & 16\\ 0 & 0 & 0 & 0 & 16 \end{array}\right)
    =\left(\begin{array}{ccccc} -1 & -\frac{3}{16} & -\frac18 & -\frac{1}{16} & 1
        \\ -\frac{3}{4} & -\frac{3}{16} & -\frac18 & -\frac{1}{16} & 1
        \\ -\frac12 & -\frac18 & -\frac18 & -\frac{1}{16} & 1
        \\ -\frac14 & -\frac{1}{16} & -\frac{1}{16} & -\frac{1}{16} & 1\\ 0 & 0 & 0 & 0 & 1 
     \end{array}\right)$$
     
    which verifies our above work.

\end{solution}