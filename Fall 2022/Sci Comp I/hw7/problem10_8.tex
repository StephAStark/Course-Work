10.8. The m-file advection\_LW\_pbc.m implements the Lax-Wendroff method for the advection equation on
$0\leq x\leq1$ with periodic boundary conditions.\\
a. Observe how this behaves with $m+1=50, 100, 200$ grid points. Change the final time to $tfinal = 0.1$
and use the m-files error\_table.m and error\_loglog.m to verify second order accuracy.\\
b. Modify the m-file to create a version advection\_up\_pbc.m implementing the upwind method and verify
that this is first order accurate.\\
c. Keep m fixed and observe what happens with advection\_up\_pbc.m if the time step $k$ is reduced, e.g.
try $k = 0.4h, k = 0.2h, k = 0.1h$. When a convergent method is applied to an ODE we expect better
accuracy as the time step is reduced and we can view the upwind method as an ODE solver applied to an
MOL system. However, you should observe decreased accuracy as $k\rightarrow0$ with $h$ fixed. Explain
this apparent paradox. Hint: What ODE system are we solving more accuracy? You might also consider the
modified equation (10.44).\\


\begin{solution}\renewcommand{\qedsymbol}{}\ \\
    a. \\

    % For this part, consider editing the function to allow tfinal to be an input along with m. Then
    % just run the LW code you made with 2 for loops where the outer one is for the different tfinal
    % values. Make sure to implement the error functions
    b. \\

    c.

\end{solution}