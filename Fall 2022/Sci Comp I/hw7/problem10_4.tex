10.4. Consider the method

$$U^{n+1}_j=U^n_j-\frac{ak}{2h}(U^n_j-U^n_{j-1}+U^{n+1}_j -U^{n+1}_{j-1})$$

for the advection equation $u_t+au_x = 0$ on $0\leq x\leq1$ with periodic boundary conditions.\\
a. This method can be viewed as the trapezoidal method applied to an ODE system $U'(t) = AU(t)$
arising from a method of lines discretization of the advection equation. What is the matrix $A$? Don't
forget the boundary conditions.\\
b. Suppose we want to fix the Courant number $\frac{ak}{h}$ as $k,h\rightarrow0$. For what range of
Courant numbers will the method be stable if $a>0$? If $a<0$? Justify your answers in terms of
eigenvalues of the matrix $A$ from part (a) and the stability regions of the trapezoidal method.\\
c. Apply von Neumann stability analysis to the method (E10.4a). What is the amplification factor
$g(\xi)$?\\
d. For what range of $\frac{ak}{h}$ will the CFL condition be satisfied for this method (with periodic
boundary conditions)?\\
e. Suppose we use the same method (E10.4a) for the initial-boundary value problem with $u(0, t)=g_0(t)$
specified. Since the method has a one-sided stencil, no numerical boundary condition is needed at the
right boundary (the formula (E10.4a) can be applied at $x_{m+1}$). For what range of $\frac{ak}{h}$ will
the CFL condition be satisfied in this case? What are the eigenvalues of the $A$ matrix for this case
and when will the method be stable?\\

\begin{solution}\renewcommand{\qedsymbol}{}\ \\
    a. The matrix $A$ is given by

    $$A=\left(\begin{array}{ccccc} -a & 0 & \cdots & \cdots & a \\ a & -a & 0 & \cdots & 0
                                \\ 0 & \ddots & \ddots & \ddots & \vdots
                                \\ \vdots & \ddots & \ddots & \ddots & \vdots
                                \\ 0 & \cdots & \cdots & a & -a \end{array}\right)$$

    with the consideration of the periodic boundry conditions.\\

    b. \\

    c. \\

    d. \\

    e.

\end{solution}