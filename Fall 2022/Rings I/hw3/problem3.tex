Let $R$ be any ring and $M$ a left $R$-module that is both artinian and noetherian. Prove that for
any $R$-module homomorphism $\phi:M\rightarrow M$,there exists $n\in\mathbb{Z}^+$ such that
$\phi^n(M)\cap\ker(\phi^n)=\{0\}$.\\\\

\begin{solution}\renewcommand{\qedsymbol}{}\ \\
    Let $R$ and $M$ be as given. Let $\phi:M\rightarrow M$ be an $R$-module homomorphism. Consider the
    ascending chain

    $$\phi(M)\subset\phi^2(M)\subset\phi^3(M)\subset\ldots$$

    and the descending chain

    $$\phi(M)\supset\phi^2(M)\supset\phi^3(M)\supset\ldots$$

    Since $\phi$ is an endomorphism, $\phi^i(M)\in M$ for all $i\in\mathbb{Z}^+$. Since $M$ is
    noetherian, we have that

    $$\phi(M)\subset\phi^2(M)\subset\phi^3(M)\subset\ldots\subset\phi^l(M)=\phi^{l+1}(M)=\ldots$$

    for some $l\in\mathbb{Z}^+$ and since $M$ is also artinian,

    $$\phi(M)\supset\phi^2(M)\supset\phi^3(M)\supset\ldots\supset\phi^m(M)=\phi^{m+1}(M)=\ldots$$

    for some $m\in\mathbb{Z}^+$. Now set $n=\max\{l,m\}$. Then, $\phi^n(M)=\phi^{n+1}(M)=\ldots$
    satisfies both the ascending and descending chain conditions on $M$. Clearly
    $0\in\phi^n(M)\cap\ker(\phi^n)$ since $\phi$ is an endomorphism. Now, let
    $x\in\phi^n(M)\cap\ker(\phi^n)$. That is, $\phi^n(x)=0$ and there exists $y\in M$ such that
    $\phi^n(y)=x$. So, $\phi^{n+1}(y)=\phi(x)$ and $\phi^{n+2}(y)=\phi^2(x)$. Continuing this process
    of composing with $\phi$ on both sides, we see that $\phi^{2n}(y)=\phi^n(x)=0$. Whence, $x=0$.
    Since $x\in\phi^n(M)\cap\ker(\phi^n)$ was arbitrary, $\phi^n(M)\cap\ker(\phi^n)\subseteq\{0\}$.
    Therefore, $\phi^n(M)\cap\ker(\phi^n)=\{0\}$ for some $n\in\mathbb{Z}^+$ as desired.

\end{solution}