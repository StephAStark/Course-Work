Let $R$ be a ring possibly without an identity. An element $e\in R$ is called a left (resp. right)
identity for $R$ if $ea=a$(resp.$ae=a$)for every $a\in R$.\\
(a) Show that a left identity for $R$ need not be a right identity.\\
(b) Show that if $R$ has a unique left identity $e$, then $e$ is also a right identity.\\
(Hint.For(b),consider$(e+ae-a)c$ for arbitrary $a,c\in R$.)\\\\

\begin{solution}\renewcommand{\qedsymbol}{}\ \\
    Let $R$ be as above.\\

    a. Consider the ring of $2\times2$ real matricies with 0's in the bottom row. That is, the ring
    $R=\{\left(\begin{array}{cc} a & b \\ c & d \end{array}\right)|a,b\in\mathbb{R}\;\;
    \text{and}\;\;c,d=0\}$. Further consider the element
    $el=\left(\begin{array}{cc} 1 & 0 \\ 0 & 0 \end{array}\right)\in R$. Now, let $m\in R$. So,

    $$el\cdot m=\left(\begin{array}{cc} 1 & 0 \\ 0 & 0 \end{array}\right)
    \left(\begin{array}{cc} a & b \\ 0 & 0 \end{array}\right)
    =\left(\begin{array}{cc} a & b \\ 0 & 0 \end{array}\right)$$
    
    However,
    
    $$m\cdot el=\left(\begin{array}{cc} a & b \\ 0 & 0 \end{array}\right)
    \left(\begin{array}{cc} 1 & 0 \\ 0 & 0 \end{array}\right)
    =\left(\begin{array}{cc} a & 0 \\ 0 & 0 \end{array}\right)$$
    
    Since $m\in R$ was arbitrary, we have that $el$ is a left identity, but not a right identity.\\

    b. Let $e\in R$ be the unique left identity for $R$. Let $a,c\in R$ and consider $(e+ae-a)c$. By
    distribution, we have that
    
    $$(e+ae-a)c=ec+(ae)c-ac=c+a(ec)-ac=c+ac-ac=c$$
    
    Since we have that $ec=(e+ae-a)c$ and that $e$ is the unique left identity for $R$, $e=e+ae-a$.
    Hence, $ae-a=0$ and thus $ae=a$. Therefore, $e$ is also a right identity for $R$ as desired.

\end{solution}