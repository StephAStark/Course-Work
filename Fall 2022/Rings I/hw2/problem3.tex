3. Let $R$ be a ring, and consider the following commuting diagram of left $R$-modules and $R$-module
homomorphisms, where each row is a short exact sequence.\\
\begin{center}
\begin{tikzcd}
    0 \arrow{r} & M_1 \arrow{r}{\phi_1} \arrow[swap]{d}{\rho_1} & M_2 \arrow{r}{\phi_2} 
    \arrow[swap]{d}{\rho_2} & M_3 \arrow{r} \arrow[swap]{d}{\rho_3} & 0\\
    0 \arrow{r} & N_1 \arrow{r}{\psi_1} & N_2 \arrow{r}{\psi_2} & N_3 \arrow{r} & 0
\end{tikzcd}
\end{center}
(a) Show that if $\rho_1$ and $\rho_3$ are $1-1$, then so is $\rho_2$.\\
(b) Show that if $\rho_1$ and $\rho_3$ are onto, then so is $\rho_2$.\\\\

\begin{solution}\renewcommand{\qedsymbol}{}\ \\
    Let the diagram be as given above.

    (a) Assume that $\rho_1$ and $\rho_3$ are $1-1$. Now, let $m\in\ker(\rho_2)$. Hence, $\rho_2(m)=0$.
    Since the above diagram commutes, we have that $(\rho_3(\phi_2(m))=(\psi_2(\rho_2(m))$. Since
    $\rho_2(m)=0$ and $\psi_2(0)=0$ by $\psi_2$ be an $R$-module homomorphism, $(\rho_3(\phi_2(m))=0$.
    By assumption $\rho_3$ is $1-1$, and therefore, $\ker(\rho_3)=\{0\}$. That is, $\phi_2(m)=0$. Since
    each row is a short exact sequence, we have that $\phi_1(M_1)=\ker(\phi_2)$. So, $m\in\phi(M_1)$
    means that there exists $m'\in M_1$ such that $\phi(m')=m$. Again, since the given diagram commutes,
    we have that $\psi_1(\rho_1(m'))=\rho_2(\phi_1(m'))$. Since $\phi(m')=m$ and $\rho_2(m)=0$,
    $\psi_1(\rho_1(m'))=0$. We also know that $\psi_1$ is $1-1$ since the each row is a short exact
    sequence by Observation 8. Hence, $\rho_1(m')=0$ and since $\rho_1$ is $1-1$ by assumption, $m'=0$.
    Thus, $\phi_1(m')=\phi_1(0)=0=m$. Hence, $\ker(\rho_2)=\{0\}$ and so $\rho_2$ is $1-1$ as desired.\\

    (b) Assume that $\rho_1$ and $\rho_3$ are onto. Let $z\in N_3$. Since we assume that $\rho_3$ is
    onto, there exists $c\in M_3$ such that $\rho_3(c)=z$. Since each row is a short exact sequence, we
    have that $\phi_2$ is also onto. Hence there exists $b\in M_2$ such that $\phi_2(b)=c$. So,
    $\rho_3(\phi_2(b))=z$. Since the diagram commutes, we have that
    $\rho_3(\phi_2(b))=\psi_2(\rho_2(b))=z$. Since each row is a short exact sequence, we also know that
    there exists $y\in N_2$ such that $\psi_2(y)=z$. Now, let $y'\in\ker(\psi_2)$. Then, we have that
    $y'\in\psi_1(N_1)$ since the bottom row is a short exact sequence. So, there exists $x\in N_1$ such
    that $\psi_1(x)=y'$. Since we assume that $\rho_1$ is onto, we have that there exists $a\in M_1$
    such that $\rho_1(a)=x$. So, $\psi_1(\rho_1(a))=y'$. Again, since the diagram commutes, we have that
    $\rho_2(\phi_1(a))=\psi_1(\rho_1(a))=y'$. Thus, we have that $\rho_2$ is also onto as desired.


\end{solution}