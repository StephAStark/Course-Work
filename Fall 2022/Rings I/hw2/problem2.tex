2. Give an example of a ring $R$ and left $R$-modules $M,N,L$ such that $M\oplus L\cong N\oplus L$, but
$M\ncong N$.\\\\

\begin{solution}\renewcommand{\qedsymbol}{}\ \\
    Consider the ring $\mathbb{R}$ and the left $\mathbb{R}$-modules
    $\{0\},\bigoplus_{i\in\mathbb{N}}\mathbb{R},$ and $\bigoplus_{i\in\mathbb{N}}\mathbb{R}$. Then, we
    have that
    $\{0\}\oplus\bigoplus_{i\in\mathbb{N}}\mathbb{R}\cong\bigoplus_{i\in\mathbb{N}}\mathbb{R}$. Also,
    $\bigoplus_{i\in\mathbb{N}}\mathbb{R}\oplus\bigoplus_{i\in\mathbb{N}}\mathbb{R}\cong
    \bigoplus_{i\in\mathbb{N}}\mathbb{R}$ via the map
    $\pi:\bigoplus_{i\in\mathbb{N}}\mathbb{R}\rightarrow\bigoplus_{i\in\mathbb{N}}\mathbb{R}\oplus
    \bigoplus_{i\in\mathbb{N}}\mathbb{R}$ defined by
    $\pi((a_i)_{i\in\mathbb{N}})=((a_{2i})_{i\in\mathbb{N}},(a_{2i+1})_{i\in\mathbb{N}})$. Clearly,
    $\pi$ is a module homomorphism. Also, we see that $\ker(\pi)=\{0\}$ as $0$ is the only element in
    $\bigoplus_{i\in\mathbb{N}}\mathbb{R}$ such that $\pi(0)=0=(0,0)$. Thus, $\pi$ is $1-1$. Also,
    $\pi$ is onto since for any element in
    $\bigoplus_{i\in\mathbb{N}}\mathbb{R}\oplus\bigoplus_{i\in\mathbb{N}}\mathbb{R}$, there are finitely
    many nonzero terms in each tuple. Since we have that
    $\bigoplus_{i\in\mathbb{N}}\mathbb{R}\oplus\bigoplus_{i\in\mathbb{N}}\mathbb{R}\cong
    \bigoplus_{i\in\mathbb{N}}\mathbb{R}$ and
    $\{0\}\oplus\bigoplus_{i\in\mathbb{N}}\mathbb{R}\cong\bigoplus_{i\in\mathbb{N}}\mathbb{R}$,
    $\bigoplus_{i\in\mathbb{N}}\mathbb{R}\oplus\bigoplus_{i\in\mathbb{N}}\mathbb{R}\cong\{0\}\oplus
    \bigoplus_{i\in\mathbb{N}}\mathbb{R}$. However, $\{0\}\ncong\bigoplus_{i\in\mathbb{N}}\mathbb{R}$.


\end{solution}