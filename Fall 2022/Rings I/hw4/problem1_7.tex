Let $B_1,\ldots, B_n$ be left ideals (resp. ideals) in a ring $R$. Show that
$R=B_1\oplus\ldots\oplus B_n$ if and only if there exist idempotents (resp. central idempotents)
$e_1,\ldots, e_n$ with sum 1 such that $e_ie_j = 0$ whenever $i\neq j$, and $B_i=Re_i$ for all $i$. In
the case where the $B_i$'s are ideals, if $R=B_1\oplus\ldots\oplus B_n$ , then each $B_i$ is a ring with
identity $e_i$, and we have an isomorphism between $R$ and the direct product of rings
$B_1\times\cdots\times B_n$. Show that any isomorphism of $R$ with a finite direct product of rings
arises in this way.\\

\begin{solution}\renewcommand{\qedsymbol}{}\ \\
    Let $R$ and $B_i$ for $1\leq i\leq n$ be as given above. Assume first that
    $R=B_1\oplus\ldots\oplus B_n$. Consider $e_i\in B_i$ for $1\leq i\leq n$ such that
    $1=e_1+\ldots+e_n$. Now, multiply this equation on the left by $e_i$. Then we have
    $e_i=e_ie_1+\ldots+e_ie_n$. Since we have that $B_j$ for $1\leq j\leq n$ are left ideals of
    $R$, $e_ie_j\in B_j$. Since $e_i\in B_i$, we have that $e_ie_j=0$ for all $i\neq j$. Hence,
    $e_i=e_ie_i=e_i^2$. That is $e_i$ is an idempotent for $B_i$ for all $1\leq i\leq n$. Also, we see
    that $B_i=Re_i$ for all $i$. So, now assume the opposite direction. Now, let $x\in R$. Then,
    $x=x1=x(e_1+\ldots+e_n)$. Since $B_i=Re_i$, we have that $(e_1+\ldots+e_n)\in B_1+\ldots+B_n$. So,
    $R=B_1+\ldots+B_n$. Now, consider $x\in B_i\cap B_j$ with $i\neq j$. Then, again since $B_i=Re_i$,
    $x=r_1e_i=r_2e_j$. Now, multiplying by $e_i$ on the right, we get $r_1e_ie_i=r_1e_i=r_2e_je_i=0$
    since $e_ie_j=0$ whenever $i\neq j$. So, $x=0$. Thus, $R=B_1\oplus\ldots\oplus B_n$ as desired.

\end{solution}