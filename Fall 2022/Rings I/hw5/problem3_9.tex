Let $R, S$ be rings such that $\mathbb{M}_m(R)\cong\mathbb{M}_n(S)$. Does this imply that $m=n$ and
$R\cong S$?\\
b. Let us call a ring $A$ a matrix ring if $A\cong\mathbb{M}_m(R)$ for some integer $m\geq2$ and some
ring $R$. True or False: "A homomorphic image of a matrix ring is also a matrix ring"?\\

\begin{solution}\renewcommand{\qedsymbol}{}\ \\
    a. This implication is not true. Consider the rings $R=\mathbb{M}_3(\mathbb{Q})$ and $\mathbb{Q}$.
    Then, $\mathbb{M}_3(\mathbb{M}_3(\mathbb{Q}))\cong\mathbb{M}_9(\mathbb{Q})$. However, $3\neq9$ and
    $R\ncong S$.\\

    b. Let $R$ be a ring and $A$ a matrix ring. Then $A\cong\mathbb{M}_m(R)$ for some $m\geq2$. Now,
    let $I$ be an ideal of $A$. Then, $I=\mathbb{M}_m(J)$ where $J$ is an ideal of $R$. Then,
    $\mathbb{M}_m(R)/I=\mathbb{M}_m(R)/\mathbb{M}_m(J)=\mathbb{M}_m(R/J)$. Let
    $f:\mathbb{M}_m(R)\rightarrow\mathbb{M}_m(R/J)$ be given by $f(x)=x+\mathbb{M}_m(J)$. We first see
    that $\ker(f)=\mathbb{M}_m(J)$. Also, note that $f$ is surjective since any $x+\mathbb{M}_m(J)$ can
    be be given by $f(x)$ for $x\in\mathbb{M}_m(R)$. Now, let $x,y\in\mathbb{M}_m(R)$ and $r\in R$.
    Then, $f(x+y)=(x+y)+\mathbb{M}_m(J)=x+\mathbb{M}_m(J)+\mathbb{M}_m(J)=f(x)+f(y)$, and also
    $f(rx)=rx+\mathbb{M}_m(J)=r(x+\mathbb{M}_m(J))=rf(x)$. Thus, $f$ is a surjective $R$-module
    homomorphism, and therefore, by the Fundemental Homomorphism Theorem, $f(A)$ is also a matrix ring.

\end{solution}