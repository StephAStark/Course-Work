A left $R$-module $M$ is said to be cohopfian if any injective $R$-endomorphism of $M$ is an
automorphism.\\
1. Show that any artinian module $M$ is cohopfian.\\
2. Show that the left regular module $_RR$ is cohopfian iff every non right-$0$-divisor in $R$ is a
unit. In this case, show that $_RR$ is also hopfian\\

\begin{solution}\renewcommand{\qedsymbol}{}\ \\
    1. Let $R$ be a ring and $M$ an artinian $R$-module. Let $f$ be an injective $R$-endomorphism of
    $M$. Consider the decending chain of submodules of $M$ by
    $M\supseteq f(M)\supseteq f^2(M)\supseteq f^3(M)\supseteq\ldots$. Since $M$ is artinian, this chain
    must stabilize for some $n\in\mathbb{Z}^+$. That is $f^n(M)=f^{n+1}(M)=\ldots$. Hence,
    $f(f^{n-1}(M))=f(f^n(M))$. Since $f$ is $1-1$, we have that $f^{n-1}(M)=f^n(M)$. Repeating this
    process for $n-1$ more iterations, using that $f$ is injective, yields $M=f(M)$. Thus, $f$ is
    surjective and therefore an automorphism. That is $M$ is cohopfian as desired.\\

    2. Let $R$ be a ring. First assume that $_RR$ is cohopfian. Let $u\in R$ be a non right-0-divisor.
    Consider that map $f:\;_RR\rightarrow\;_RR$ given by $f(r)=ru$ for all $r\in\;_RR$. Now, let
    $a,b\in\;_RR$ and $m\in R$. Then, $f(a+b)=(a+b)u=au+bu=f(a)+f(u)$ and $f(ma)=(ma)u=mau=m(au)=f(a)$.
    So, $f$ is an endomorphism of $_RR$. Now assume that $f(a)=f(b)$. Whence, $au=bu$ and therefore,
    $au-bu=(a-b)u=0$. Since $u$ is not a right-0-divisor, we have that $a-b=0$. So, $a=b$. That means
    $f$ is also $1-1$ and so by assumption, $f$ is an automorphism. So, there exists $v\in R$ such that
    $f(v)=vu=1$. Hence, $uvu=u$ and so $uvu-u=(uv-1)u=0$. Since $u$ is not a right-0-divisor, $uv-1=0$.
    That is $uv=1$ and thus $u$ is a unit in $R$. Conversley assume that every non right-0-divisor in
    $R$ is a unit. Let $g$ be an injective endomorphism of $_RR$. So, $g(r)=ru$ for all $r\in\;_RR$ and
    some $u\in\;_RR$. Since $g$ is $1-1$, if $g(r)=0$, then $r=0$. So, if $g(r)=ru=0$, $r=0$. This means
    that $u$ is not a right-0-divisor. So by assumption, $u$ is a unit in $R$. Hence, there exists
    $v\in R$ such that $uv=vu=1$. Now, let $s\in\;_RR$. Since $_RR$ is an $R$-module, $sv\in\;_RR$.
    Whence, $g(sv)=(sv)u=svu=s$. Thus, $g$ is sujective making $_RR$ cohopfian since $g$ was arbitrary.
    Now, assume that $_RR$ is cohopfian. Let $h$ be a surjective endomorphism of $_RR$. Since $h$ is
    surjective, there exists $x\in\;_RR$ such that $h(x)=0$. We know that every endomorphism has the
    form $h(x)=xa$ for all $x\in\;_RR$ and some $a\in R$. So, $h(x)=xa=0$. Let $x,y\in\;_RR$ and assume
    that $h(x)=h(y)$. Then, $h(x)-h(y)=h(x-y)=0$ since $h$ is an endomorphism. Hence, $(x-y)a=0$.
    Clearly, $a\neq0$ since $h$ is onto. So, since $a$ is not a right-0-divisor, $a$ is a unit in $R$.
    Therefore, there exists $b\in R$ such that $ab=ba=1$. Whence, $(x-y)ab=x-y=0$ and so $x=y$. Thus,
    $h$ is $1-1$ and an automorphism. Therfore, $_RR$ is hopfian.

\end{solution}