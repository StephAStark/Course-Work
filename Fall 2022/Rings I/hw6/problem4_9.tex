Let $R$ be a $J$-semisimple domain and $a$ be a nonzero central element of $R$. Show that the
intersection of all maximal left ideals not containing $a$ is zero.\\

\begin{solution}\renewcommand{\qedsymbol}{}\ \\
    Let $R$ and $a$ be as above. If $a$ is not in any maximal left ideal of $R$, then the intersection
    of all maximal left ideals of $R$ not containing $a$ is $rad R=\{0\}$ by assumption. Similarly, if
    $a$ is in all nonzero maximal ideals of $R$, then the intersection of all maximal left ideals of $R$
    not containing $a$ is simply the set $\{0\}$. So, let $X$ be the collection of all maximal left
    ideals of $R$ not containg $a$. Assume by way of contradiction that $\bigcap X\neq\{0\}$. So, there
    exists $b\in\bigcap X$ with $b\neq0$ Since $X$ is also an ideal, $ab\in X$. Cosider $\mathcal{R}$
    the intersection of all maximal left ideals of $R$. So, $\mathcal{R}\setminus X=Y$ the intersection
    of all maximal ideals of $R$ containing $a$. So, $ab\in Y$. That is, $ab=ba\in X\cap Y=rad R$. Since
    $a,b\neq0$ and $R$ is a domain, $ab\neq0$. This contradicts $R$ being $J$-semisimple. Thus,
    $\bigcap X=\{0\}$ as desired. \\

    \textit{In case a is to be considered an idempotent}: Let $R$ be a $J$-semisimple domain and $a$ a
    nonzero idempotent. Let $X$ be the intersection of all maximal ideals of $R$ not containg $a$. If
    $X$ is all of the maximal ideals of $R$, then $X=rad R=\{0\}$. If $a$ is in every nonzero maximal
    ideal, then $X=\{0\}$. Now consider $b\in X$. Now, $ab=a^2b$ as $a$ is an idempotent. So,
    $ab-a^2b=a(b-ab)=a(1-a)b=0$. Since $R$ is a domain, either $a=0, 1-a=0$, or $b=0$. By assumption,
    $a\neq0$, and if $1-a=0$, then $1=a$ and no maximal ideal of $R$ can contain $1$ which would result
    in the trivial case above. So, we have that $b=0$ and since $b$ was arbitrary, $X=\{0\}$ as desired.

\end{solution}