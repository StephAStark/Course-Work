Let $R$ be a commutative ring, and $S$ be a ring. Prove that $S$ is an $R$-algebra if and only if there
is a ring homomorphism $\phi:R\rightarrow S$ such that for all $a\in\phi(R)$ and $b\in S$, $ab=ba$.

\begin{solution}\renewcommand{\qedsymbol}{}\ \\
    Let $R$ and $S$ be as given. First, assume first that there exists a ring homomorphism
    $\phi:R\rightarrow S$ such that for all $a\in\phi(R)$ and $b\in S$, $ab=ba$. Define the action on
    $S$ by $\phi(a)b=ab$ for all $a\in R$ and $b\in S$. Now, let $a\in R$ and $b,c\in S$. Then we have
    that $a(bc)=\phi(a)(bc)=(\phi(a)b)c=(ab)c$ by the definition of $\phi$ and by associativity on $S$.
    Then, $(ab)c=(\phi(a)b)c=(b\phi(a))c=b(\phi(a)c)=b(ac)$ by our assumption about the homomorphism
    $\phi$, the action we defined for $\phi$, and by the associativity on $S$. Thus, we have that $S$ is
    an $R$-algebra. Now assume that $S$ is an $R$-algebra. Define the function $\phi:R\rightarrow S$ by
    $\phi(a)=a$. So clearly, $\phi(1_R)=1_S$. Now, let $x,y\in R$. Then,
    $\phi(x+y)=(x+y)=x+y=\phi(x)+\phi(y)$ and $\phi(xy)=(xy)=xy=\phi(x)\phi(y)$ by the definition of
    $\phi$. Thus, $\phi$ is a ring homomorphism. Let $x\in R$ and $y\in S$. Then, by definition of
    $\phi$, $x=\phi(x)\in\phi(R)$. Since $S$ is an $R$-algebra, we have that $xy=x(y1_S)=y(x1_S)=yx$.
    Thus, since $x$ and $y$ were arbitrary, we have that there exists a ring homomorphism such that for
    all $a\in\phi(R)$ and $b\in S$, $ab=ba$.

\end{solution}