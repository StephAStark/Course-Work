Let $R$ be a ring and $\phi:M\rightarrow N$ be a homomorphism of left $R$-modules. Prove that
$\phi$ is $1-1$ if and only if $\ker(\phi)=\{0\}$.

\begin{solution}\renewcommand{\qedsymbol}{}\ \\
    Let $R$ and $\phi$ be as given, and let $M,N$ be left $R$-modules. First, we will show that
    $\phi(0_M)=0_N$. Since $\phi$ is a homomorphism and $0_M=0_M+0_M$, we have that
    $\phi(0_M)=\phi(0_M+0_M)=\phi(0_M)+\phi(0_M)$. Now, $\phi(0_M)=0_N+\phi(0_M)$. Therefore,
    $0_N+\phi(0_M)=\phi(0_M)+\phi(0_M)$ and by cancellation, $0_N=\phi(0_M)$. Thus we have that
    $0\in\ker(\phi)$. Now, assume first that $\phi$ is injective. Let $m\in M$ such that $\phi(m)=0$.
    We will show that $m=0$. We have that $\phi(m)=\phi(m+0)=\phi(m)+\phi(0)$ since $\phi$ is a
    homomorphism. By our assumption and by what was shown above,
    $\phi(m)+\phi(0)=0+\phi(0)=\phi(0)+\phi(0)$. So by cancellation, $\phi(m)=\phi(0)$. Since
    $\phi$ is $1-1$, we have that $m=0$ and since $m\in M$ was arbitrary and we showed that
    $0\in\ker(\phi)$, we have that $\ker(\phi)=\{0\}$. On the other hand, asssume that
    $\ker(\phi)=\{0\}$. Let $m,n\in M$ and assume that $\phi(m)=\phi(n)$. We can now see that
    $\phi(m)-\phi(n)=0$ and since $\phi$ is a homomorphism, $\phi(m-n)=0$. Since $\ker(\phi)=\{0\}$, we
    have that $m-n=0$. Hence, $m=n$. So by definition, $\phi$ is $1-1$. Thus $\phi$ is $1-1$ if and only
    if $\ker(\phi)=\{0\}$ as desired.

\end{solution}