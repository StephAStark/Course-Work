8. Let $R$ be a ring, $M$ a left $R$-module, and $N$ an $R$-submodule of $M$. Prove that if $M/N$ and
$N$ are both finitely generated as left $R$-modules, then so is $M$.

\begin{solution}\renewcommand{\qedsymbol}{}\ \\
    Let $R, M$, and $N$ be as given, and suppose that $M/N$ and $N$ are both finitely generated as left
    $R$-modules. We then have that $N=RS$ and $M/N=RT$ for some finite sets $S\subseteq N$ and
    $T\subseteq M/N$. Now, define $\phi:M\rightarrow M/N$ by $\phi(m)=m+N$. By theorem 8, we have that
    $\phi$ is an $R$-module homomorphism. Now, take $X=\{x_1,\ldots, x_n\}\subseteq M$ for some
    $n\in\mathbb{N}$ such that $\phi(X)=T$. Let $m\in M$. if $m\in\ker(\phi)$, then we have that
    $\phi(m)=0$. Thus, $m\in N$ and $m=a_1s_1+\cdots+a_{\alpha}s_{\alpha}$ where $\alpha\in\mathbb{N}$,
    $a_i\in R$ and $s_i\in S$ for $1\leq i\leq\alpha$. If, however, $m\notin\ker(\phi)$, then
    $\phi(m)\in M/N$. Hence, since $\phi$ is a homomorphism,
    
    $$\phi(m)=a_1\phi(x_1)+\cdots+a_n\phi(x_n)=\phi(a_1x_1+\cdots+a_nx_n)$$
    
    where $a_i\in R$ for $1\leq i\leq n$. Then,
    
    $$\phi(m)-\phi(a_1x_1+\cdots+a_nx_n)=\phi(m-(a_1x_1+\cdots+a_nx_n))=0$$
    
    since $\phi$ is a homomorphism. Therefore $m-(a_1x_1+\cdots+a_nx_n)\in N$. So,
    
    $$m-(a_1x_1+\cdots+a_nx_n)=b_1s_1+\cdots+b_{\alpha}s_{\alpha}$$
    
    where $b_i\in R$ for $1\leq i\leq\alpha$. Thus,
    
    $$m=(b_1s_1+\cdots+b_{\alpha}s_{\alpha})+(a_1x_1+\cdots+a_nx_n)$$
    
    and so $m\in R(S\cup X)$. Since $m\in m$ was arbitrary and $S$ and $X$ were both finite, we have
    that $M$ is finitely generated as $S\cup X$ is also finite.

\end{solution}