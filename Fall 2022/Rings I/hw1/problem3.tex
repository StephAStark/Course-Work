Let $F$ be a field, $n>1$, $R=\mathbb{M}_n(F)$, and $M\subseteq R$ the set of all matricies that have
arbitrary entries in the first column but zeros elsewhere. Show that $M$ is an $R$-submodule of $ _RR$,
but not an $R$-submodule of $R_R$.

\begin{solution}\renewcommand{\qedsymbol}{}\ \\
    Let $F, n, R$, and $M$ be as given. Let $z\in R$ and $x,y\in M$. Clearly $M\neq\emptyset$ since the
    zero $n\times n$ matrix

    $$\left(\begin{array}{ccc} 0 & \cdots & 0 \\ \vdots & \ddots & \vdots
                            \\ 0 & \cdots & 0 \end{array}\right)\in M$$
    
    Since $x,y\in M$ we have that

    $$x=\left(\begin{array}{cccc} a_1 & 0 & \cdots & 0 \\ a_2 & 0 & \ddots & \vdots
                               \\ \vdots & \vdots & \ddots & \vdots \\ a_n & 0 & \cdots & 0
        \end{array}\right)$$
    
    and

    $$y=\left(\begin{array}{cccc} b_1 & 0 & \cdots & 0 \\ b_2 & 0 & \ddots & \vdots
                               \\ \vdots & \vdots & \ddots & \vdots \\ b_n & 0 & \cdots & 0
        \end{array}\right)$$
    
    where $a_i,b_i\in F$ for $1\leq i\leq n$. Then

    $$x+y=\left(\begin{array}{cccc} a_1 & 0 & \cdots & 0 \\ a_2 & 0 & \ddots & \vdots
                                 \\ \vdots & \vdots & \ddots & \vdots \\ a_n & 0 & \cdots & 0
          \end{array}\right)$$
    $$+\left(\begin{array}{cccc} b_1 & 0 & \cdots & 0 \\ b_2 & 0 & \ddots & \vdots
                              \\ \vdots & \vdots & \ddots & \vdots \\ b_n & 0 & \cdots & 0
       \end{array}\right)
      =\left(\begin{array}{cccc} (a_1+b_1) & 0 & \cdots & 0 \\ (a_2+b_2) & 0 & \ddots & \vdots
                              \\ \vdots & \vdots & \ddots & \vdots\\ (a_n+b_n) & 0 & \cdots & 0
       \end{array}\right)\in M$$
    
    So, $M$ is closed under addition. Since 
    
    $$z\in R, z=\left(\begin{array}{cccc} c_{11} & c_{12} & \cdots & c_{1n}
                                         \\ c_{21} & c_{22} & \ddots & \vdots
                                         \\ \vdots & \vdots & \ddots & \vdots
                                         \\ c_{n1} & c_{n2} & \cdots & c_{nn} \end{array}\right)$$
    
    where $c_{ij}\in F$ for $1\leq i,j\leq n$. Hence,

    $$zx=\left(\begin{array}{cccc} c_{11} & c_{12} & \cdots & c_{1n}\\ c_{21} & c_{22} & \ddots & \vdots
                                                  \\ \vdots & \vdots & \ddots & \vdots 
                                                  \\ c_{n1} & c_{n2} & \cdots & c_{nn}
         \end{array}\right)
         \left(\begin{array}{cccc} a_1 & 0 & \cdots & 0 \\ a_2 & 0 & \ddots & \vdots
                                \\ \vdots & \vdots & \ddots & \vdots \\ a_n & 0 & \cdots & 0
         \end{array}\right)$$
    $$=\left(\begin{array}{cccc} (a_1c_{11}+a_2c_{12}+\cdots+a_nc_{1n}) & 0 & \cdots & 0
                              \\ (a_1c_{21}+a_2c_{22}+\cdots+a_nc_{2n}) & 0 & \ddots & \vdots
                              \\ \vdots & \vdots & \ddots & \vdots
                              \\ (a_1c_{n1}+a_2c_{n2}+\cdots+a_nc_{nn}) & 0 & \cdots & 0
        \end{array}\right)\in M$$
    
    However,

    $$xz=\left(\begin{array}{cccc} a_1 & 0 & \cdots & 0 \\ a_2 & 0 & \ddots & \vdots
                                \\ \vdots & \vdots & \ddots & \vdots \\ a_n & 0 & \cdots & 0
         \end{array}\right)
         \left(\begin{array}{cccc} c_{11} & c_{12} & \cdots & c_{1n}\\ c_{21} & c_{22} & \ddots & \vdots
                                \\ \vdots & \vdots & \ddots & \vdots\\ c_{n1} & c_{n2} & \cdots & c_{nn}
        \end{array}\right)$$
    $$=\left(\begin{array}{cccc} a_1c_{11} & a_1c_{12} & \cdots & a_1c_{1n}
                              \\ a_2c_{11} & a_2c_{12} & \ddots & \vdots
                              \\ \vdots & \vdots & \ddots & \vdots
                              \\ a_nc_{11} & a_nc_{12} & \cdots & a_nc_{1n} \end{array}\right)\notin M$$
    
    Thus, we have that $M$ is an $R$-submodule of $ _RR$ but is not an $R$-submodule of $R_R$ since
    $z,x,y$ were arbitrary.


\end{solution}