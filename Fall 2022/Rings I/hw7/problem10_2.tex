Let $\mathfrak{p}\subset R$ be a prime ideal, $\mathfrak{A}$ be a left ideal and $\mathfrak{B}$ be
a right ideal. Does $\mathfrak{A}\mathfrak{B}\subseteq\mathfrak{p}$ imply that
$\mathfrak{A}\subseteq\mathfrak{p}$ or $\mathfrak{B}\subseteq\mathfrak{p}$?\\

\begin{solution}\renewcommand{\qedsymbol}{}\ \\
    Let $R, \mathfrak{p}, \mathfrak{A}$, and $\mathfrak{B}$ be as above. Assume that
    $\mathfrak{A}\mathfrak{B}\subseteq\mathfrak{p}$. Since $\mathfrak{A}$ is a left ideal,
    $\mathfrak{A}=aR$ for some element $a\in R$ and since $\mathfrak{B}$ is a right ideal, we have that
    $\mathfrak{B}=Rb$ for some $b\in R$. So, $\mathfrak{A}\mathfrak{B}=(aR)(Rb)=aRb$. Since
    $\mathfrak{p}$ is a prime ideal and we assume that $aRb\subseteq\mathfrak{p}$, we have that
    $a\in\mathfrak{A}$ or $b\in\mathfrak{B}$. So, $\mathfrak{A}\subseteq\mathfrak{p}$ or
    $\mathfrak{B}\subseteq\mathfrak{p}$. Thus, the above implication is true.

\end{solution}