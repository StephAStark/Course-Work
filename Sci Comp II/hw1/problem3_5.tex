Using the commands tic and toc, study the time taken by MATLAB (which is far from optimal in this
regard) to compute an $N$-point FFT, as a function of $N$. If $N=2^k$ for $k=0,1, ... ,15$, for example,
what is the dependence of the time on $N$? What if $N=500, 501, ... , 519,520$? From a plot of the
latter results, can you spot the prime numbers in this range? (Hints. The commands isprime and factor
may be useful. To get good tic/toc data it may help to compute each FFT 10 or 100 times in a loop.)\\\\

\begin{solution}\renewcommand{\qedsymbol}{}\ \\
    For $N$ in the first case, we have that the time for the FFT on $2\sin(3x+2)$ decreases repidly and
    then slowly increases. For $N$ in the second case, we have that the FFT on the same function
    countinuously decreses in terms of the computational time required.

\end{solution}

\newpage
\lstinputlisting{p3_5.m}
\newpage